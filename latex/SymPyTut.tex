
% Default to the notebook output style

    


% Inherit from the specified cell style.




    
\documentclass{article}

    
    
    \usepackage{graphicx} % Used to insert images
    \usepackage{adjustbox} % Used to constrain images to a maximum size 
    \usepackage{color} % Allow colors to be defined
    \usepackage{enumerate} % Needed for markdown enumerations to work
    \usepackage{geometry} % Used to adjust the document margins
    \usepackage{amsmath} % Equations
    \usepackage{amssymb} % Equations
    \usepackage{eurosym} % defines \euro
    \usepackage[mathletters]{ucs} % Extended unicode (utf-8) support
    \usepackage[utf8x]{inputenc} % Allow utf-8 characters in the tex document
    \usepackage{fancyvrb} % verbatim replacement that allows latex
    \usepackage{grffile} % extends the file name processing of package graphics 
                         % to support a larger range 
    % The hyperref package gives us a pdf with properly built
    % internal navigation ('pdf bookmarks' for the table of contents,
    % internal cross-reference links, web links for URLs, etc.)
    \usepackage{hyperref}
    \usepackage{longtable} % longtable support required by pandoc >1.10
    \usepackage{booktabs}  % table support for pandoc > 1.12.2
    

    
    
    \definecolor{orange}{cmyk}{0,0.4,0.8,0.2}
    \definecolor{darkorange}{rgb}{.71,0.21,0.01}
    \definecolor{darkgreen}{rgb}{.12,.54,.11}
    \definecolor{myteal}{rgb}{.26, .44, .56}
    \definecolor{gray}{gray}{0.45}
    \definecolor{lightgray}{gray}{.95}
    \definecolor{mediumgray}{gray}{.8}
    \definecolor{inputbackground}{rgb}{.95, .95, .85}
    \definecolor{outputbackground}{rgb}{.95, .95, .95}
    \definecolor{traceback}{rgb}{1, .95, .95}
    % ansi colors
    \definecolor{red}{rgb}{.6,0,0}
    \definecolor{green}{rgb}{0,.65,0}
    \definecolor{brown}{rgb}{0.6,0.6,0}
    \definecolor{blue}{rgb}{0,.145,.698}
    \definecolor{purple}{rgb}{.698,.145,.698}
    \definecolor{cyan}{rgb}{0,.698,.698}
    \definecolor{lightgray}{gray}{0.5}
    
    % bright ansi colors
    \definecolor{darkgray}{gray}{0.25}
    \definecolor{lightred}{rgb}{1.0,0.39,0.28}
    \definecolor{lightgreen}{rgb}{0.48,0.99,0.0}
    \definecolor{lightblue}{rgb}{0.53,0.81,0.92}
    \definecolor{lightpurple}{rgb}{0.87,0.63,0.87}
    \definecolor{lightcyan}{rgb}{0.5,1.0,0.83}
    
    % commands and environments needed by pandoc snippets
    % extracted from the output of `pandoc -s`
    \DefineVerbatimEnvironment{Highlighting}{Verbatim}{commandchars=\\\{\}}
    % Add ',fontsize=\small' for more characters per line
    \newenvironment{Shaded}{}{}
    \newcommand{\KeywordTok}[1]{\textcolor[rgb]{0.00,0.44,0.13}{\textbf{{#1}}}}
    \newcommand{\DataTypeTok}[1]{\textcolor[rgb]{0.56,0.13,0.00}{{#1}}}
    \newcommand{\DecValTok}[1]{\textcolor[rgb]{0.25,0.63,0.44}{{#1}}}
    \newcommand{\BaseNTok}[1]{\textcolor[rgb]{0.25,0.63,0.44}{{#1}}}
    \newcommand{\FloatTok}[1]{\textcolor[rgb]{0.25,0.63,0.44}{{#1}}}
    \newcommand{\CharTok}[1]{\textcolor[rgb]{0.25,0.44,0.63}{{#1}}}
    \newcommand{\StringTok}[1]{\textcolor[rgb]{0.25,0.44,0.63}{{#1}}}
    \newcommand{\CommentTok}[1]{\textcolor[rgb]{0.38,0.63,0.69}{\textit{{#1}}}}
    \newcommand{\OtherTok}[1]{\textcolor[rgb]{0.00,0.44,0.13}{{#1}}}
    \newcommand{\AlertTok}[1]{\textcolor[rgb]{1.00,0.00,0.00}{\textbf{{#1}}}}
    \newcommand{\FunctionTok}[1]{\textcolor[rgb]{0.02,0.16,0.49}{{#1}}}
    \newcommand{\RegionMarkerTok}[1]{{#1}}
    \newcommand{\ErrorTok}[1]{\textcolor[rgb]{1.00,0.00,0.00}{\textbf{{#1}}}}
    \newcommand{\NormalTok}[1]{{#1}}
    
    % Define a nice break command that doesn't care if a line doesn't already
    % exist.
    \def\br{\hspace*{\fill} \\* }
    % Math Jax compatability definitions
    \def\gt{>}
    \def\lt{<}
    % Document parameters
    \title{SymPyTut}
    
    
    

    % Pygments definitions
    
\makeatletter
\def\PY@reset{\let\PY@it=\relax \let\PY@bf=\relax%
    \let\PY@ul=\relax \let\PY@tc=\relax%
    \let\PY@bc=\relax \let\PY@ff=\relax}
\def\PY@tok#1{\csname PY@tok@#1\endcsname}
\def\PY@toks#1+{\ifx\relax#1\empty\else%
    \PY@tok{#1}\expandafter\PY@toks\fi}
\def\PY@do#1{\PY@bc{\PY@tc{\PY@ul{%
    \PY@it{\PY@bf{\PY@ff{#1}}}}}}}
\def\PY#1#2{\PY@reset\PY@toks#1+\relax+\PY@do{#2}}

\def\PY@tok@gd{\def\PY@tc##1{\textcolor[rgb]{0.63,0.00,0.00}{##1}}}
\def\PY@tok@gu{\let\PY@bf=\textbf\def\PY@tc##1{\textcolor[rgb]{0.50,0.00,0.50}{##1}}}
\def\PY@tok@gt{\def\PY@tc##1{\textcolor[rgb]{0.00,0.25,0.82}{##1}}}
\def\PY@tok@gs{\let\PY@bf=\textbf}
\def\PY@tok@gr{\def\PY@tc##1{\textcolor[rgb]{1.00,0.00,0.00}{##1}}}
\def\PY@tok@cm{\let\PY@it=\textit\def\PY@tc##1{\textcolor[rgb]{0.25,0.50,0.50}{##1}}}
\def\PY@tok@vg{\def\PY@tc##1{\textcolor[rgb]{0.10,0.09,0.49}{##1}}}
\def\PY@tok@m{\def\PY@tc##1{\textcolor[rgb]{0.40,0.40,0.40}{##1}}}
\def\PY@tok@mh{\def\PY@tc##1{\textcolor[rgb]{0.40,0.40,0.40}{##1}}}
\def\PY@tok@go{\def\PY@tc##1{\textcolor[rgb]{0.50,0.50,0.50}{##1}}}
\def\PY@tok@ge{\let\PY@it=\textit}
\def\PY@tok@vc{\def\PY@tc##1{\textcolor[rgb]{0.10,0.09,0.49}{##1}}}
\def\PY@tok@il{\def\PY@tc##1{\textcolor[rgb]{0.40,0.40,0.40}{##1}}}
\def\PY@tok@cs{\let\PY@it=\textit\def\PY@tc##1{\textcolor[rgb]{0.25,0.50,0.50}{##1}}}
\def\PY@tok@cp{\def\PY@tc##1{\textcolor[rgb]{0.74,0.48,0.00}{##1}}}
\def\PY@tok@gi{\def\PY@tc##1{\textcolor[rgb]{0.00,0.63,0.00}{##1}}}
\def\PY@tok@gh{\let\PY@bf=\textbf\def\PY@tc##1{\textcolor[rgb]{0.00,0.00,0.50}{##1}}}
\def\PY@tok@ni{\let\PY@bf=\textbf\def\PY@tc##1{\textcolor[rgb]{0.60,0.60,0.60}{##1}}}
\def\PY@tok@nl{\def\PY@tc##1{\textcolor[rgb]{0.63,0.63,0.00}{##1}}}
\def\PY@tok@nn{\let\PY@bf=\textbf\def\PY@tc##1{\textcolor[rgb]{0.00,0.00,1.00}{##1}}}
\def\PY@tok@no{\def\PY@tc##1{\textcolor[rgb]{0.53,0.00,0.00}{##1}}}
\def\PY@tok@na{\def\PY@tc##1{\textcolor[rgb]{0.49,0.56,0.16}{##1}}}
\def\PY@tok@nb{\def\PY@tc##1{\textcolor[rgb]{0.00,0.50,0.00}{##1}}}
\def\PY@tok@nc{\let\PY@bf=\textbf\def\PY@tc##1{\textcolor[rgb]{0.00,0.00,1.00}{##1}}}
\def\PY@tok@nd{\def\PY@tc##1{\textcolor[rgb]{0.67,0.13,1.00}{##1}}}
\def\PY@tok@ne{\let\PY@bf=\textbf\def\PY@tc##1{\textcolor[rgb]{0.82,0.25,0.23}{##1}}}
\def\PY@tok@nf{\def\PY@tc##1{\textcolor[rgb]{0.00,0.00,1.00}{##1}}}
\def\PY@tok@si{\let\PY@bf=\textbf\def\PY@tc##1{\textcolor[rgb]{0.73,0.40,0.53}{##1}}}
\def\PY@tok@s2{\def\PY@tc##1{\textcolor[rgb]{0.73,0.13,0.13}{##1}}}
\def\PY@tok@vi{\def\PY@tc##1{\textcolor[rgb]{0.10,0.09,0.49}{##1}}}
\def\PY@tok@nt{\let\PY@bf=\textbf\def\PY@tc##1{\textcolor[rgb]{0.00,0.50,0.00}{##1}}}
\def\PY@tok@nv{\def\PY@tc##1{\textcolor[rgb]{0.10,0.09,0.49}{##1}}}
\def\PY@tok@s1{\def\PY@tc##1{\textcolor[rgb]{0.73,0.13,0.13}{##1}}}
\def\PY@tok@sh{\def\PY@tc##1{\textcolor[rgb]{0.73,0.13,0.13}{##1}}}
\def\PY@tok@sc{\def\PY@tc##1{\textcolor[rgb]{0.73,0.13,0.13}{##1}}}
\def\PY@tok@sx{\def\PY@tc##1{\textcolor[rgb]{0.00,0.50,0.00}{##1}}}
\def\PY@tok@bp{\def\PY@tc##1{\textcolor[rgb]{0.00,0.50,0.00}{##1}}}
\def\PY@tok@c1{\let\PY@it=\textit\def\PY@tc##1{\textcolor[rgb]{0.25,0.50,0.50}{##1}}}
\def\PY@tok@kc{\let\PY@bf=\textbf\def\PY@tc##1{\textcolor[rgb]{0.00,0.50,0.00}{##1}}}
\def\PY@tok@c{\let\PY@it=\textit\def\PY@tc##1{\textcolor[rgb]{0.25,0.50,0.50}{##1}}}
\def\PY@tok@mf{\def\PY@tc##1{\textcolor[rgb]{0.40,0.40,0.40}{##1}}}
\def\PY@tok@err{\def\PY@bc##1{\fcolorbox[rgb]{1.00,0.00,0.00}{1,1,1}{##1}}}
\def\PY@tok@kd{\let\PY@bf=\textbf\def\PY@tc##1{\textcolor[rgb]{0.00,0.50,0.00}{##1}}}
\def\PY@tok@ss{\def\PY@tc##1{\textcolor[rgb]{0.10,0.09,0.49}{##1}}}
\def\PY@tok@sr{\def\PY@tc##1{\textcolor[rgb]{0.73,0.40,0.53}{##1}}}
\def\PY@tok@mo{\def\PY@tc##1{\textcolor[rgb]{0.40,0.40,0.40}{##1}}}
\def\PY@tok@kn{\let\PY@bf=\textbf\def\PY@tc##1{\textcolor[rgb]{0.00,0.50,0.00}{##1}}}
\def\PY@tok@mi{\def\PY@tc##1{\textcolor[rgb]{0.40,0.40,0.40}{##1}}}
\def\PY@tok@gp{\let\PY@bf=\textbf\def\PY@tc##1{\textcolor[rgb]{0.00,0.00,0.50}{##1}}}
\def\PY@tok@o{\def\PY@tc##1{\textcolor[rgb]{0.40,0.40,0.40}{##1}}}
\def\PY@tok@kr{\let\PY@bf=\textbf\def\PY@tc##1{\textcolor[rgb]{0.00,0.50,0.00}{##1}}}
\def\PY@tok@s{\def\PY@tc##1{\textcolor[rgb]{0.73,0.13,0.13}{##1}}}
\def\PY@tok@kp{\def\PY@tc##1{\textcolor[rgb]{0.00,0.50,0.00}{##1}}}
\def\PY@tok@w{\def\PY@tc##1{\textcolor[rgb]{0.73,0.73,0.73}{##1}}}
\def\PY@tok@kt{\def\PY@tc##1{\textcolor[rgb]{0.69,0.00,0.25}{##1}}}
\def\PY@tok@ow{\let\PY@bf=\textbf\def\PY@tc##1{\textcolor[rgb]{0.67,0.13,1.00}{##1}}}
\def\PY@tok@sb{\def\PY@tc##1{\textcolor[rgb]{0.73,0.13,0.13}{##1}}}
\def\PY@tok@k{\let\PY@bf=\textbf\def\PY@tc##1{\textcolor[rgb]{0.00,0.50,0.00}{##1}}}
\def\PY@tok@se{\let\PY@bf=\textbf\def\PY@tc##1{\textcolor[rgb]{0.73,0.40,0.13}{##1}}}
\def\PY@tok@sd{\let\PY@it=\textit\def\PY@tc##1{\textcolor[rgb]{0.73,0.13,0.13}{##1}}}

\def\PYZbs{\char`\\}
\def\PYZus{\char`\_}
\def\PYZob{\char`\{}
\def\PYZcb{\char`\}}
\def\PYZca{\char`\^}
% for compatibility with earlier versions
\def\PYZat{@}
\def\PYZlb{[}
\def\PYZrb{]}
\makeatother


    % Exact colors from NB
    \definecolor{incolor}{rgb}{0.0, 0.0, 0.5}
    \definecolor{outcolor}{rgb}{0.545, 0.0, 0.0}



    
    % Prevent overflowing lines due to hard-to-break entities
    \sloppy 
    % Setup hyperref package
    \hypersetup{
      breaklinks=true,  % so long urls are correctly broken across lines
      colorlinks=true,
      urlcolor=blue,
      linkcolor=darkorange,
      citecolor=darkgreen,
      }
    % Slightly bigger margins than the latex defaults
    
    \geometry{verbose,tmargin=1in,bmargin=1in,lmargin=1in,rmargin=1in}
    
    

    \begin{document}
    
    
    \maketitle
    
    

    
    \section{Taming math and physics using
\texttt{SymPy}}\label{taming-math-and-physics-using-sympy}

    Tutorial based on the \href{http://minireference.com/}{No bullshit
guide} series of textbooks by
\href{mailto:ivan.savov+SYMPYTUT@gmail.com}{Ivan Savov}

    \subsection{Abstract}\label{abstract}

    Most people consider math and physics to be scary beasts from which it
is best to keep one's distance. Computers, however, can help us tame the
complexity and tedious arithmetic manipulations associated with these
subjects. Indeed, math and physics are much more approachable once you
have the power of computers on your side.

This tutorial serves a dual purpose. On one hand, it serves as a review
of the fundamental concepts of mathematics for computer-literate people.
On the other hand, this tutorial serves to demonstrate to students how a
computer algebra system can help them with their classwork. A word of
warning is in order. Please don't use \texttt{SymPy} to avoid the
suffering associated with your homework! Teachers assign homework
problems to you because they want you to learn. Do your homework by
hand, but if you want, you can check your answers using \texttt{SymPy}.
Better yet, use \texttt{SymPy} to invent extra practice problems for
yourself.

    \subsection{Contents}\label{contents}

    \begin{itemize}
\itemsep1pt\parskip0pt\parsep0pt
\item
  \hyperref[Fundamentals-of-mathematics]{Fundamentals of mathematics}
\item
  \hyperref[Complex-numbers]{Complex numbers}
\item
  \hyperref[Calculus]{Calculus}
\item
  \hyperref[Vectors]{Vectors}
\item
  \hyperref[Mechanics]{Mechanics}
\item
  \hyperref[Linear-algebra]{Linear algebra}
\end{itemize}

    \subsection{Introduction}\label{introduction}

    You can use a computer algebra system (CAS) to compute complicated math
expressions, solve equations, perform calculus procedures, and simulate
physics systems.

All computer algebra systems offer essentially the same functionality,
so it doesn't matter which system you use: there are free systems like
\texttt{SymPy}, \texttt{Magma}, or \texttt{Octave}, and commercial
systems like \texttt{Maple}, \texttt{MATLAB}, and \texttt{Mathematica}.
This tutorial is an introduction to \texttt{SymPy}, which is a
\emph{symbolic} computer algebra system written in the programming
language \texttt{Python}. In a symbolic CAS, numbers and operations are
represented symbolically, so the answers obtained are exact. For
example, the number √2 is represented in \texttt{SymPy} as the object
\texttt{Pow(2,1/2)}, whereas in numerical computer algebra systems like
\texttt{Octave}, the number √2 is represented as the approximation
1.41421356237310 (a \texttt{float}). For most purposes the approximation
is okay, but sometimes approximations can lead to problems:
\texttt{float(sqrt(2))*float(sqrt(2))} = 2.00000000000000044 ≠ 2.
Because \texttt{SymPy} uses exact representations, you'll never run into
such problems: \texttt{Pow(2,1/2)*Pow(2,1/2)} = 2.

This tutorial is organized as follows. We'll begin by introducing the
\texttt{SymPy} basics and the bread-and-butter functions used for
manipulating expressions and solving equations. Afterward, we'll discuss
the \texttt{SymPy} functions that implement calculus operations like
differentiation and integration. We'll also introduce the functions used
to deal with vectors and complex numbers. Later we'll see how to use
vectors and integrals to understand Newtonian mechanics. In the last
section, we'll introduce the linear algebra functions available in
\texttt{SymPy}.

This tutorial presents many explanations as blocks of code. Be sure to
try the code examples on your own by typing the commands into
\texttt{SymPy}. It's always important to verify for yourself!

    \subsection{Using SymPy}\label{using-sympy}

    The easiest way to use \texttt{SymPy}, provided you're connected to the
Internet, is to visit http://live.sympy.org. You'll be presented with an
interactive prompt into which you can enter your commands---right in
your browser.

If you want to use \texttt{SymPy} on your own computer, you must install
\texttt{Python} and the python package \texttt{sympy}. You can then open
a command prompt and start a \texttt{SymPy} session using:

\begin{verbatim}
you@host$ python
Python X.Y.Z
[GCC a.b.c (Build Info)] on platform
Type "help", "copyright", or "license" for more information.
>>> from sympy import *
>>>
\end{verbatim}

The \texttt{\textgreater{}\textgreater{}\textgreater{}} prompt indicates
you're in the Python shell which accepts Python commands. The command
\texttt{from sympy import *} imports all the \texttt{SymPy} functions
into the current namespace. All \texttt{SymPy} functions are now
available to you. To exit the python shell press \texttt{CTRL+D}.

I highly recommend you also install \texttt{ipython}, which is an
improved interactive python shell. If you have \texttt{ipython} and
\texttt{SymPy} installed, you can start an \texttt{ipython} shell with
\texttt{SymPy} pre-imported using the command \texttt{isympy}. For an
even better experience, you can try \texttt{ipython notebook}, which is
a web frontend for the \texttt{ipython} shell.

You can start your session the same way as \texttt{isympy} do, by
running following commands, which will be detaily described latter.

    \begin{Verbatim}[commandchars=\\\{\}]
{\color{incolor}In [{\color{incolor}1}]:} \PY{k+kn}{from} \PY{n+nn}{sympy} \PY{k}{import} \PY{n}{init\PYZus{}session}
        \PY{n}{init\PYZus{}session}\PY{p}{(}\PY{p}{)}
\end{Verbatim}

    \begin{Verbatim}[commandchars=\\\{\}]
IPython console for SymPy 0.7.6 (Python 3.4.2-64-bit) (ground types: gmpy)

These commands were executed:
>>> from \_\_future\_\_ import division
>>> from sympy import *
>>> x, y, z, t = symbols('x y z t')
>>> k, m, n = symbols('k m n', integer=True)
>>> f, g, h = symbols('f g h', cls=Function)
>>> init\_printing()

Documentation can be found at http://www.sympy.org
    \end{Verbatim}

    \subsection{Fundamentals of
mathematics}\label{fundamentals-of-mathematics}

    Let's begin by learning about the basic \texttt{SymPy} objects and the
operations we can carry out on them. We'll learn the \texttt{SymPy}
equivalents of many math verbs like ``to solve'' (an equation), ``to
expand'' (an expression), ``to factor'' (a polynomial).

    \subsubsection{Numbers}\label{numbers}

    In \texttt{Python}, there are two types of number objects: \texttt{int}s
and \texttt{float}s.

    \begin{Verbatim}[commandchars=\\\{\}]
{\color{incolor}In [{\color{incolor}2}]:} \PY{l+m+mi}{3}         \PY{c}{# an int}
\end{Verbatim}
\texttt{\color{outcolor}Out[{\color{outcolor}2}]:}
    
    
        \begin{equation*}\adjustbox{max width=\hsize}{$
        3
        $}\end{equation*}

    

    \begin{Verbatim}[commandchars=\\\{\}]
{\color{incolor}In [{\color{incolor}3}]:} \PY{l+m+mf}{3.0}       \PY{c}{# a float}
\end{Verbatim}
\texttt{\color{outcolor}Out[{\color{outcolor}3}]:}
    
    
        \begin{equation*}\adjustbox{max width=\hsize}{$
        3.0
        $}\end{equation*}

    

    Integer objects in \texttt{Python} are a faithful representation of the
set of integers $\mathbb{Z}=\{\ldots,-2,-1,0,1,2,\ldots\}$. Floating
point numbers are approximate representations of the reals $\mathbb{R}$.
Regardless of its absolute size, a floating point number is only
accurate to 16 decimals.

Special care is required when specifying rational numbers, because
integer division might not produce the answer you want. In other words,
Python will not automatically convert the answer to a floating point
number, but instead round the answer to the closest integer:

    \begin{Verbatim}[commandchars=\\\{\}]
{\color{incolor}In [{\color{incolor}4}]:} \PY{l+m+mi}{1}\PY{o}{/}\PY{l+m+mi}{7}       \PY{c}{# int/int gives int}
\end{Verbatim}
\texttt{\color{outcolor}Out[{\color{outcolor}4}]:}
    
    
        \begin{equation*}\adjustbox{max width=\hsize}{$
        0.14285714285714285
        $}\end{equation*}

    

    To avoid this problem, you can force \texttt{float} division by using
the number \texttt{1.0} instead of \texttt{1}:

    \begin{Verbatim}[commandchars=\\\{\}]
{\color{incolor}In [{\color{incolor}5}]:} \PY{l+m+mf}{1.0}\PY{o}{/}\PY{l+m+mi}{7}     \PY{c}{# float/int gives float}
\end{Verbatim}
\texttt{\color{outcolor}Out[{\color{outcolor}5}]:}
    
    
        \begin{equation*}\adjustbox{max width=\hsize}{$
        0.14285714285714285
        $}\end{equation*}

    

    This result is better, but it's still only an approximation of the exact
number $\frac{1}{7} \in \mathbb{Q}$, since a \texttt{float} has 16
decimals while the decimal expansion of $\frac{1}{7}$ is infinitely
long. To obtain an \emph{exact} representation of $\frac{1}{7}$ you need
to create a \texttt{SymPy} expression. You can sympify any expression
using the shortcut function \texttt{S()}:

    \begin{Verbatim}[commandchars=\\\{\}]
{\color{incolor}In [{\color{incolor}6}]:} \PY{n}{S}\PY{p}{(}\PY{l+s}{'}\PY{l+s}{1/7}\PY{l+s}{'}\PY{p}{)}  \PY{c}{# = Rational(1,7)}
\end{Verbatim}
\texttt{\color{outcolor}Out[{\color{outcolor}6}]:}
    
    
        \begin{equation*}\adjustbox{max width=\hsize}{$
        \frac{1}{7}
        $}\end{equation*}

    

    Note the input to \texttt{S()} is specified as a text string delimited
by quotes. We could have achieved the same result using
\texttt{S('1')/7} since a \texttt{SymPy} object divided by an
\texttt{int} is a \texttt{SymPy} object.

Except for the tricky \texttt{Python} division operator, other math
operators like addition \texttt{+}, subtraction \texttt{-}, and
multiplication \texttt{*} work as you would expect. The syntax
\texttt{**} is used in \texttt{Python} to denote exponentiation:

    \begin{Verbatim}[commandchars=\\\{\}]
{\color{incolor}In [{\color{incolor}7}]:} \PY{l+m+mi}{2}\PY{o}{*}\PY{o}{*}\PY{l+m+mi}{10}     \PY{c}{# same as S('2\PYZca{}10')}
\end{Verbatim}
\texttt{\color{outcolor}Out[{\color{outcolor}7}]:}
    
    
        \begin{equation*}\adjustbox{max width=\hsize}{$
        1024
        $}\end{equation*}

    

    When solving math problems, it's best to work with \texttt{SymPy}
objects, and wait to compute the numeric answer in the end. To obtain a
numeric approximation of a \texttt{SymPy} object as a \texttt{float},
call its \texttt{.evalf()} method:

    \begin{Verbatim}[commandchars=\\\{\}]
{\color{incolor}In [{\color{incolor}8}]:} \PY{n}{pi}
\end{Verbatim}
\texttt{\color{outcolor}Out[{\color{outcolor}8}]:}
    
    
        \begin{equation*}\adjustbox{max width=\hsize}{$
        \pi
        $}\end{equation*}

    

    \begin{Verbatim}[commandchars=\\\{\}]
{\color{incolor}In [{\color{incolor}9}]:} \PY{n}{pi}\PY{o}{.}\PY{n}{evalf}\PY{p}{(}\PY{p}{)}
\end{Verbatim}
\texttt{\color{outcolor}Out[{\color{outcolor}9}]:}
    
    
        \begin{equation*}\adjustbox{max width=\hsize}{$
        3.14159265358979
        $}\end{equation*}

    

    The method \texttt{.n()} is equivalent to \texttt{.evalf()}. The global
\texttt{SymPy} function \texttt{N()} can also be used to to compute
numerical values. You can easily change the number of digits of
precision of the approximation. Enter \texttt{pi.n(400)} to obtain an
approximation of $\pi$ to 400 decimals.

    \subsubsection{Symbols}\label{symbols}

    Python is a civilized language so there's no need to define variables
before assigning values to them. When you write \texttt{a = 3}, you
define a new name \texttt{a} and set it to the value \texttt{3}. You can
now use the name \texttt{a} in subsequent calculations.

Most interesting \texttt{SymPy} calculations require us to define
\texttt{symbols}, which are the \texttt{SymPy} objects for representing
variables and unknowns. For your convenience, when
\href{http://live.sympy.org}{live.sympy.org} starts, it runs the
following commands automatically:

    \begin{Verbatim}[commandchars=\\\{\}]
{\color{incolor}In [{\color{incolor}10}]:} \PY{k+kn}{from} \PY{n+nn}{\PYZus{}\PYZus{}future\PYZus{}\PYZus{}} \PY{k}{import} \PY{n}{division}
         \PY{k+kn}{from} \PY{n+nn}{sympy} \PY{k}{import} \PY{o}{*}
         \PY{n}{x}\PY{p}{,} \PY{n}{y}\PY{p}{,} \PY{n}{z}\PY{p}{,} \PY{n}{t} \PY{o}{=} \PY{n}{symbols}\PY{p}{(}\PY{l+s}{'}\PY{l+s}{x y z t}\PY{l+s}{'}\PY{p}{)}
         \PY{n}{k}\PY{p}{,} \PY{n}{m}\PY{p}{,} \PY{n}{n} \PY{o}{=} \PY{n}{symbols}\PY{p}{(}\PY{l+s}{'}\PY{l+s}{k m n}\PY{l+s}{'}\PY{p}{,} \PY{n}{integer}\PY{o}{=}\PY{k}{True}\PY{p}{)}
         \PY{n}{f}\PY{p}{,} \PY{n}{g}\PY{p}{,} \PY{n}{h} \PY{o}{=} \PY{n}{symbols}\PY{p}{(}\PY{l+s}{'}\PY{l+s}{f g h}\PY{l+s}{'}\PY{p}{,} \PY{n}{cls}\PY{o}{=}\PY{n}{Function}\PY{p}{)}
\end{Verbatim}

    The first statement instructs python to convert \texttt{1/7} to
\texttt{1.0/7} when dividing, potentially saving you from any int
division confusion. The second statement imports all the \texttt{SymPy}
functions. The remaining statements define some generic symbols
\texttt{x}, \texttt{y}, \texttt{z}, and \texttt{t}, and several other
symbols with special properties.

Note the difference between the following two statements:

    \begin{Verbatim}[commandchars=\\\{\}]
{\color{incolor}In [{\color{incolor}11}]:} \PY{n}{x} \PY{o}{+} \PY{l+m+mi}{2}            \PY{c}{# an Add expression}
\end{Verbatim}
\texttt{\color{outcolor}Out[{\color{outcolor}11}]:}
    
    
        \begin{equation*}\adjustbox{max width=\hsize}{$
        x + 2
        $}\end{equation*}

    

    \begin{Verbatim}[commandchars=\\\{\}]
{\color{incolor}In [{\color{incolor}12}]:} \PY{n}{p} \PY{o}{+} \PY{l+m+mi}{2}
\end{Verbatim}

    \begin{Verbatim}[commandchars=\\\{\}]

        ---------------------------------------------------------------------------
    NameError                                 Traceback (most recent call last)

        <ipython-input-12-d62eaef0cf31> in <module>()
    ----> 1 p + 2
    

        NameError: name 'p' is not defined

    \end{Verbatim}

    The name \texttt{x} is defined as a symbol, so \texttt{SymPy} knows that
\texttt{x + 2} is an expression; but the variable \texttt{p} is not
defined, so \texttt{SymPy} doesn't know what to make of \texttt{p + 2}.
To use \texttt{p} in expressions, you must first define it as a symbol:

    \begin{Verbatim}[commandchars=\\\{\}]
{\color{incolor}In [{\color{incolor}13}]:} \PY{n}{p} \PY{o}{=} \PY{n}{Symbol}\PY{p}{(}\PY{l+s}{'}\PY{l+s}{p}\PY{l+s}{'}\PY{p}{)}  \PY{c}{# the same as p = symbols('p')}
         \PY{n}{p} \PY{o}{+} \PY{l+m+mi}{2}            \PY{c}{# = Add(Symbol('p'), Integer(2))}
\end{Verbatim}
\texttt{\color{outcolor}Out[{\color{outcolor}13}]:}
    
    
        \begin{equation*}\adjustbox{max width=\hsize}{$
        p + 2
        $}\end{equation*}

    

    You can define a sequence of variables using the following notation:

    \begin{Verbatim}[commandchars=\\\{\}]
{\color{incolor}In [{\color{incolor}14}]:} \PY{n}{a0}\PY{p}{,} \PY{n}{a1}\PY{p}{,} \PY{n}{a2}\PY{p}{,} \PY{n}{a3} \PY{o}{=} \PY{n}{symbols}\PY{p}{(}\PY{l+s}{'}\PY{l+s}{a0:4}\PY{l+s}{'}\PY{p}{)}
\end{Verbatim}

    You can use any name you want for a variable, but it's best if you avoid
the letters \texttt{Q,C,O,S,I,N} and \texttt{E} because they have
special uses in \texttt{SymPy}: \texttt{I} is the unit imaginary number
$i \equiv \sqrt(-1)$, \texttt{E} is the base of the natural logarithm,
\texttt{S()} is the sympify function, \texttt{N()} is used to obtain
numeric approximations, and \texttt{O} is used for big-O notation.

The underscore symbol \texttt{\_} is a special variable that contains
the result of the last printed value. The variable \texttt{\_} is
analogous to the \texttt{ans} button on certain calculators, and is
useful in multi-step calculations:

    \begin{Verbatim}[commandchars=\\\{\}]
{\color{incolor}In [{\color{incolor}15}]:} \PY{l+m+mi}{3}\PY{o}{+}\PY{l+m+mi}{3}
\end{Verbatim}
\texttt{\color{outcolor}Out[{\color{outcolor}15}]:}
    
    
        \begin{equation*}\adjustbox{max width=\hsize}{$
        6
        $}\end{equation*}

    

    \begin{Verbatim}[commandchars=\\\{\}]
{\color{incolor}In [{\color{incolor}16}]:} \PY{n}{\PYZus{}}\PY{o}{*}\PY{l+m+mi}{2}
\end{Verbatim}
\texttt{\color{outcolor}Out[{\color{outcolor}16}]:}
    
    
        \begin{equation*}\adjustbox{max width=\hsize}{$
        12
        $}\end{equation*}

    

    \subsubsection{Expresions}\label{expresions}

    You define \texttt{SymPy} expressions by combining symbols with basic
math operations and other functions:

    \begin{Verbatim}[commandchars=\\\{\}]
{\color{incolor}In [{\color{incolor}17}]:} \PY{n}{expr} \PY{o}{=} \PY{l+m+mi}{2}\PY{o}{*}\PY{n}{x} \PY{o}{+} \PY{l+m+mi}{3}\PY{o}{*}\PY{n}{x} \PY{o}{-} \PY{n}{sin}\PY{p}{(}\PY{n}{x}\PY{p}{)} \PY{o}{-} \PY{l+m+mi}{3}\PY{o}{*}\PY{n}{x} \PY{o}{+} \PY{l+m+mi}{42}
         \PY{n}{simplify}\PY{p}{(}\PY{n}{expr}\PY{p}{)}
\end{Verbatim}
\texttt{\color{outcolor}Out[{\color{outcolor}17}]:}
    
    
        \begin{equation*}\adjustbox{max width=\hsize}{$
        2 x - \sin{\left (x \right )} + 42
        $}\end{equation*}

    

    The function \texttt{simplify} can be used on any expression to simplify
it. The examples below illustrate other useful \texttt{SymPy} functions
that correspond to common mathematical operations on expressions:

    \begin{Verbatim}[commandchars=\\\{\}]
{\color{incolor}In [{\color{incolor}18}]:} \PY{n}{factor}\PY{p}{(} \PY{n}{x}\PY{o}{*}\PY{o}{*}\PY{l+m+mi}{2}\PY{o}{-}\PY{l+m+mi}{2}\PY{o}{*}\PY{n}{x}\PY{o}{-}\PY{l+m+mi}{8} \PY{p}{)}
\end{Verbatim}
\texttt{\color{outcolor}Out[{\color{outcolor}18}]:}
    
    
        \begin{equation*}\adjustbox{max width=\hsize}{$
        \left(x - 4\right) \left(x + 2\right)
        $}\end{equation*}

    

    \begin{Verbatim}[commandchars=\\\{\}]
{\color{incolor}In [{\color{incolor}19}]:} \PY{n}{expand}\PY{p}{(} \PY{p}{(}\PY{n}{x}\PY{o}{-}\PY{l+m+mi}{4}\PY{p}{)}\PY{o}{*}\PY{p}{(}\PY{n}{x}\PY{o}{+}\PY{l+m+mi}{2}\PY{p}{)} \PY{p}{)}
\end{Verbatim}
\texttt{\color{outcolor}Out[{\color{outcolor}19}]:}
    
    
        \begin{equation*}\adjustbox{max width=\hsize}{$
        x^{2} - 2 x - 8
        $}\end{equation*}

    

    \begin{Verbatim}[commandchars=\\\{\}]
{\color{incolor}In [{\color{incolor}20}]:} \PY{n}{a}\PY{p}{,} \PY{n}{b} \PY{o}{=} \PY{n}{symbols}\PY{p}{(}\PY{l+s}{'}\PY{l+s}{a b}\PY{l+s}{'}\PY{p}{)}
         \PY{n}{collect}\PY{p}{(}\PY{n}{x}\PY{o}{*}\PY{o}{*}\PY{l+m+mi}{2} \PY{o}{+} \PY{n}{x}\PY{o}{*}\PY{n}{b} \PY{o}{+} \PY{n}{a}\PY{o}{*}\PY{n}{x} \PY{o}{+} \PY{n}{a}\PY{o}{*}\PY{n}{b}\PY{p}{,} \PY{n}{x}\PY{p}{)}  \PY{c}{# collect terms for diff. pows of x}
\end{Verbatim}
\texttt{\color{outcolor}Out[{\color{outcolor}20}]:}
    
    
        \begin{equation*}\adjustbox{max width=\hsize}{$
        a b + x^{2} + x \left(a + b\right)
        $}\end{equation*}

    

    To substitute a given value into an expression, call the
\texttt{.subs()} method, passing in a python dictionary object
\texttt{\{ key:val, ... \}} with the symbol--value substitutions you
want to make:

    \begin{Verbatim}[commandchars=\\\{\}]
{\color{incolor}In [{\color{incolor}21}]:} \PY{n}{expr} \PY{o}{=} \PY{n}{sin}\PY{p}{(}\PY{n}{x}\PY{p}{)} \PY{o}{+} \PY{n}{cos}\PY{p}{(}\PY{n}{y}\PY{p}{)}
         \PY{n}{expr}
\end{Verbatim}
\texttt{\color{outcolor}Out[{\color{outcolor}21}]:}
    
    
        \begin{equation*}\adjustbox{max width=\hsize}{$
        \sin{\left (x \right )} + \cos{\left (y \right )}
        $}\end{equation*}

    

    \begin{Verbatim}[commandchars=\\\{\}]
{\color{incolor}In [{\color{incolor}22}]:} \PY{n}{expr}\PY{o}{.}\PY{n}{subs}\PY{p}{(}\PY{p}{\PYZob{}}\PY{n}{x}\PY{p}{:}\PY{l+m+mi}{1}\PY{p}{,} \PY{n}{y}\PY{p}{:}\PY{l+m+mi}{2}\PY{p}{\PYZcb{}}\PY{p}{)}
\end{Verbatim}
\texttt{\color{outcolor}Out[{\color{outcolor}22}]:}
    
    
        \begin{equation*}\adjustbox{max width=\hsize}{$
        \cos{\left (2 \right )} + \sin{\left (1 \right )}
        $}\end{equation*}

    

    \begin{Verbatim}[commandchars=\\\{\}]
{\color{incolor}In [{\color{incolor}23}]:} \PY{n}{expr}\PY{o}{.}\PY{n}{subs}\PY{p}{(}\PY{p}{\PYZob{}}\PY{n}{x}\PY{p}{:}\PY{l+m+mi}{1}\PY{p}{,} \PY{n}{y}\PY{p}{:}\PY{l+m+mi}{2}\PY{p}{\PYZcb{}}\PY{p}{)}\PY{o}{.}\PY{n}{n}\PY{p}{(}\PY{p}{)}
\end{Verbatim}
\texttt{\color{outcolor}Out[{\color{outcolor}23}]:}
    
    
        \begin{equation*}\adjustbox{max width=\hsize}{$
        0.425324148260754
        $}\end{equation*}

    

    Note how we used \texttt{.n()} to obtain the expression's numeric value.

    \subsubsection{Solving equations}\label{solving-equations}

    The function \texttt{solve} is the main workhorse in \texttt{SymPy}.
This incredibly powerful function knows how to solve all kinds of
equations. In fact \texttt{solve} can solve pretty much any equation!
When high school students learn about this function, they get really
angry---why did they spend five years of their life learning to solve
various equations by hand, when all along there was this \texttt{solve}
thing that could do all the math for them? Don't worry, learning math is
never a waste of time.

The function \texttt{solve} takes two arguments. Use
\texttt{solve(expr,var)} to solve the equation \texttt{expr==0} for the
variable \texttt{var}. You can rewrite any equation in the form
\texttt{expr==0} by moving all the terms to one side of the equation;
the solutions to $A(x) = B(x)$ are the same as the solutions to
$A(x) - B(x) = 0$.

For example, to solve the quadratic equation $x^2 + 2x - 8 = 0$, use

    \begin{Verbatim}[commandchars=\\\{\}]
{\color{incolor}In [{\color{incolor}24}]:} \PY{n}{solve}\PY{p}{(} \PY{n}{x}\PY{o}{*}\PY{o}{*}\PY{l+m+mi}{2} \PY{o}{+} \PY{l+m+mi}{2}\PY{o}{*}\PY{n}{x} \PY{o}{-} \PY{l+m+mi}{8}\PY{p}{,} \PY{n}{x}\PY{p}{)}
\end{Verbatim}
\texttt{\color{outcolor}Out[{\color{outcolor}24}]:}
    
    
        \begin{equation*}\adjustbox{max width=\hsize}{$
        \left [ -4, \quad 2\right ]
        $}\end{equation*}

    

    In this case the equation has two solutions so \texttt{solve} returns a
list. Check that $x = 2$ and $x = -4$ satisfy the equation
$x^2 + 2x - 8 = 0$.

The best part about \texttt{solve} and \texttt{SymPy} is that you can
obtain symbolic answers when solving equations. Instead of solving one
specific quadratic equation, we can solve all possible equations of the
form $ax^2 + bx + c = 0$ using the following steps:

    \begin{Verbatim}[commandchars=\\\{\}]
{\color{incolor}In [{\color{incolor}25}]:} \PY{n}{a}\PY{p}{,} \PY{n}{b}\PY{p}{,} \PY{n}{c} \PY{o}{=} \PY{n}{symbols}\PY{p}{(}\PY{l+s}{'}\PY{l+s}{a b c}\PY{l+s}{'}\PY{p}{)}
         \PY{n}{solve}\PY{p}{(} \PY{n}{a}\PY{o}{*}\PY{n}{x}\PY{o}{*}\PY{o}{*}\PY{l+m+mi}{2} \PY{o}{+} \PY{n}{b}\PY{o}{*}\PY{n}{x} \PY{o}{+} \PY{n}{c}\PY{p}{,} \PY{n}{x}\PY{p}{)}
\end{Verbatim}
\texttt{\color{outcolor}Out[{\color{outcolor}25}]:}
    
    
        \begin{equation*}\adjustbox{max width=\hsize}{$
        \left [ \frac{1}{2 a} \left(- b + \sqrt{- 4 a c + b^{2}}\right), \quad - \frac{1}{2 a} \left(b + \sqrt{- 4 a c + b^{2}}\right)\right ]
        $}\end{equation*}

    

    In this case \texttt{solve} calculated the solution in terms of the
symbols \texttt{a}, \texttt{b}, and \texttt{c}. You should be able to
recognize the expressions in the solution---it's the quadratic formula
$x_{1,2} = \frac{-b \pm \sqrt{b^2 - 4ac}}{2a}$.

To solve a specific equation like $x^2 + 2x - 8 = 0$, we can substitute
the coefficients $a = 1$, $b = 2$, and $c = -8$ into the general
solution to obtain the same result:

    \begin{Verbatim}[commandchars=\\\{\}]
{\color{incolor}In [{\color{incolor}26}]:} \PY{n}{gen\PYZus{}sol} \PY{o}{=} \PY{n}{solve}\PY{p}{(} \PY{n}{a}\PY{o}{*}\PY{n}{x}\PY{o}{*}\PY{o}{*}\PY{l+m+mi}{2} \PY{o}{+} \PY{n}{b}\PY{o}{*}\PY{n}{x} \PY{o}{+} \PY{n}{c}\PY{p}{,} \PY{n}{x}\PY{p}{)}
         \PY{p}{[} \PY{n}{gen\PYZus{}sol}\PY{p}{[}\PY{l+m+mi}{0}\PY{p}{]}\PY{o}{.}\PY{n}{subs}\PY{p}{(}\PY{p}{\PYZob{}}\PY{l+s}{'}\PY{l+s}{a}\PY{l+s}{'}\PY{p}{:}\PY{l+m+mi}{1}\PY{p}{,}\PY{l+s}{'}\PY{l+s}{b}\PY{l+s}{'}\PY{p}{:}\PY{l+m+mi}{2}\PY{p}{,}\PY{l+s}{'}\PY{l+s}{c}\PY{l+s}{'}\PY{p}{:}\PY{o}{-}\PY{l+m+mi}{8}\PY{p}{\PYZcb{}}\PY{p}{)}\PY{p}{,}
           \PY{n}{gen\PYZus{}sol}\PY{p}{[}\PY{l+m+mi}{1}\PY{p}{]}\PY{o}{.}\PY{n}{subs}\PY{p}{(}\PY{p}{\PYZob{}}\PY{l+s}{'}\PY{l+s}{a}\PY{l+s}{'}\PY{p}{:}\PY{l+m+mi}{1}\PY{p}{,}\PY{l+s}{'}\PY{l+s}{b}\PY{l+s}{'}\PY{p}{:}\PY{l+m+mi}{2}\PY{p}{,}\PY{l+s}{'}\PY{l+s}{c}\PY{l+s}{'}\PY{p}{:}\PY{o}{-}\PY{l+m+mi}{8}\PY{p}{\PYZcb{}}\PY{p}{)} \PY{p}{]}
\end{Verbatim}
\texttt{\color{outcolor}Out[{\color{outcolor}26}]:}
    
    
        \begin{equation*}\adjustbox{max width=\hsize}{$
        \left [ 2, \quad -4\right ]
        $}\end{equation*}

    

    To solve a \emph{system of equations}, you can feed \texttt{solve} with
the list of equations as the first argument, and specify the list of
unknowns you want to solve for as the second argument. For example, to
solve for $x$ and $y$ in the system of equations $x + y = 3$ and
$3x - 2y = 0$, use

    \begin{Verbatim}[commandchars=\\\{\}]
{\color{incolor}In [{\color{incolor}27}]:} \PY{n}{solve}\PY{p}{(}\PY{p}{[}\PY{n}{x} \PY{o}{+} \PY{n}{y} \PY{o}{-} \PY{l+m+mi}{3}\PY{p}{,} \PY{l+m+mi}{3}\PY{o}{*}\PY{n}{x} \PY{o}{-} \PY{l+m+mi}{2}\PY{o}{*}\PY{n}{y}\PY{p}{]}\PY{p}{,} \PY{p}{[}\PY{n}{x}\PY{p}{,} \PY{n}{y}\PY{p}{]}\PY{p}{)}
\end{Verbatim}
\texttt{\color{outcolor}Out[{\color{outcolor}27}]:}
    
    
        \begin{equation*}\adjustbox{max width=\hsize}{$
        \left \{ x : \frac{6}{5}, \quad y : \frac{9}{5}\right \}
        $}\end{equation*}

    

    The function \texttt{solve} is like a Swiss Army knife you can use to
solve all kind of problems. Suppose you want to \emph{complete the
square} in the expression $x^2 - 4x + 7$, that is, you want to find
constants $h$ and $k$ such that $x^2 -4x + 7 = (x-h)^2 + k$. There is no
special ``complete the square'' function in \texttt{SymPy}, but you can
call solve on the equation $(x - h)^2 + k - (x^2 - 4x + 7) = 0$ to find
the unknowns $h$ and $k$:

    \begin{Verbatim}[commandchars=\\\{\}]
{\color{incolor}In [{\color{incolor}28}]:} \PY{n}{h}\PY{p}{,} \PY{n}{k} \PY{o}{=} \PY{n}{symbols}\PY{p}{(}\PY{l+s}{'}\PY{l+s}{h k}\PY{l+s}{'}\PY{p}{)}
         \PY{n}{solve}\PY{p}{(} \PY{p}{(}\PY{n}{x}\PY{o}{-}\PY{n}{h}\PY{p}{)}\PY{o}{*}\PY{o}{*}\PY{l+m+mi}{2} \PY{o}{+} \PY{n}{k} \PY{o}{-} \PY{p}{(}\PY{n}{x}\PY{o}{*}\PY{o}{*}\PY{l+m+mi}{2}\PY{o}{-}\PY{l+m+mi}{4}\PY{o}{*}\PY{n}{x}\PY{o}{+}\PY{l+m+mi}{7}\PY{p}{)}\PY{p}{,} \PY{p}{[}\PY{n}{h}\PY{p}{,}\PY{n}{k}\PY{p}{]} \PY{p}{)}
\end{Verbatim}
\texttt{\color{outcolor}Out[{\color{outcolor}28}]:}
    
    
        \begin{equation*}\adjustbox{max width=\hsize}{$
        \left [ \left ( 2, \quad 3\right )\right ]
        $}\end{equation*}

    

    \begin{Verbatim}[commandchars=\\\{\}]
{\color{incolor}In [{\color{incolor}29}]:} \PY{p}{(}\PY{p}{(}\PY{n}{x}\PY{o}{-}\PY{l+m+mi}{2}\PY{p}{)}\PY{o}{*}\PY{o}{*}\PY{l+m+mi}{2}\PY{o}{+}\PY{l+m+mi}{3}\PY{p}{)}\PY{o}{.}\PY{n}{expand}\PY{p}{(}\PY{p}{)}  \PY{c}{# so h = 2 and k = 3, verify...}
\end{Verbatim}
\texttt{\color{outcolor}Out[{\color{outcolor}29}]:}
    
    
        \begin{equation*}\adjustbox{max width=\hsize}{$
        x^{2} - 4 x + 7
        $}\end{equation*}

    

    Learn the basic \texttt{SymPy} commands and you'll never need to suffer
another tedious arithmetic calculation painstakingly performed by hand
again!

    \subsubsection{Rational functions}\label{rational-functions}

    By default, \texttt{SymPy} will not combine or split rational
expressions. You need to use \texttt{together} to symbolically calculate
the addition of fractions:

    \begin{Verbatim}[commandchars=\\\{\}]
{\color{incolor}In [{\color{incolor}30}]:} \PY{n}{a}\PY{p}{,} \PY{n}{b}\PY{p}{,} \PY{n}{c}\PY{p}{,} \PY{n}{d} \PY{o}{=} \PY{n}{symbols}\PY{p}{(}\PY{l+s}{'}\PY{l+s}{a b c d}\PY{l+s}{'}\PY{p}{)}
         \PY{n}{a}\PY{o}{/}\PY{n}{b} \PY{o}{+} \PY{n}{c}\PY{o}{/}\PY{n}{d}
\end{Verbatim}
\texttt{\color{outcolor}Out[{\color{outcolor}30}]:}
    
    
        \begin{equation*}\adjustbox{max width=\hsize}{$
        \frac{a}{b} + \frac{c}{d}
        $}\end{equation*}

    

    \begin{Verbatim}[commandchars=\\\{\}]
{\color{incolor}In [{\color{incolor}31}]:} \PY{n}{together}\PY{p}{(}\PY{n}{a}\PY{o}{/}\PY{n}{b} \PY{o}{+} \PY{n}{c}\PY{o}{/}\PY{n}{d}\PY{p}{)}
\end{Verbatim}
\texttt{\color{outcolor}Out[{\color{outcolor}31}]:}
    
    
        \begin{equation*}\adjustbox{max width=\hsize}{$
        \frac{1}{b d} \left(a d + b c\right)
        $}\end{equation*}

    

    Alternately, if you have a rational expression and want to divide the
numerator by the denominator, use the \texttt{apart} function:

    \begin{Verbatim}[commandchars=\\\{\}]
{\color{incolor}In [{\color{incolor}32}]:} \PY{n}{apart}\PY{p}{(} \PY{p}{(}\PY{n}{x}\PY{o}{*}\PY{o}{*}\PY{l+m+mi}{2}\PY{o}{+}\PY{n}{x}\PY{o}{+}\PY{l+m+mi}{4}\PY{p}{)}\PY{o}{/}\PY{p}{(}\PY{n}{x}\PY{o}{+}\PY{l+m+mi}{2}\PY{p}{)} \PY{p}{)}
\end{Verbatim}
\texttt{\color{outcolor}Out[{\color{outcolor}32}]:}
    
    
        \begin{equation*}\adjustbox{max width=\hsize}{$
        x - 1 + \frac{6}{x + 2}
        $}\end{equation*}

    

    \subsubsection{Exponentials and
logarithms}\label{exponentials-and-logarithms}

    Euler's constant $e = 2.71828\dots$ is defined one of several ways,

\[
e \equiv \lim_{n\to\infty}\left(1+\frac{1}{n}\right)^n
  \equiv \lim_{\epsilon\to 0}(1+\epsilon)^{1/\epsilon}
  \equiv \sum_{n=0}^{\infty}\frac{1}{n!},
\]

and is denoted \texttt{E} in \texttt{SymPy}. Using \texttt{exp(x)} is
equivalent to \texttt{E**x}.

The functions \texttt{log} and \texttt{ln} both compute the logarithm
base $e$:

    \begin{Verbatim}[commandchars=\\\{\}]
{\color{incolor}In [{\color{incolor}33}]:} \PY{n}{log}\PY{p}{(}\PY{n}{E}\PY{o}{*}\PY{o}{*}\PY{l+m+mi}{3}\PY{p}{)}  \PY{c}{# same as ln(E**3)}
\end{Verbatim}
\texttt{\color{outcolor}Out[{\color{outcolor}33}]:}
    
    
        \begin{equation*}\adjustbox{max width=\hsize}{$
        3
        $}\end{equation*}

    

    By default, \texttt{SymPy} assumes the inputs to functions like
\texttt{exp} and \texttt{log} are complex numbers, so it will not expand
certain logarithmic expressions. However, indicating to \texttt{SymPy}
that the inputs are positive real numbers will make the expansions work:

    \begin{Verbatim}[commandchars=\\\{\}]
{\color{incolor}In [{\color{incolor}34}]:} \PY{n}{x}\PY{p}{,} \PY{n}{y} \PY{o}{=} \PY{n}{symbols}\PY{p}{(}\PY{l+s}{'}\PY{l+s}{x y}\PY{l+s}{'}\PY{p}{)}
         \PY{n}{log}\PY{p}{(}\PY{n}{x}\PY{o}{*}\PY{n}{y}\PY{p}{)}\PY{o}{.}\PY{n}{expand}\PY{p}{(}\PY{p}{)}
\end{Verbatim}
\texttt{\color{outcolor}Out[{\color{outcolor}34}]:}
    
    
        \begin{equation*}\adjustbox{max width=\hsize}{$
        \log{\left (x y \right )}
        $}\end{equation*}

    

    \begin{Verbatim}[commandchars=\\\{\}]
{\color{incolor}In [{\color{incolor}35}]:} \PY{n}{a}\PY{p}{,} \PY{n}{b} \PY{o}{=} \PY{n}{symbols}\PY{p}{(}\PY{l+s}{'}\PY{l+s}{a b}\PY{l+s}{'}\PY{p}{,} \PY{n}{positive}\PY{o}{=}\PY{k}{True}\PY{p}{)}
         \PY{n}{log}\PY{p}{(}\PY{n}{a}\PY{o}{*}\PY{n}{b}\PY{p}{)}\PY{o}{.}\PY{n}{expand}\PY{p}{(}\PY{p}{)}
\end{Verbatim}
\texttt{\color{outcolor}Out[{\color{outcolor}35}]:}
    
    
        \begin{equation*}\adjustbox{max width=\hsize}{$
        \log{\left (a \right )} + \log{\left (b \right )}
        $}\end{equation*}

    

    \subsubsection{Polynomials}\label{polynomials}

    Let's define a polynomial $P$ with roots at $x = 1$, $x = 2$, and
$x = 3$:

    \begin{Verbatim}[commandchars=\\\{\}]
{\color{incolor}In [{\color{incolor}36}]:} \PY{n}{P} \PY{o}{=} \PY{p}{(}\PY{n}{x}\PY{o}{-}\PY{l+m+mi}{1}\PY{p}{)}\PY{o}{*}\PY{p}{(}\PY{n}{x}\PY{o}{-}\PY{l+m+mi}{2}\PY{p}{)}\PY{o}{*}\PY{p}{(}\PY{n}{x}\PY{o}{-}\PY{l+m+mi}{3}\PY{p}{)}
         \PY{n}{P}
\end{Verbatim}
\texttt{\color{outcolor}Out[{\color{outcolor}36}]:}
    
    
        \begin{equation*}\adjustbox{max width=\hsize}{$
        \left(x - 3\right) \left(x - 2\right) \left(x - 1\right)
        $}\end{equation*}

    

    To see the expanded version of the polynomial, call its \texttt{expand}
method:

    \begin{Verbatim}[commandchars=\\\{\}]
{\color{incolor}In [{\color{incolor}37}]:} \PY{n}{P}\PY{o}{.}\PY{n}{expand}\PY{p}{(}\PY{p}{)}
\end{Verbatim}
\texttt{\color{outcolor}Out[{\color{outcolor}37}]:}
    
    
        \begin{equation*}\adjustbox{max width=\hsize}{$
        x^{3} - 6 x^{2} + 11 x - 6
        $}\end{equation*}

    

    When the polynomial is expressed in it's expanded form
$P(x) = x^3 - 6x^2 + 11x - 6$, we can't immediately identify its roots.
This is why the factored form $P(x) = (x - 1)(x - 2)(x - 3)$ is
preferable. To factor a polynomial, call its \texttt{factor} method or
simplify it:

    \begin{Verbatim}[commandchars=\\\{\}]
{\color{incolor}In [{\color{incolor}38}]:} \PY{n}{P}\PY{o}{.}\PY{n}{factor}\PY{p}{(}\PY{p}{)}
\end{Verbatim}
\texttt{\color{outcolor}Out[{\color{outcolor}38}]:}
    
    
        \begin{equation*}\adjustbox{max width=\hsize}{$
        \left(x - 3\right) \left(x - 2\right) \left(x - 1\right)
        $}\end{equation*}

    

    \begin{Verbatim}[commandchars=\\\{\}]
{\color{incolor}In [{\color{incolor}39}]:} \PY{n}{P}\PY{o}{.}\PY{n}{simplify}\PY{p}{(}\PY{p}{)}
\end{Verbatim}
\texttt{\color{outcolor}Out[{\color{outcolor}39}]:}
    
    
        \begin{equation*}\adjustbox{max width=\hsize}{$
        \left(x - 3\right) \left(x - 2\right) \left(x - 1\right)
        $}\end{equation*}

    

    Recall that the roots of the polynomial $P(x)$ are defined as the
solutions to the equation $P(x) = 0$. We can use the \texttt{solve}
function to find the roots of the polynomial:

    \begin{Verbatim}[commandchars=\\\{\}]
{\color{incolor}In [{\color{incolor}40}]:} \PY{n}{roots} \PY{o}{=} \PY{n}{solve}\PY{p}{(}\PY{n}{P}\PY{p}{,}\PY{n}{x}\PY{p}{)}
         \PY{n}{roots}
\end{Verbatim}
\texttt{\color{outcolor}Out[{\color{outcolor}40}]:}
    
    
        \begin{equation*}\adjustbox{max width=\hsize}{$
        \left [ 1, \quad 2, \quad 3\right ]
        $}\end{equation*}

    

    \begin{Verbatim}[commandchars=\\\{\}]
{\color{incolor}In [{\color{incolor}41}]:} \PY{c}{# let's check if P equals (x-1)(x-2)(x-3)}
         \PY{n}{simplify}\PY{p}{(} \PY{n}{P} \PY{o}{-} \PY{p}{(}\PY{n}{x}\PY{o}{-}\PY{n}{roots}\PY{p}{[}\PY{l+m+mi}{0}\PY{p}{]}\PY{p}{)}\PY{o}{*}\PY{p}{(}\PY{n}{x}\PY{o}{-}\PY{n}{roots}\PY{p}{[}\PY{l+m+mi}{1}\PY{p}{]}\PY{p}{)}\PY{o}{*}\PY{p}{(}\PY{n}{x}\PY{o}{-}\PY{n}{roots}\PY{p}{[}\PY{l+m+mi}{2}\PY{p}{]}\PY{p}{)} \PY{p}{)}
\end{Verbatim}
\texttt{\color{outcolor}Out[{\color{outcolor}41}]:}
    
    
        \begin{equation*}\adjustbox{max width=\hsize}{$
        0
        $}\end{equation*}

    

    \subsubsection{Equality checking}\label{equality-checking}

    In the last example, we used the \texttt{simplify} function to check
whether two expressions were equal. This way of checking equality works
because $P = Q$ if and only if $P - Q = 0$. This is the best way to
check if two expressions are equal in \texttt{SymPy} because it attempts
all possible simplifications when comparing the expressions. Below is a
list of other ways to check whether two quantities are equal with
example cases where they fail:

    \begin{Verbatim}[commandchars=\\\{\}]
{\color{incolor}In [{\color{incolor}42}]:} \PY{n}{p} \PY{o}{=} \PY{p}{(}\PY{n}{x}\PY{o}{-}\PY{l+m+mi}{5}\PY{p}{)}\PY{o}{*}\PY{p}{(}\PY{n}{x}\PY{o}{+}\PY{l+m+mi}{5}\PY{p}{)}
         \PY{n}{q} \PY{o}{=} \PY{n}{x}\PY{o}{*}\PY{o}{*}\PY{l+m+mi}{2} \PY{o}{-} \PY{l+m+mi}{25}
\end{Verbatim}

    \begin{Verbatim}[commandchars=\\\{\}]
{\color{incolor}In [{\color{incolor}43}]:} \PY{n}{p} \PY{o}{==} \PY{n}{q}                      \PY{c}{# fail}
\end{Verbatim}

            \begin{Verbatim}[commandchars=\\\{\}]
{\color{outcolor}Out[{\color{outcolor}43}]:} False
\end{Verbatim}
        
    \begin{Verbatim}[commandchars=\\\{\}]
{\color{incolor}In [{\color{incolor}44}]:} \PY{n}{p} \PY{o}{-} \PY{n}{q} \PY{o}{==} \PY{l+m+mi}{0}                  \PY{c}{# fail}
\end{Verbatim}

            \begin{Verbatim}[commandchars=\\\{\}]
{\color{outcolor}Out[{\color{outcolor}44}]:} False
\end{Verbatim}
        
    \begin{Verbatim}[commandchars=\\\{\}]
{\color{incolor}In [{\color{incolor}45}]:} \PY{n}{simplify}\PY{p}{(}\PY{n}{p} \PY{o}{-} \PY{n}{q}\PY{p}{)} \PY{o}{==} \PY{l+m+mi}{0}
\end{Verbatim}

            \begin{Verbatim}[commandchars=\\\{\}]
{\color{outcolor}Out[{\color{outcolor}45}]:} True
\end{Verbatim}
        
    \begin{Verbatim}[commandchars=\\\{\}]
{\color{incolor}In [{\color{incolor}46}]:} \PY{n}{sin}\PY{p}{(}\PY{n}{x}\PY{p}{)}\PY{o}{*}\PY{o}{*}\PY{l+m+mi}{2} \PY{o}{+} \PY{n}{cos}\PY{p}{(}\PY{n}{x}\PY{p}{)}\PY{o}{*}\PY{o}{*}\PY{l+m+mi}{2} \PY{o}{==} \PY{l+m+mi}{1}  \PY{c}{# fail}
\end{Verbatim}

            \begin{Verbatim}[commandchars=\\\{\}]
{\color{outcolor}Out[{\color{outcolor}46}]:} False
\end{Verbatim}
        
    \begin{Verbatim}[commandchars=\\\{\}]
{\color{incolor}In [{\color{incolor}47}]:} \PY{n}{simplify}\PY{p}{(} \PY{n}{sin}\PY{p}{(}\PY{n}{x}\PY{p}{)}\PY{o}{*}\PY{o}{*}\PY{l+m+mi}{2} \PY{o}{+} \PY{n}{cos}\PY{p}{(}\PY{n}{x}\PY{p}{)}\PY{o}{*}\PY{o}{*}\PY{l+m+mi}{2} \PY{o}{-} \PY{l+m+mi}{1}\PY{p}{)} \PY{o}{==} \PY{l+m+mi}{0}
\end{Verbatim}

            \begin{Verbatim}[commandchars=\\\{\}]
{\color{outcolor}Out[{\color{outcolor}47}]:} True
\end{Verbatim}
        
    \subsubsection{Trigonometry}\label{trigonometry}

    The trigonometric functions \texttt{sin} and \texttt{cos} take inputs in
radians:

    \begin{Verbatim}[commandchars=\\\{\}]
{\color{incolor}In [{\color{incolor}48}]:} \PY{n}{sin}\PY{p}{(}\PY{n}{pi}\PY{o}{/}\PY{l+m+mi}{6}\PY{p}{)}
\end{Verbatim}
\texttt{\color{outcolor}Out[{\color{outcolor}48}]:}
    
    
        \begin{equation*}\adjustbox{max width=\hsize}{$
        \frac{1}{2}
        $}\end{equation*}

    

    \begin{Verbatim}[commandchars=\\\{\}]
{\color{incolor}In [{\color{incolor}49}]:} \PY{n}{cos}\PY{p}{(}\PY{n}{pi}\PY{o}{/}\PY{l+m+mi}{6}\PY{p}{)}
\end{Verbatim}
\texttt{\color{outcolor}Out[{\color{outcolor}49}]:}
    
    
        \begin{equation*}\adjustbox{max width=\hsize}{$
        \frac{\sqrt{3}}{2}
        $}\end{equation*}

    

    For angles in degrees, you need a conversion factor of
$\frac{\pi}{180}${[}rad/$^\circ${]}:

    \begin{Verbatim}[commandchars=\\\{\}]
{\color{incolor}In [{\color{incolor}50}]:} \PY{n}{sin}\PY{p}{(}\PY{l+m+mi}{30}\PY{o}{*}\PY{n}{pi}\PY{o}{/}\PY{l+m+mi}{180}\PY{p}{)}  \PY{c}{# 30 deg = pi/6 rads}
\end{Verbatim}
\texttt{\color{outcolor}Out[{\color{outcolor}50}]:}
    
    
        \begin{equation*}\adjustbox{max width=\hsize}{$
        \frac{1}{2}
        $}\end{equation*}

    

    The inverse trigonometric functions $\sin^{-1}(x) \equiv \arcsin(x)$ and
$\cos^{-1}(x) \equiv \arccos(x)$ are used as follows:

    \begin{Verbatim}[commandchars=\\\{\}]
{\color{incolor}In [{\color{incolor}51}]:} \PY{n}{asin}\PY{p}{(}\PY{l+m+mi}{1}\PY{o}{/}\PY{l+m+mi}{2}\PY{p}{)}
\end{Verbatim}
\texttt{\color{outcolor}Out[{\color{outcolor}51}]:}
    
    
        \begin{equation*}\adjustbox{max width=\hsize}{$
        0.523598775598299
        $}\end{equation*}

    

    \begin{Verbatim}[commandchars=\\\{\}]
{\color{incolor}In [{\color{incolor}52}]:} \PY{n}{acos}\PY{p}{(}\PY{n}{sqrt}\PY{p}{(}\PY{l+m+mi}{3}\PY{p}{)}\PY{o}{/}\PY{l+m+mi}{2}\PY{p}{)}
\end{Verbatim}
\texttt{\color{outcolor}Out[{\color{outcolor}52}]:}
    
    
        \begin{equation*}\adjustbox{max width=\hsize}{$
        \frac{\pi}{6}
        $}\end{equation*}

    

    Recall that $\tan(x) \equiv \frac{\sin(x)}{\cos(x)}$. The inverse
function of $\tan(x)$ is $\tan^{-1}(x) \equiv \arctan(x) \equiv$
\texttt{atan(x)}

    \begin{Verbatim}[commandchars=\\\{\}]
{\color{incolor}In [{\color{incolor}53}]:} \PY{n}{tan}\PY{p}{(}\PY{n}{pi}\PY{o}{/}\PY{l+m+mi}{6}\PY{p}{)}
\end{Verbatim}
\texttt{\color{outcolor}Out[{\color{outcolor}53}]:}
    
    
        \begin{equation*}\adjustbox{max width=\hsize}{$
        \frac{\sqrt{3}}{3}
        $}\end{equation*}

    

    \begin{Verbatim}[commandchars=\\\{\}]
{\color{incolor}In [{\color{incolor}54}]:} \PY{n}{atan}\PY{p}{(} \PY{l+m+mi}{1}\PY{o}{/}\PY{n}{sqrt}\PY{p}{(}\PY{l+m+mi}{3}\PY{p}{)} \PY{p}{)}
\end{Verbatim}
\texttt{\color{outcolor}Out[{\color{outcolor}54}]:}
    
    
        \begin{equation*}\adjustbox{max width=\hsize}{$
        \frac{\pi}{6}
        $}\end{equation*}

    

    The function \texttt{acos} returns angles in the range $[0, \pi]$, while
\texttt{asin} and \texttt{atan} return angles in the range
$[-\frac{\pi}{2},\frac{\pi}{2}]$.

Here are some trigonometric identities that \texttt{SymPy} knows:

    \begin{Verbatim}[commandchars=\\\{\}]
{\color{incolor}In [{\color{incolor}55}]:} \PY{n}{sin}\PY{p}{(}\PY{n}{x}\PY{p}{)} \PY{o}{==} \PY{n}{cos}\PY{p}{(}\PY{n}{x} \PY{o}{-} \PY{n}{pi}\PY{o}{/}\PY{l+m+mi}{2}\PY{p}{)}
\end{Verbatim}

            \begin{Verbatim}[commandchars=\\\{\}]
{\color{outcolor}Out[{\color{outcolor}55}]:} True
\end{Verbatim}
        
    \begin{Verbatim}[commandchars=\\\{\}]
{\color{incolor}In [{\color{incolor}56}]:} \PY{n}{simplify}\PY{p}{(} \PY{n}{sin}\PY{p}{(}\PY{n}{x}\PY{p}{)}\PY{o}{*}\PY{n}{cos}\PY{p}{(}\PY{n}{y}\PY{p}{)}\PY{o}{+}\PY{n}{cos}\PY{p}{(}\PY{n}{x}\PY{p}{)}\PY{o}{*}\PY{n}{sin}\PY{p}{(}\PY{n}{y}\PY{p}{)} \PY{p}{)}
\end{Verbatim}
\texttt{\color{outcolor}Out[{\color{outcolor}56}]:}
    
    
        \begin{equation*}\adjustbox{max width=\hsize}{$
        \sin{\left (x + y \right )}
        $}\end{equation*}

    

    \begin{Verbatim}[commandchars=\\\{\}]
{\color{incolor}In [{\color{incolor}57}]:} \PY{n}{e} \PY{o}{=} \PY{l+m+mi}{2}\PY{o}{*}\PY{n}{sin}\PY{p}{(}\PY{n}{x}\PY{p}{)}\PY{o}{*}\PY{o}{*}\PY{l+m+mi}{2} \PY{o}{+} \PY{l+m+mi}{2}\PY{o}{*}\PY{n}{cos}\PY{p}{(}\PY{n}{x}\PY{p}{)}\PY{o}{*}\PY{o}{*}\PY{l+m+mi}{2}
         \PY{n}{trigsimp}\PY{p}{(}\PY{n}{e}\PY{p}{)}
\end{Verbatim}
\texttt{\color{outcolor}Out[{\color{outcolor}57}]:}
    
    
        \begin{equation*}\adjustbox{max width=\hsize}{$
        2
        $}\end{equation*}

    

    \begin{Verbatim}[commandchars=\\\{\}]
{\color{incolor}In [{\color{incolor}58}]:} \PY{n}{trigsimp}\PY{p}{(}\PY{n}{log}\PY{p}{(}\PY{n}{e}\PY{p}{)}\PY{p}{)}
\end{Verbatim}
\texttt{\color{outcolor}Out[{\color{outcolor}58}]:}
    
    
        \begin{equation*}\adjustbox{max width=\hsize}{$
        \log{\left (2 \right )}
        $}\end{equation*}

    

    \begin{Verbatim}[commandchars=\\\{\}]
{\color{incolor}In [{\color{incolor}59}]:} \PY{n}{trigsimp}\PY{p}{(}\PY{n}{log}\PY{p}{(}\PY{n}{e}\PY{p}{)}\PY{p}{,} \PY{n}{deep}\PY{o}{=}\PY{k}{True}\PY{p}{)}
\end{Verbatim}
\texttt{\color{outcolor}Out[{\color{outcolor}59}]:}
    
    
        \begin{equation*}\adjustbox{max width=\hsize}{$
        \log{\left (2 \right )}
        $}\end{equation*}

    

    \begin{Verbatim}[commandchars=\\\{\}]
{\color{incolor}In [{\color{incolor}60}]:} \PY{n}{simplify}\PY{p}{(}\PY{n}{sin}\PY{p}{(}\PY{n}{x}\PY{p}{)}\PY{o}{*}\PY{o}{*}\PY{l+m+mi}{4} \PY{o}{-} \PY{l+m+mi}{2}\PY{o}{*}\PY{n}{cos}\PY{p}{(}\PY{n}{x}\PY{p}{)}\PY{o}{*}\PY{o}{*}\PY{l+m+mi}{2}\PY{o}{*}\PY{n}{sin}\PY{p}{(}\PY{n}{x}\PY{p}{)}\PY{o}{*}\PY{o}{*}\PY{l+m+mi}{2} \PY{o}{+} \PY{n}{cos}\PY{p}{(}\PY{n}{x}\PY{p}{)}\PY{o}{*}\PY{o}{*}\PY{l+m+mi}{4}\PY{p}{)}
\end{Verbatim}
\texttt{\color{outcolor}Out[{\color{outcolor}60}]:}
    
    
        \begin{equation*}\adjustbox{max width=\hsize}{$
        \frac{1}{2} \cos{\left (4 x \right )} + \frac{1}{2}
        $}\end{equation*}

    

    The function \texttt{trigsimp} does essentially the same job as
\texttt{simplify}.

If instead of simplifying you want to expand a trig expression, you
should use \texttt{expand\_trig}, because the default \texttt{expand}
won't touch trig functions:

    \begin{Verbatim}[commandchars=\\\{\}]
{\color{incolor}In [{\color{incolor}61}]:} \PY{n}{expand}\PY{p}{(}\PY{n}{sin}\PY{p}{(}\PY{l+m+mi}{2}\PY{o}{*}\PY{n}{x}\PY{p}{)}\PY{p}{)}       \PY{c}{# = (sin(2*x)).expand()}
\end{Verbatim}
\texttt{\color{outcolor}Out[{\color{outcolor}61}]:}
    
    
        \begin{equation*}\adjustbox{max width=\hsize}{$
        \sin{\left (2 x \right )}
        $}\end{equation*}

    

    \begin{Verbatim}[commandchars=\\\{\}]
{\color{incolor}In [{\color{incolor}62}]:} \PY{n}{expand\PYZus{}trig}\PY{p}{(}\PY{n}{sin}\PY{p}{(}\PY{l+m+mi}{2}\PY{o}{*}\PY{n}{x}\PY{p}{)}\PY{p}{)}  \PY{c}{# = (sin(2*x)).expand(trig=True)}
\end{Verbatim}
\texttt{\color{outcolor}Out[{\color{outcolor}62}]:}
    
    
        \begin{equation*}\adjustbox{max width=\hsize}{$
        2 \sin{\left (x \right )} \cos{\left (x \right )}
        $}\end{equation*}

    

    \subsubsection{Hyperbolic trigonometric
functions}\label{hyperbolic-trigonometric-functions}

    The hyperbolic sine and cosine in \texttt{SymPy} are denoted
\texttt{sinh} and \texttt{cosh} respectively and \texttt{SymPy} is smart
enough to recognize them when simplifying expressions:

    \begin{Verbatim}[commandchars=\\\{\}]
{\color{incolor}In [{\color{incolor}63}]:} \PY{n}{simplify}\PY{p}{(} \PY{p}{(}\PY{n}{exp}\PY{p}{(}\PY{n}{x}\PY{p}{)}\PY{o}{+}\PY{n}{exp}\PY{p}{(}\PY{o}{-}\PY{n}{x}\PY{p}{)}\PY{p}{)}\PY{o}{/}\PY{l+m+mi}{2} \PY{p}{)}
\end{Verbatim}
\texttt{\color{outcolor}Out[{\color{outcolor}63}]:}
    
    
        \begin{equation*}\adjustbox{max width=\hsize}{$
        \cosh{\left (x \right )}
        $}\end{equation*}

    

    \begin{Verbatim}[commandchars=\\\{\}]
{\color{incolor}In [{\color{incolor}64}]:} \PY{n}{simplify}\PY{p}{(} \PY{p}{(}\PY{n}{exp}\PY{p}{(}\PY{n}{x}\PY{p}{)}\PY{o}{-}\PY{n}{exp}\PY{p}{(}\PY{o}{-}\PY{n}{x}\PY{p}{)}\PY{p}{)}\PY{o}{/}\PY{l+m+mi}{2} \PY{p}{)}
\end{Verbatim}
\texttt{\color{outcolor}Out[{\color{outcolor}64}]:}
    
    
        \begin{equation*}\adjustbox{max width=\hsize}{$
        \sinh{\left (x \right )}
        $}\end{equation*}

    

    Recall that $x = \cosh(\mu)$ and $y = \sinh(\mu)$ are defined as $x$ and
$y$ coordinates of a point on the the hyperbola with equation
$x^2 - y^2 = 1$ and therefore satisfy the identity
$\cosh^2 x - \sinh^2 x = 1$:

    \begin{Verbatim}[commandchars=\\\{\}]
{\color{incolor}In [{\color{incolor}65}]:} \PY{n}{simplify}\PY{p}{(} \PY{n}{cosh}\PY{p}{(}\PY{n}{x}\PY{p}{)}\PY{o}{*}\PY{o}{*}\PY{l+m+mi}{2} \PY{o}{-} \PY{n}{sinh}\PY{p}{(}\PY{n}{x}\PY{p}{)}\PY{o}{*}\PY{o}{*}\PY{l+m+mi}{2} \PY{p}{)}
\end{Verbatim}
\texttt{\color{outcolor}Out[{\color{outcolor}65}]:}
    
    
        \begin{equation*}\adjustbox{max width=\hsize}{$
        1
        $}\end{equation*}

    

    \subsection{Complex numbers}\label{complex-numbers}

    Ever since Newton, the word ``number'' has been used to refer to one of
the following types of math objects: the naturals $\mathbb{N}$, the
integers $\mathbb{Z}$, the rationals $\mathbb{Q}$, and the real numbers
$\mathbb{R}$. Each set of numbers is associated with a different class
of equations. The natural numbers $\mathbb{N}$ appear as solutions of
the equation $m + n = x$, where $m$ and $n$ are natural numbers (denoted
$m, n \in \mathbb{N}$). The integers $\mathbb{Z}$ are the solutions to
equations of the form $x + m = n$, where $m, n \in \mathbb{N}$. The
rational numbers $\mathbb{Q}$ are necessary to solve for $x$ in
$mx = n$, with $m, n \in \mathbb{Z}$. The solutions to $x^2 = 2$ are
irrational (so $\not\in \mathbb{Q}$) so we need an even larger set that
contains \emph{all} possible numbers: real set of numbers $\mathbb{R}$.
A pattern emerges where more complicated equations require the invention
of new types of numbers.

Consider the quadratic equation $x^2 = -1$. There are no real solutions
to this equation, but we can define an imaginary number $i = \sqrt{-1}$
(denoted \texttt{I} in \texttt{SymPy}) that satisfies this equation:

    \begin{Verbatim}[commandchars=\\\{\}]
{\color{incolor}In [{\color{incolor}66}]:} \PY{n}{I}\PY{o}{*}\PY{n}{I}
\end{Verbatim}
\texttt{\color{outcolor}Out[{\color{outcolor}66}]:}
    
    
        \begin{equation*}\adjustbox{max width=\hsize}{$
        -1
        $}\end{equation*}

    

    \begin{Verbatim}[commandchars=\\\{\}]
{\color{incolor}In [{\color{incolor}67}]:} \PY{n}{solve}\PY{p}{(} \PY{n}{x}\PY{o}{*}\PY{o}{*}\PY{l+m+mi}{2} \PY{o}{+} \PY{l+m+mi}{1} \PY{p}{,} \PY{n}{x}\PY{p}{)}
\end{Verbatim}
\texttt{\color{outcolor}Out[{\color{outcolor}67}]:}
    
    
        \begin{equation*}\adjustbox{max width=\hsize}{$
        \left [ - i, \quad i\right ]
        $}\end{equation*}

    

    The solutions are $x = i$ and $x = -i$, and indeed we can verify that
$i^2 + 1 = 0$ and $(-i)^2 + 1 = 0$ since $i^2 = -1$.

The complex numbers $\mathbb{C}$ are defined as
$\{ a+bi \,|\, a,b \in \mathbb{R} \}$. Complex numbers contain a real
part and an imaginary part:

    \begin{Verbatim}[commandchars=\\\{\}]
{\color{incolor}In [{\color{incolor}68}]:} \PY{n}{z} \PY{o}{=} \PY{l+m+mi}{4} \PY{o}{+} \PY{l+m+mi}{3}\PY{o}{*}\PY{n}{I}
         \PY{n}{z}
\end{Verbatim}
\texttt{\color{outcolor}Out[{\color{outcolor}68}]:}
    
    
        \begin{equation*}\adjustbox{max width=\hsize}{$
        4 + 3 i
        $}\end{equation*}

    

    \begin{Verbatim}[commandchars=\\\{\}]
{\color{incolor}In [{\color{incolor}69}]:} \PY{n}{re}\PY{p}{(}\PY{n}{z}\PY{p}{)}
\end{Verbatim}
\texttt{\color{outcolor}Out[{\color{outcolor}69}]:}
    
    
        \begin{equation*}\adjustbox{max width=\hsize}{$
        4
        $}\end{equation*}

    

    \begin{Verbatim}[commandchars=\\\{\}]
{\color{incolor}In [{\color{incolor}70}]:} \PY{n}{im}\PY{p}{(}\PY{n}{z}\PY{p}{)}
\end{Verbatim}
\texttt{\color{outcolor}Out[{\color{outcolor}70}]:}
    
    
        \begin{equation*}\adjustbox{max width=\hsize}{$
        3
        $}\end{equation*}

    

    The \emph{polar} representation of a complex number is
$z\!\equiv\!|z|\angle\theta\!\equiv \!|z|e^{i\theta}$. For a complex
number $z=a+bi$, the quantity $|z|=\sqrt{a^2+b^2}$ is known as the
absolute value of $z$, and $\theta$ is its \emph{phase} or its
\emph{argument}:

    \begin{Verbatim}[commandchars=\\\{\}]
{\color{incolor}In [{\color{incolor}71}]:} \PY{n}{Abs}\PY{p}{(}\PY{n}{z}\PY{p}{)}
\end{Verbatim}
\texttt{\color{outcolor}Out[{\color{outcolor}71}]:}
    
    
        \begin{equation*}\adjustbox{max width=\hsize}{$
        5
        $}\end{equation*}

    

    \begin{Verbatim}[commandchars=\\\{\}]
{\color{incolor}In [{\color{incolor}72}]:} \PY{n}{arg}\PY{p}{(}\PY{n}{z}\PY{p}{)}
\end{Verbatim}
\texttt{\color{outcolor}Out[{\color{outcolor}72}]:}
    
    
        \begin{equation*}\adjustbox{max width=\hsize}{$
        \operatorname{atan}{\left (\frac{3}{4} \right )}
        $}\end{equation*}

    

    The complex conjugate of $z = a + bi$ is the number $\bar{z} = a - bi$:

    \begin{Verbatim}[commandchars=\\\{\}]
{\color{incolor}In [{\color{incolor}73}]:} \PY{n}{conjugate}\PY{p}{(} \PY{n}{z} \PY{p}{)}
\end{Verbatim}
\texttt{\color{outcolor}Out[{\color{outcolor}73}]:}
    
    
        \begin{equation*}\adjustbox{max width=\hsize}{$
        4 - 3 i
        $}\end{equation*}

    

    Complex conjugation is important for computing the absolute value of $z$
$\left(|z|\equiv\sqrt{z\bar{z}}\right)$ and for division by $z$
$\left(\frac{1}{z}\equiv\frac{\bar{z}}{|z|^2}\right)$.

    \subsubsection{Euler's formula}\label{eulers-formula}

    {[}Euler's formula{]}(https://en.wikipedia.org/wiki/Euler's\_formula)
shows an important relation between the exponential function $e^x$ and
the trigonometric functions $sin(x)$ and $cos(x)$:

\[e^{ix} = \cos x + i \sin x.\]

To obtain this result in \texttt{SymPy}, you must specify that the
number $x$ is real and also tell \texttt{expand} that you're interested
in complex expansions:

    \begin{Verbatim}[commandchars=\\\{\}]
{\color{incolor}In [{\color{incolor}74}]:} \PY{n}{x} \PY{o}{=} \PY{n}{symbols}\PY{p}{(}\PY{l+s}{'}\PY{l+s}{x}\PY{l+s}{'}\PY{p}{,} \PY{n}{real}\PY{o}{=}\PY{k}{True}\PY{p}{)}
         \PY{n}{exp}\PY{p}{(}\PY{n}{I}\PY{o}{*}\PY{n}{x}\PY{p}{)}\PY{o}{.}\PY{n}{expand}\PY{p}{(}\PY{n+nb}{complex}\PY{o}{=}\PY{k}{True}\PY{p}{)}
\end{Verbatim}
\texttt{\color{outcolor}Out[{\color{outcolor}74}]:}
    
    
        \begin{equation*}\adjustbox{max width=\hsize}{$
        i \sin{\left (x \right )} + \cos{\left (x \right )}
        $}\end{equation*}

    

    \begin{Verbatim}[commandchars=\\\{\}]
{\color{incolor}In [{\color{incolor}75}]:} \PY{n}{re}\PY{p}{(} \PY{n}{exp}\PY{p}{(}\PY{n}{I}\PY{o}{*}\PY{n}{x}\PY{p}{)} \PY{p}{)}
\end{Verbatim}
\texttt{\color{outcolor}Out[{\color{outcolor}75}]:}
    
    
        \begin{equation*}\adjustbox{max width=\hsize}{$
        \cos{\left (x \right )}
        $}\end{equation*}

    

    \begin{Verbatim}[commandchars=\\\{\}]
{\color{incolor}In [{\color{incolor}76}]:} \PY{n}{im}\PY{p}{(} \PY{n}{exp}\PY{p}{(}\PY{n}{I}\PY{o}{*}\PY{n}{x}\PY{p}{)} \PY{p}{)}
\end{Verbatim}
\texttt{\color{outcolor}Out[{\color{outcolor}76}]:}
    
    
        \begin{equation*}\adjustbox{max width=\hsize}{$
        \sin{\left (x \right )}
        $}\end{equation*}

    

    Basically, $\cos(x)$ is the real part of $e^{ix}$, and $\sin(x)$ is the
imaginary part of $e^{ix}$. Whaaat? I know it's weird, but weird things
are bound to happen when you input imaginary numbers to functions.

Euler's formula is often used to rewrite the functions \texttt{sin} and
\texttt{cos} in terms of complex exponentials. For example,

    \begin{Verbatim}[commandchars=\\\{\}]
{\color{incolor}In [{\color{incolor}77}]:} \PY{p}{(}\PY{n}{cos}\PY{p}{(}\PY{n}{x}\PY{p}{)}\PY{p}{)}\PY{o}{.}\PY{n}{rewrite}\PY{p}{(}\PY{n}{exp}\PY{p}{)}
\end{Verbatim}
\texttt{\color{outcolor}Out[{\color{outcolor}77}]:}
    
    
        \begin{equation*}\adjustbox{max width=\hsize}{$
        \frac{e^{i x}}{2} + \frac{1}{2} e^{- i x}
        $}\end{equation*}

    

    Compare this expression with the definition of hyperbolic cosine.

    \subsection{Calculus}\label{calculus}

    Calculus is the study of the properties of functions. The operations of
calculus are used to describe the limit behaviour of functions,
calculate their rates of change, and calculate the areas under their
graphs. In this section we'll learn about the \texttt{SymPy} functions
for calculating limits, derivatives, integrals, and summations.

    \subsubsection{Infinity}\label{infinity}

    The infinity symbol is denoted \texttt{oo} (two lowercase \texttt{o}s)
in \texttt{SymPy}. Infinity is not a number but a process: the process
of counting forever. Thus, $\infty + 1 = \infty$, $\infty$ is greater
than any finite number, and $1/\infty$ is an infinitely small number.
\texttt{Sympy} knows how to correctly treat infinity in expressions:

    \begin{Verbatim}[commandchars=\\\{\}]
{\color{incolor}In [{\color{incolor}78}]:} \PY{n}{oo}\PY{o}{+}\PY{l+m+mi}{1}
\end{Verbatim}
\texttt{\color{outcolor}Out[{\color{outcolor}78}]:}
    
    
        \begin{equation*}\adjustbox{max width=\hsize}{$
        \infty
        $}\end{equation*}

    

    \begin{Verbatim}[commandchars=\\\{\}]
{\color{incolor}In [{\color{incolor}79}]:} \PY{l+m+mi}{5000} \PY{o}{<} \PY{n}{oo}
\end{Verbatim}
\texttt{\color{outcolor}Out[{\color{outcolor}79}]:}
    
    
        \begin{equation*}\adjustbox{max width=\hsize}{$
        \mathrm{True}
        $}\end{equation*}

    

    \begin{Verbatim}[commandchars=\\\{\}]
{\color{incolor}In [{\color{incolor}80}]:} \PY{l+m+mi}{1}\PY{o}{/}\PY{n}{oo}
\end{Verbatim}
\texttt{\color{outcolor}Out[{\color{outcolor}80}]:}
    
    
        \begin{equation*}\adjustbox{max width=\hsize}{$
        0
        $}\end{equation*}

    

    \subsubsection{Limits}\label{limits}

    We use limits to describe, with mathematical precision, infinitely large
quantities, infinitely small quantities, and procedures with infinitely
many steps.

The number $e$ is defined as the limit
$e \equiv \lim_{n\to\infty}\left(1+\frac{1}{n}\right)^n$:

    \begin{Verbatim}[commandchars=\\\{\}]
{\color{incolor}In [{\color{incolor}81}]:} \PY{n}{limit}\PY{p}{(} \PY{p}{(}\PY{l+m+mi}{1}\PY{o}{+}\PY{l+m+mi}{1}\PY{o}{/}\PY{n}{n}\PY{p}{)}\PY{o}{*}\PY{o}{*}\PY{n}{n}\PY{p}{,} \PY{n}{n}\PY{p}{,} \PY{n}{oo}\PY{p}{)}
\end{Verbatim}
\texttt{\color{outcolor}Out[{\color{outcolor}81}]:}
    
    
        \begin{equation*}\adjustbox{max width=\hsize}{$
        e
        $}\end{equation*}

    

    This limit expression describes the annual growth rate of a loan with a
nominal interest rate of 100\% and infinitely frequent compounding.
Borrow \$1000 in such a scheme, and you'll owe \$2718.28 after one year.

Limits are also useful to describe the behaviour of functions. Consider
the function $f(x) = \frac{1}{x}$. The \texttt{limit} command shows us
what happens to $f(x)$ near $x = 0$ and as $x$ goes to infinity:

    \begin{Verbatim}[commandchars=\\\{\}]
{\color{incolor}In [{\color{incolor}82}]:} \PY{n}{limit}\PY{p}{(} \PY{l+m+mi}{1}\PY{o}{/}\PY{n}{x}\PY{p}{,} \PY{n}{x}\PY{p}{,} \PY{l+m+mi}{0}\PY{p}{,} \PY{n+nb}{dir}\PY{o}{=}\PY{l+s}{"}\PY{l+s}{+}\PY{l+s}{"}\PY{p}{)}
\end{Verbatim}
\texttt{\color{outcolor}Out[{\color{outcolor}82}]:}
    
    
        \begin{equation*}\adjustbox{max width=\hsize}{$
        \infty
        $}\end{equation*}

    

    \begin{Verbatim}[commandchars=\\\{\}]
{\color{incolor}In [{\color{incolor}83}]:} \PY{n}{limit}\PY{p}{(} \PY{l+m+mi}{1}\PY{o}{/}\PY{n}{x}\PY{p}{,} \PY{n}{x}\PY{p}{,} \PY{l+m+mi}{0}\PY{p}{,} \PY{n+nb}{dir}\PY{o}{=}\PY{l+s}{"}\PY{l+s}{-}\PY{l+s}{"}\PY{p}{)}
\end{Verbatim}
\texttt{\color{outcolor}Out[{\color{outcolor}83}]:}
    
    
        \begin{equation*}\adjustbox{max width=\hsize}{$
        -\infty
        $}\end{equation*}

    

    \begin{Verbatim}[commandchars=\\\{\}]
{\color{incolor}In [{\color{incolor}84}]:} \PY{n}{limit}\PY{p}{(} \PY{l+m+mi}{1}\PY{o}{/}\PY{n}{x}\PY{p}{,} \PY{n}{x}\PY{p}{,} \PY{n}{oo}\PY{p}{)}
\end{Verbatim}
\texttt{\color{outcolor}Out[{\color{outcolor}84}]:}
    
    
        \begin{equation*}\adjustbox{max width=\hsize}{$
        0
        $}\end{equation*}

    

    As $x$ becomes larger and larger, the fraction $\frac{1}{x}$ becomes
smaller and smaller. In the limit where $x$ goes to infinity,
$\frac{1}{x}$ approaches zero: $\lim_{x\to\infty}\frac{1}{x} = 0$. On
the other hand, when $x$ takes on smaller and smaller positive values,
the expression $\frac{1}{x}$ becomes infinite:
$\lim_{x\to0^+}\frac{1}{x} = \infty$. When $x$ approaches 0 from the
left, we have $\lim_{x\to0^-}\frac{1}{x}=-\infty$. If these calculations
are not clear to you, study the graph of $f(x) = \frac{1}{x}$.

Here are some other examples of limits:

    \begin{Verbatim}[commandchars=\\\{\}]
{\color{incolor}In [{\color{incolor}85}]:} \PY{n}{limit}\PY{p}{(}\PY{n}{sin}\PY{p}{(}\PY{n}{x}\PY{p}{)}\PY{o}{/}\PY{n}{x}\PY{p}{,} \PY{n}{x}\PY{p}{,} \PY{l+m+mi}{0}\PY{p}{)}
\end{Verbatim}
\texttt{\color{outcolor}Out[{\color{outcolor}85}]:}
    
    
        \begin{equation*}\adjustbox{max width=\hsize}{$
        1
        $}\end{equation*}

    

    \begin{Verbatim}[commandchars=\\\{\}]
{\color{incolor}In [{\color{incolor}86}]:} \PY{n}{limit}\PY{p}{(}\PY{n}{sin}\PY{p}{(}\PY{n}{x}\PY{p}{)}\PY{o}{*}\PY{o}{*}\PY{l+m+mi}{2}\PY{o}{/}\PY{n}{x}\PY{p}{,} \PY{n}{x}\PY{p}{,} \PY{l+m+mi}{0}\PY{p}{)}
\end{Verbatim}
\texttt{\color{outcolor}Out[{\color{outcolor}86}]:}
    
    
        \begin{equation*}\adjustbox{max width=\hsize}{$
        0
        $}\end{equation*}

    

    \begin{Verbatim}[commandchars=\\\{\}]
{\color{incolor}In [{\color{incolor}87}]:} \PY{n}{limit}\PY{p}{(}\PY{n}{exp}\PY{p}{(}\PY{n}{x}\PY{p}{)}\PY{o}{/}\PY{n}{x}\PY{o}{*}\PY{o}{*}\PY{l+m+mi}{100}\PY{p}{,}\PY{n}{x}\PY{p}{,}\PY{n}{oo}\PY{p}{)}  \PY{c}{# which is bigger e\PYZca{}x or x\PYZca{}100 ?}
                                    \PY{c}{# exp f >> all poly f for big x}
\end{Verbatim}
\texttt{\color{outcolor}Out[{\color{outcolor}87}]:}
    
    
        \begin{equation*}\adjustbox{max width=\hsize}{$
        \infty
        $}\end{equation*}

    

    Limits are used to define the derivative and the integral operations.

    \subsubsection{Derivatives}\label{derivatives}

    The derivative function, denoted $f'(x)$, $\frac{d}{dx}f(x)$,
$\frac{df}{dx}$, or $\frac{dy}{dx}$, describes the \emph{rate of change}
of the function $f(x)$. The \texttt{SymPy} function \texttt{diff}
computes the derivative of any expression:

    \begin{Verbatim}[commandchars=\\\{\}]
{\color{incolor}In [{\color{incolor}88}]:} \PY{n}{diff}\PY{p}{(}\PY{n}{x}\PY{o}{*}\PY{o}{*}\PY{l+m+mi}{3}\PY{p}{,} \PY{n}{x}\PY{p}{)}
\end{Verbatim}
\texttt{\color{outcolor}Out[{\color{outcolor}88}]:}
    
    
        \begin{equation*}\adjustbox{max width=\hsize}{$
        3 x^{2}
        $}\end{equation*}

    

    The differentiation operation knows about the product rule
$[f(x)g(x)]^\prime=f^\prime(x)g(x)+f(x)g^\prime(x)$, the chain rule
$f(g(x))' = f'(g(x))g'(x)$, and the quotient rule
$\left[\frac{f(x)}{g(x)}\right]^\prime = \frac{f'(x)g(x) - f(x)g'(x)}{g(x)^2}$:

    \begin{Verbatim}[commandchars=\\\{\}]
{\color{incolor}In [{\color{incolor}89}]:} \PY{n}{diff}\PY{p}{(} \PY{n}{x}\PY{o}{*}\PY{o}{*}\PY{l+m+mi}{2}\PY{o}{*}\PY{n}{sin}\PY{p}{(}\PY{n}{x}\PY{p}{)}\PY{p}{,} \PY{n}{x} \PY{p}{)}
\end{Verbatim}
\texttt{\color{outcolor}Out[{\color{outcolor}89}]:}
    
    
        \begin{equation*}\adjustbox{max width=\hsize}{$
        x^{2} \cos{\left (x \right )} + 2 x \sin{\left (x \right )}
        $}\end{equation*}

    

    \begin{Verbatim}[commandchars=\\\{\}]
{\color{incolor}In [{\color{incolor}90}]:} \PY{n}{diff}\PY{p}{(} \PY{n}{sin}\PY{p}{(}\PY{n}{x}\PY{o}{*}\PY{o}{*}\PY{l+m+mi}{2}\PY{p}{)}\PY{p}{,} \PY{n}{x} \PY{p}{)}
\end{Verbatim}
\texttt{\color{outcolor}Out[{\color{outcolor}90}]:}
    
    
        \begin{equation*}\adjustbox{max width=\hsize}{$
        2 x \cos{\left (x^{2} \right )}
        $}\end{equation*}

    

    \begin{Verbatim}[commandchars=\\\{\}]
{\color{incolor}In [{\color{incolor}91}]:} \PY{n}{diff}\PY{p}{(} \PY{n}{x}\PY{o}{*}\PY{o}{*}\PY{l+m+mi}{2}\PY{o}{/}\PY{n}{sin}\PY{p}{(}\PY{n}{x}\PY{p}{)}\PY{p}{,} \PY{n}{x} \PY{p}{)}
\end{Verbatim}
\texttt{\color{outcolor}Out[{\color{outcolor}91}]:}
    
    
        \begin{equation*}\adjustbox{max width=\hsize}{$
        - \frac{x^{2} \cos{\left (x \right )}}{\sin^{2}{\left (x \right )}} + \frac{2 x}{\sin{\left (x \right )}}
        $}\end{equation*}

    

    The second derivative of a function \texttt{f} is \texttt{diff(f,x,2)}:

    \begin{Verbatim}[commandchars=\\\{\}]
{\color{incolor}In [{\color{incolor}92}]:} \PY{n}{diff}\PY{p}{(}\PY{n}{x}\PY{o}{*}\PY{o}{*}\PY{l+m+mi}{3}\PY{p}{,} \PY{n}{x}\PY{p}{,} \PY{l+m+mi}{2}\PY{p}{)}   \PY{c}{# same as diff(diff(x**3, x), x)}
\end{Verbatim}
\texttt{\color{outcolor}Out[{\color{outcolor}92}]:}
    
    
        \begin{equation*}\adjustbox{max width=\hsize}{$
        6 x
        $}\end{equation*}

    

    The exponential function $f(x)=e^x$ is special because it is equal to
its derivative:

    \begin{Verbatim}[commandchars=\\\{\}]
{\color{incolor}In [{\color{incolor}93}]:} \PY{n}{diff}\PY{p}{(} \PY{n}{exp}\PY{p}{(}\PY{n}{x}\PY{p}{)}\PY{p}{,} \PY{n}{x} \PY{p}{)}  \PY{c}{# same as diff( E**x, x  )}
\end{Verbatim}
\texttt{\color{outcolor}Out[{\color{outcolor}93}]:}
    
    
        \begin{equation*}\adjustbox{max width=\hsize}{$
        e^{x}
        $}\end{equation*}

    

    A differential equation is an equation that relates some unknown
function $f(x)$ to its derivative. An example of a differential equation
is $f'(x)=f(x)$. What is the function $f(x)$ which is equal to its
derivative? You can either try to guess what $f(x)$ is or use the
\texttt{dsolve} function:

    \begin{Verbatim}[commandchars=\\\{\}]
{\color{incolor}In [{\color{incolor}94}]:} \PY{n}{x} \PY{o}{=} \PY{n}{symbols}\PY{p}{(}\PY{l+s}{'}\PY{l+s}{x}\PY{l+s}{'}\PY{p}{)}
         \PY{n}{f} \PY{o}{=} \PY{n}{symbols}\PY{p}{(}\PY{l+s}{'}\PY{l+s}{f}\PY{l+s}{'}\PY{p}{,} \PY{n}{cls}\PY{o}{=}\PY{n}{Function}\PY{p}{)}  \PY{c}{# can now use f(x)}
         \PY{n}{dsolve}\PY{p}{(} \PY{n}{f}\PY{p}{(}\PY{n}{x}\PY{p}{)} \PY{o}{-} \PY{n}{diff}\PY{p}{(}\PY{n}{f}\PY{p}{(}\PY{n}{x}\PY{p}{)}\PY{p}{,}\PY{n}{x}\PY{p}{)}\PY{p}{,} \PY{n}{f}\PY{p}{(}\PY{n}{x}\PY{p}{)} \PY{p}{)}
\end{Verbatim}
\texttt{\color{outcolor}Out[{\color{outcolor}94}]:}
    
    
        \begin{equation*}\adjustbox{max width=\hsize}{$
        f{\left (x \right )} = C_{1} e^{x}
        $}\end{equation*}

    

    We'll discuss \texttt{dsolve} again in the section on mechanics.

    \subsubsection{Tangent lines}\label{tangent-lines}

    The \emph{tangent line} to the function $f(x)$ at $x=x_0$ is the line
that passes through the point $(x_0, f(x_0))$ and has the same slope as
the function at that point. The tangent line to the function $f(x)$ at
the point $x=x_0$ is described by the equation

\[
   T_1(x) =  f(x_0) \ + \  f'(x_0)(x-x_0).
\]

What is the equation of the tangent line to $f(x)=\frac{1}{2}x^2$ at
$x_0=1$?

    \begin{Verbatim}[commandchars=\\\{\}]
{\color{incolor}In [{\color{incolor}95}]:} \PY{n}{f} \PY{o}{=} \PY{n}{S}\PY{p}{(}\PY{l+s}{'}\PY{l+s}{1/2}\PY{l+s}{'}\PY{p}{)}\PY{o}{*}\PY{n}{x}\PY{o}{*}\PY{o}{*}\PY{l+m+mi}{2}
         \PY{n}{f}
\end{Verbatim}
\texttt{\color{outcolor}Out[{\color{outcolor}95}]:}
    
    
        \begin{equation*}\adjustbox{max width=\hsize}{$
        \frac{x^{2}}{2}
        $}\end{equation*}

    

    \begin{Verbatim}[commandchars=\\\{\}]
{\color{incolor}In [{\color{incolor}96}]:} \PY{n}{df} \PY{o}{=} \PY{n}{diff}\PY{p}{(}\PY{n}{f}\PY{p}{,}\PY{n}{x}\PY{p}{)}
         \PY{n}{df}
\end{Verbatim}
\texttt{\color{outcolor}Out[{\color{outcolor}96}]:}
    
    
        \begin{equation*}\adjustbox{max width=\hsize}{$
        x
        $}\end{equation*}

    

    \begin{Verbatim}[commandchars=\\\{\}]
{\color{incolor}In [{\color{incolor}97}]:} \PY{n}{T\PYZus{}1} \PY{o}{=} \PY{n}{f}\PY{o}{.}\PY{n}{subs}\PY{p}{(}\PY{p}{\PYZob{}}\PY{n}{x}\PY{p}{:}\PY{l+m+mi}{1}\PY{p}{\PYZcb{}}\PY{p}{)} \PY{o}{+} \PY{n}{df}\PY{o}{.}\PY{n}{subs}\PY{p}{(}\PY{p}{\PYZob{}}\PY{n}{x}\PY{p}{:}\PY{l+m+mi}{1}\PY{p}{\PYZcb{}}\PY{p}{)}\PY{o}{*}\PY{p}{(}\PY{n}{x} \PY{o}{-} \PY{l+m+mi}{1}\PY{p}{)}
         \PY{n}{T\PYZus{}1}
\end{Verbatim}
\texttt{\color{outcolor}Out[{\color{outcolor}97}]:}
    
    
        \begin{equation*}\adjustbox{max width=\hsize}{$
        x - \frac{1}{2}
        $}\end{equation*}

    

    The tangent line $T_1(x)$ has the same value and slope as the function
$f(x)$ at $x=1$:

    \begin{Verbatim}[commandchars=\\\{\}]
{\color{incolor}In [{\color{incolor}98}]:} \PY{n}{T\PYZus{}1}\PY{o}{.}\PY{n}{subs}\PY{p}{(}\PY{p}{\PYZob{}}\PY{n}{x}\PY{p}{:}\PY{l+m+mi}{1}\PY{p}{\PYZcb{}}\PY{p}{)} \PY{o}{==} \PY{n}{f}\PY{o}{.}\PY{n}{subs}\PY{p}{(}\PY{p}{\PYZob{}}\PY{n}{x}\PY{p}{:}\PY{l+m+mi}{1}\PY{p}{\PYZcb{}}\PY{p}{)}
\end{Verbatim}

            \begin{Verbatim}[commandchars=\\\{\}]
{\color{outcolor}Out[{\color{outcolor}98}]:} True
\end{Verbatim}
        
    \begin{Verbatim}[commandchars=\\\{\}]
{\color{incolor}In [{\color{incolor}99}]:} \PY{n}{diff}\PY{p}{(}\PY{n}{T\PYZus{}1}\PY{p}{,}\PY{n}{x}\PY{p}{)}\PY{o}{.}\PY{n}{subs}\PY{p}{(}\PY{p}{\PYZob{}}\PY{n}{x}\PY{p}{:}\PY{l+m+mi}{1}\PY{p}{\PYZcb{}}\PY{p}{)} \PY{o}{==} \PY{n}{diff}\PY{p}{(}\PY{n}{f}\PY{p}{,}\PY{n}{x}\PY{p}{)}\PY{o}{.}\PY{n}{subs}\PY{p}{(}\PY{p}{\PYZob{}}\PY{n}{x}\PY{p}{:}\PY{l+m+mi}{1}\PY{p}{\PYZcb{}}\PY{p}{)}
\end{Verbatim}

            \begin{Verbatim}[commandchars=\\\{\}]
{\color{outcolor}Out[{\color{outcolor}99}]:} True
\end{Verbatim}
        
    \subsubsection{Optimization}\label{optimization}

    Optimization is about choosing an input for a function $f(x)$ that
results in the best value for $f(x)$. The best value usually means the
\emph{maximum} value (if the function represents something desirable
like profits) or the \emph{minimum} value (if the function represents
something undesirable like costs).

The derivative $f'(x)$ encodes the information about the \emph{slope} of
$f(x)$. Positive slope $f'(x)>0$ means $f(x)$ is increasing, negative
slope $f'(x)<0$ means $f(x)$ is decreasing, and zero slope $f'(x)=0$
means the graph of the function is horizontal. The \emph{critical
points} of a function $f(x)$ are the solutions to the equation
$f'(x)=0$. Each critical point is a candidate to be either a maximum or
a minimum of the function.

The second derivative $f^{\prime\prime}(x)$ encodes the information
about the \emph{curvature} of $f(x)$. Positive curvature means the
function looks like $x^2$, negative curvature means the function looks
like $-x^2$.

Let's find the critical points of the function $f(x)=x^3-2x^2+x$ and use
the information from its second derivative to find the maximum of the
function on the interval $x \in [0,1]$.

    \begin{Verbatim}[commandchars=\\\{\}]
{\color{incolor}In [{\color{incolor}100}]:} \PY{n}{x} \PY{o}{=} \PY{n}{Symbol}\PY{p}{(}\PY{l+s}{'}\PY{l+s}{x}\PY{l+s}{'}\PY{p}{)}
          \PY{n}{f} \PY{o}{=} \PY{n}{x}\PY{o}{*}\PY{o}{*}\PY{l+m+mi}{3}\PY{o}{-}\PY{l+m+mi}{2}\PY{o}{*}\PY{n}{x}\PY{o}{*}\PY{o}{*}\PY{l+m+mi}{2}\PY{o}{+}\PY{n}{x}
          \PY{n}{diff}\PY{p}{(}\PY{n}{f}\PY{p}{,} \PY{n}{x}\PY{p}{)}
\end{Verbatim}
\texttt{\color{outcolor}Out[{\color{outcolor}100}]:}
    
    
        \begin{equation*}\adjustbox{max width=\hsize}{$
        3 x^{2} - 4 x + 1
        $}\end{equation*}

    

    \begin{Verbatim}[commandchars=\\\{\}]
{\color{incolor}In [{\color{incolor}101}]:} \PY{n}{sols} \PY{o}{=} \PY{n}{solve}\PY{p}{(} \PY{n}{diff}\PY{p}{(}\PY{n}{f}\PY{p}{,}\PY{n}{x}\PY{p}{)}\PY{p}{,}  \PY{n}{x}\PY{p}{)}
          \PY{n}{sols}
\end{Verbatim}
\texttt{\color{outcolor}Out[{\color{outcolor}101}]:}
    
    
        \begin{equation*}\adjustbox{max width=\hsize}{$
        \left [ \frac{1}{3}, \quad 1\right ]
        $}\end{equation*}

    

    \begin{Verbatim}[commandchars=\\\{\}]
{\color{incolor}In [{\color{incolor}102}]:} \PY{n}{diff}\PY{p}{(}\PY{n}{diff}\PY{p}{(}\PY{n}{f}\PY{p}{,}\PY{n}{x}\PY{p}{)}\PY{p}{,} \PY{n}{x}\PY{p}{)}\PY{o}{.}\PY{n}{subs}\PY{p}{(} \PY{p}{\PYZob{}}\PY{n}{x}\PY{p}{:}\PY{n}{sols}\PY{p}{[}\PY{l+m+mi}{0}\PY{p}{]}\PY{p}{\PYZcb{}} \PY{p}{)}
\end{Verbatim}
\texttt{\color{outcolor}Out[{\color{outcolor}102}]:}
    
    
        \begin{equation*}\adjustbox{max width=\hsize}{$
        -2
        $}\end{equation*}

    

    \begin{Verbatim}[commandchars=\\\{\}]
{\color{incolor}In [{\color{incolor}103}]:} \PY{n}{diff}\PY{p}{(}\PY{n}{diff}\PY{p}{(}\PY{n}{f}\PY{p}{,}\PY{n}{x}\PY{p}{)}\PY{p}{,} \PY{n}{x}\PY{p}{)}\PY{o}{.}\PY{n}{subs}\PY{p}{(} \PY{p}{\PYZob{}}\PY{n}{x}\PY{p}{:}\PY{n}{sols}\PY{p}{[}\PY{l+m+mi}{1}\PY{p}{]}\PY{p}{\PYZcb{}} \PY{p}{)}
\end{Verbatim}
\texttt{\color{outcolor}Out[{\color{outcolor}103}]:}
    
    
        \begin{equation*}\adjustbox{max width=\hsize}{$
        2
        $}\end{equation*}

    

    \href{https://www.google.com/\#safe=off\&q=plot+x**3-2*x**2\%2Bx}{It
will help to look at the graph of this function.} The point
$x=\frac{1}{3}$ is a local maximum because it is a critical point of
$f(x)$ where the curvature is negative, meaning $f(x)$ looks like the
peak of a mountain at $x=\frac{1}{3}$. The maximum value of $f(x)$ on
the interval $x\in [0,1]$ is $f\!\left(\frac{1}{3}\right)=\frac{4}{27}$.
The point $x=1$ is a local minimum because it is a critical point with
positive curvature, meaning $f(x)$ looks like the bottom of a valley at
$x=1$.

    \subsubsection{Integrals}\label{integrals}

    The \emph{integral} of $f(x)$ corresponds to the computation of the area
under the graph of $f(x)$. The area under $f(x)$ between the points
$x=a$ and $x=b$ is denoted as follows:

\[
 A(a,b) = \int_a^b f(x) \: dx.
\]

The \emph{integral function} $F$ corresponds to the area calculation as
a function of the upper limit of integration:

\[
  F(c) \equiv \int_0^c \! f(x)\:dx\,.
\]

The area under $f(x)$ between $x=a$ and $x=b$ is obtained by calculating
the \emph{change} in the integral function:

\[
   A(a,b) = \int_a^b \! f(x)\:dx  =  F(b)-F(a).
\]

In \texttt{SymPy} we use \texttt{integrate(f, x)} to obtain the integral
function $F(x)$ of any function $f(x)$: $F(x) = \int_0^x f(u)\,du$.

    \begin{Verbatim}[commandchars=\\\{\}]
{\color{incolor}In [{\color{incolor}104}]:} \PY{n}{integrate}\PY{p}{(}\PY{n}{x}\PY{o}{*}\PY{o}{*}\PY{l+m+mi}{3}\PY{p}{,} \PY{n}{x}\PY{p}{)}
\end{Verbatim}
\texttt{\color{outcolor}Out[{\color{outcolor}104}]:}
    
    
        \begin{equation*}\adjustbox{max width=\hsize}{$
        \frac{x^{4}}{4}
        $}\end{equation*}

    

    \begin{Verbatim}[commandchars=\\\{\}]
{\color{incolor}In [{\color{incolor}105}]:} \PY{n}{integrate}\PY{p}{(}\PY{n}{sin}\PY{p}{(}\PY{n}{x}\PY{p}{)}\PY{p}{,} \PY{n}{x}\PY{p}{)}
\end{Verbatim}
\texttt{\color{outcolor}Out[{\color{outcolor}105}]:}
    
    
        \begin{equation*}\adjustbox{max width=\hsize}{$
        - \cos{\left (x \right )}
        $}\end{equation*}

    

    \begin{Verbatim}[commandchars=\\\{\}]
{\color{incolor}In [{\color{incolor}106}]:} \PY{n}{integrate}\PY{p}{(}\PY{n}{ln}\PY{p}{(}\PY{n}{x}\PY{p}{)}\PY{p}{,} \PY{n}{x}\PY{p}{)}
\end{Verbatim}
\texttt{\color{outcolor}Out[{\color{outcolor}106}]:}
    
    
        \begin{equation*}\adjustbox{max width=\hsize}{$
        x \log{\left (x \right )} - x
        $}\end{equation*}

    

    This is known as an \emph{indefinite integral} since the limits of
integration are not defined.

In contrast, a \emph{definite integral} computes the area under $f(x)$
between $x=a$ and $x=b$. Use \texttt{integrate(f, (x,a,b))} to compute
the definite integrals of the form $A(a,b)=\int_a^b f(x) \, dx$:

    \begin{Verbatim}[commandchars=\\\{\}]
{\color{incolor}In [{\color{incolor}107}]:} \PY{n}{integrate}\PY{p}{(}\PY{n}{x}\PY{o}{*}\PY{o}{*}\PY{l+m+mi}{3}\PY{p}{,} \PY{p}{(}\PY{n}{x}\PY{p}{,}\PY{l+m+mi}{0}\PY{p}{,}\PY{l+m+mi}{1}\PY{p}{)}\PY{p}{)}  \PY{c}{# the area under x\PYZca{}3 from x=0 to x=1}
\end{Verbatim}
\texttt{\color{outcolor}Out[{\color{outcolor}107}]:}
    
    
        \begin{equation*}\adjustbox{max width=\hsize}{$
        \frac{1}{4}
        $}\end{equation*}

    

    We can obtain the same area by first calculating the indefinite integral
$F(c)=\int_0^c \!f(x)\,dx$, then using
$A(a,b) = F(x)\big\vert_a^b \equiv F(b) - F(a)$:

    \begin{Verbatim}[commandchars=\\\{\}]
{\color{incolor}In [{\color{incolor}108}]:} \PY{n}{F} \PY{o}{=} \PY{n}{integrate}\PY{p}{(}\PY{n}{x}\PY{o}{*}\PY{o}{*}\PY{l+m+mi}{3}\PY{p}{,} \PY{n}{x}\PY{p}{)}
          \PY{n}{F}\PY{o}{.}\PY{n}{subs}\PY{p}{(}\PY{p}{\PYZob{}}\PY{n}{x}\PY{p}{:}\PY{l+m+mi}{1}\PY{p}{\PYZcb{}}\PY{p}{)} \PY{o}{-} \PY{n}{F}\PY{o}{.}\PY{n}{subs}\PY{p}{(}\PY{p}{\PYZob{}}\PY{n}{x}\PY{p}{:}\PY{l+m+mi}{0}\PY{p}{\PYZcb{}}\PY{p}{)}
\end{Verbatim}
\texttt{\color{outcolor}Out[{\color{outcolor}108}]:}
    
    
        \begin{equation*}\adjustbox{max width=\hsize}{$
        \frac{1}{4}
        $}\end{equation*}

    

    Integrals correspond to \emph{signed} area calculations:

    \begin{Verbatim}[commandchars=\\\{\}]
{\color{incolor}In [{\color{incolor}109}]:} \PY{n}{integrate}\PY{p}{(}\PY{n}{sin}\PY{p}{(}\PY{n}{x}\PY{p}{)}\PY{p}{,} \PY{p}{(}\PY{n}{x}\PY{p}{,}\PY{l+m+mi}{0}\PY{p}{,}\PY{n}{pi}\PY{p}{)}\PY{p}{)}
\end{Verbatim}
\texttt{\color{outcolor}Out[{\color{outcolor}109}]:}
    
    
        \begin{equation*}\adjustbox{max width=\hsize}{$
        2
        $}\end{equation*}

    

    \begin{Verbatim}[commandchars=\\\{\}]
{\color{incolor}In [{\color{incolor}110}]:} \PY{n}{integrate}\PY{p}{(}\PY{n}{sin}\PY{p}{(}\PY{n}{x}\PY{p}{)}\PY{p}{,} \PY{p}{(}\PY{n}{x}\PY{p}{,}\PY{n}{pi}\PY{p}{,}\PY{l+m+mi}{2}\PY{o}{*}\PY{n}{pi}\PY{p}{)}\PY{p}{)}
\end{Verbatim}
\texttt{\color{outcolor}Out[{\color{outcolor}110}]:}
    
    
        \begin{equation*}\adjustbox{max width=\hsize}{$
        -2
        $}\end{equation*}

    

    \begin{Verbatim}[commandchars=\\\{\}]
{\color{incolor}In [{\color{incolor}111}]:} \PY{n}{integrate}\PY{p}{(}\PY{n}{sin}\PY{p}{(}\PY{n}{x}\PY{p}{)}\PY{p}{,} \PY{p}{(}\PY{n}{x}\PY{p}{,}\PY{l+m+mi}{0}\PY{p}{,}\PY{l+m+mi}{2}\PY{o}{*}\PY{n}{pi}\PY{p}{)}\PY{p}{)}
\end{Verbatim}
\texttt{\color{outcolor}Out[{\color{outcolor}111}]:}
    
    
        \begin{equation*}\adjustbox{max width=\hsize}{$
        0
        $}\end{equation*}

    

    During the first half of its $2\pi$-cycle, the graph of $\sin(x)$ is
above the $x$-axis, so it has a positive contribution to the area under
the curve. During the second half of its cycle (from $x=\pi$ to
$x=2\pi$), $\sin(x)$ is below the $x$-axis, so it contributes negative
area. Draw a graph of $\sin(x)$ to see what is going on.

    \subsubsection{Fundamental theorem of
calculus}\label{fundamental-theorem-of-calculus}

    The integral is the ``inverse operation'' of the derivative. If you
perform the integral operation followed by the derivative operation on
some function, you'll obtain the same function:

\[
  \left(\frac{d}{dx} \circ \int dx \right) f(x) \equiv \frac{d}{dx} \int_c^x f(u)\:du = f(x).
\]

    \begin{Verbatim}[commandchars=\\\{\}]
{\color{incolor}In [{\color{incolor}112}]:} \PY{n}{f} \PY{o}{=} \PY{n}{x}\PY{o}{*}\PY{o}{*}\PY{l+m+mi}{2}
          \PY{n}{F} \PY{o}{=} \PY{n}{integrate}\PY{p}{(}\PY{n}{f}\PY{p}{,} \PY{n}{x}\PY{p}{)}
          \PY{n}{F}
\end{Verbatim}
\texttt{\color{outcolor}Out[{\color{outcolor}112}]:}
    
    
        \begin{equation*}\adjustbox{max width=\hsize}{$
        \frac{x^{3}}{3}
        $}\end{equation*}

    

    \begin{Verbatim}[commandchars=\\\{\}]
{\color{incolor}In [{\color{incolor}113}]:} \PY{n}{diff}\PY{p}{(}\PY{n}{F}\PY{p}{,}\PY{n}{x}\PY{p}{)}
\end{Verbatim}
\texttt{\color{outcolor}Out[{\color{outcolor}113}]:}
    
    
        \begin{equation*}\adjustbox{max width=\hsize}{$
        x^{2}
        $}\end{equation*}

    

    Alternately, if you compute the derivative of a function followed by the
integral, you will obtain the original function $f(x)$ (up to a
constant):

\[
  \left( \int dx \circ \frac{d}{dx}\right) f(x) \equiv \int_c^x f'(u)\;du = f(x) + C.
\]

    \begin{Verbatim}[commandchars=\\\{\}]
{\color{incolor}In [{\color{incolor}114}]:} \PY{n}{f} \PY{o}{=} \PY{n}{x}\PY{o}{*}\PY{o}{*}\PY{l+m+mi}{2}
          \PY{n}{df} \PY{o}{=} \PY{n}{diff}\PY{p}{(}\PY{n}{f}\PY{p}{,}\PY{n}{x}\PY{p}{)}
          \PY{n}{df}
\end{Verbatim}
\texttt{\color{outcolor}Out[{\color{outcolor}114}]:}
    
    
        \begin{equation*}\adjustbox{max width=\hsize}{$
        2 x
        $}\end{equation*}

    

    \begin{Verbatim}[commandchars=\\\{\}]
{\color{incolor}In [{\color{incolor}115}]:} \PY{n}{integrate}\PY{p}{(}\PY{n}{df}\PY{p}{,} \PY{n}{x}\PY{p}{)}
\end{Verbatim}
\texttt{\color{outcolor}Out[{\color{outcolor}115}]:}
    
    
        \begin{equation*}\adjustbox{max width=\hsize}{$
        x^{2}
        $}\end{equation*}

    

    The fundamental theorem of calculus is important because it tells us how
to solve differential equations. If we have to solve for $f(x)$ in the
differential equation $\frac{d}{dx}f(x) = g(x)$, we can take the
integral on both sides of the equation to obtain the answer
$f(x) = \int g(x)\,dx + C$.

    \subsubsection{Sequences}\label{sequences}

    Sequences are functions that take whole numbers as inputs. Instead of
continuous inputs $x\in \mathbb{R}$, sequences take natural numbers
$n\in\mathbb{N}$ as inputs. We denote sequences as $a_n$ instead of the
usual function notation $a(n)$.

We define a sequence by specifying an expression for its $n^\mathrm{th}$
term:

    \begin{Verbatim}[commandchars=\\\{\}]
{\color{incolor}In [{\color{incolor}116}]:} \PY{n}{a\PYZus{}n} \PY{o}{=} \PY{l+m+mi}{1}\PY{o}{/}\PY{n}{n}
          \PY{n}{b\PYZus{}n} \PY{o}{=} \PY{l+m+mi}{1}\PY{o}{/}\PY{n}{factorial}\PY{p}{(}\PY{n}{n}\PY{p}{)}
\end{Verbatim}

    Substitute the desired value of $n$ to see the value of the
$n^\mathrm{th}$ term:

    \begin{Verbatim}[commandchars=\\\{\}]
{\color{incolor}In [{\color{incolor}117}]:} \PY{n}{a\PYZus{}n}\PY{o}{.}\PY{n}{subs}\PY{p}{(}\PY{p}{\PYZob{}}\PY{n}{n}\PY{p}{:}\PY{l+m+mi}{5}\PY{p}{\PYZcb{}}\PY{p}{)}
\end{Verbatim}
\texttt{\color{outcolor}Out[{\color{outcolor}117}]:}
    
    
        \begin{equation*}\adjustbox{max width=\hsize}{$
        \frac{1}{5}
        $}\end{equation*}

    

    The \texttt{Python} list comprehension syntax
\texttt{{[}item for item in list{]}} can be used to print the sequence
values for some range of indices:

    \begin{Verbatim}[commandchars=\\\{\}]
{\color{incolor}In [{\color{incolor}118}]:} \PY{p}{[} \PY{n}{a\PYZus{}n}\PY{o}{.}\PY{n}{subs}\PY{p}{(}\PY{p}{\PYZob{}}\PY{n}{n}\PY{p}{:}\PY{n}{i}\PY{p}{\PYZcb{}}\PY{p}{)} \PY{k}{for} \PY{n}{i} \PY{o+ow}{in} \PY{n+nb}{range}\PY{p}{(}\PY{l+m+mi}{0}\PY{p}{,}\PY{l+m+mi}{8}\PY{p}{)} \PY{p}{]}
\end{Verbatim}
\texttt{\color{outcolor}Out[{\color{outcolor}118}]:}
    
    
        \begin{equation*}\adjustbox{max width=\hsize}{$
        \left [ \tilde{\infty}, \quad 1, \quad \frac{1}{2}, \quad \frac{1}{3}, \quad \frac{1}{4}, \quad \frac{1}{5}, \quad \frac{1}{6}, \quad \frac{1}{7}\right ]
        $}\end{equation*}

    

    \begin{Verbatim}[commandchars=\\\{\}]
{\color{incolor}In [{\color{incolor}119}]:} \PY{p}{[} \PY{n}{b\PYZus{}n}\PY{o}{.}\PY{n}{subs}\PY{p}{(}\PY{p}{\PYZob{}}\PY{n}{n}\PY{p}{:}\PY{n}{i}\PY{p}{\PYZcb{}}\PY{p}{)} \PY{k}{for} \PY{n}{i} \PY{o+ow}{in} \PY{n+nb}{range}\PY{p}{(}\PY{l+m+mi}{0}\PY{p}{,}\PY{l+m+mi}{8}\PY{p}{)} \PY{p}{]}
\end{Verbatim}
\texttt{\color{outcolor}Out[{\color{outcolor}119}]:}
    
    
        \begin{equation*}\adjustbox{max width=\hsize}{$
        \left [ 1, \quad 1, \quad \frac{1}{2}, \quad \frac{1}{6}, \quad \frac{1}{24}, \quad \frac{1}{120}, \quad \frac{1}{720}, \quad \frac{1}{5040}\right ]
        $}\end{equation*}

    

    Observe that $a_n$ is not properly defined for $n=0$ since $\frac{1}{0}$
is a division-by-zero error. To be precise, we should say $a_n$'s domain
is the positive naturals $a_n:\mathbb{N}^+ \to \mathbb{R}$. Observe how
quickly the \texttt{factorial} function
$n!=1\cdot2\cdot3\cdots(n-1)\cdot n$ grows: $7!= 5040$, $10!=3628800$,
$20! > 10^{18}$.

We're often interested in calculating the limits of sequences as
$n\to \infty$. What happens to the terms in the sequence when $n$
becomes large?

    \begin{Verbatim}[commandchars=\\\{\}]
{\color{incolor}In [{\color{incolor}120}]:} \PY{n}{limit}\PY{p}{(}\PY{n}{a\PYZus{}n}\PY{p}{,} \PY{n}{n}\PY{p}{,} \PY{n}{oo}\PY{p}{)}
\end{Verbatim}
\texttt{\color{outcolor}Out[{\color{outcolor}120}]:}
    
    
        \begin{equation*}\adjustbox{max width=\hsize}{$
        0
        $}\end{equation*}

    

    \begin{Verbatim}[commandchars=\\\{\}]
{\color{incolor}In [{\color{incolor}121}]:} \PY{n}{limit}\PY{p}{(}\PY{n}{b\PYZus{}n}\PY{p}{,} \PY{n}{n}\PY{p}{,} \PY{n}{oo}\PY{p}{)}
\end{Verbatim}
\texttt{\color{outcolor}Out[{\color{outcolor}121}]:}
    
    
        \begin{equation*}\adjustbox{max width=\hsize}{$
        0
        $}\end{equation*}

    

    Both $a_n=\frac{1}{n}$ and $b_n = \frac{1}{n!}$ \emph{converge} to $0$
as $n\to\infty$.

Many important math quantities are defined as limit expressions. An
interesting example to consider is the number $\pi$, which is defined as
the area of a circle of radius $1$. We can approximate the area of the
unit circle by drawing a many-sided regular polygon around the circle.
Splitting the $n$-sided regular polygon into identical triangular
splices, we can obtain a formula for its area $A_n$. In the limit as
$n\to \infty$, the $n$-sided-polygon approximation to the area of the
unit-circle becomes exact:

    \begin{Verbatim}[commandchars=\\\{\}]
{\color{incolor}In [{\color{incolor}122}]:} \PY{n}{A\PYZus{}n} \PY{o}{=} \PY{n}{n}\PY{o}{*}\PY{n}{tan}\PY{p}{(}\PY{l+m+mi}{2}\PY{o}{*}\PY{n}{pi}\PY{o}{/}\PY{p}{(}\PY{l+m+mi}{2}\PY{o}{*}\PY{n}{n}\PY{p}{)}\PY{p}{)}
          \PY{n}{limit}\PY{p}{(}\PY{n}{A\PYZus{}n}\PY{p}{,} \PY{n}{n}\PY{p}{,} \PY{n}{oo}\PY{p}{)}
\end{Verbatim}
\texttt{\color{outcolor}Out[{\color{outcolor}122}]:}
    
    
        \begin{equation*}\adjustbox{max width=\hsize}{$
        \pi
        $}\end{equation*}

    

    \subsubsection{Series}\label{series}

    Suppose we're given a sequence $a_n$ and we want to compute the sum of
all the values in this sequence $\sum_{n}^\infty a_n$. Series are sums
of sequences. Summing the values of a sequence
$a_n:\mathbb{N}\to \mathbb{R}$ is analogous to taking the integral of a
function $f:\mathbb{R}\to \mathbb{R}$.

To work with series in \texttt{SymPy}, use the \texttt{summation}
function whose syntax is analogous to the \texttt{integrate} function:

    \begin{Verbatim}[commandchars=\\\{\}]
{\color{incolor}In [{\color{incolor}123}]:} \PY{n}{a\PYZus{}n} \PY{o}{=} \PY{l+m+mi}{1}\PY{o}{/}\PY{n}{n}
          \PY{n}{summation}\PY{p}{(}\PY{n}{a\PYZus{}n}\PY{p}{,} \PY{p}{[}\PY{n}{n}\PY{p}{,} \PY{l+m+mi}{1}\PY{p}{,} \PY{n}{oo}\PY{p}{]}\PY{p}{)}
\end{Verbatim}
\texttt{\color{outcolor}Out[{\color{outcolor}123}]:}
    
    
        \begin{equation*}\adjustbox{max width=\hsize}{$
        \infty
        $}\end{equation*}

    

    \begin{Verbatim}[commandchars=\\\{\}]
{\color{incolor}In [{\color{incolor}124}]:} \PY{n}{b\PYZus{}n} \PY{o}{=} \PY{l+m+mi}{1}\PY{o}{/}\PY{n}{factorial}\PY{p}{(}\PY{n}{n}\PY{p}{)}
          \PY{n}{summation}\PY{p}{(}\PY{n}{b\PYZus{}n}\PY{p}{,} \PY{p}{[}\PY{n}{n}\PY{p}{,} \PY{l+m+mi}{0}\PY{p}{,} \PY{n}{oo}\PY{p}{]}\PY{p}{)}
\end{Verbatim}
\texttt{\color{outcolor}Out[{\color{outcolor}124}]:}
    
    
        \begin{equation*}\adjustbox{max width=\hsize}{$
        e
        $}\end{equation*}

    

    We say the series $\sum a_n$ \emph{diverges} to infinity (or \emph{is
divergent}) while the series $\sum b_n$ converges (or \emph{is
convergent}). As we sum together more and more terms of the sequence
$b_n$, the total becomes closer and closer to some finite number. In
this case, the infinite sum $\sum_{n=0}^\infty \frac{1}{n!}$ converges
to the number $e=2.71828\ldots$.

The \texttt{summation} command is useful because it allows us to compute
\texttt{infinite} sums, but for most practical applications we don't
need to take an infinite number of terms in a series to obtain a good
approximation. This is why series are so neat: they represent a great
way to obtain approximations.

Using standard \texttt{Python} commands,\\we can obtain an approximation
to $e$ that is accurate to six decimals by summing 10 terms in the
series:

    \begin{Verbatim}[commandchars=\\\{\}]
{\color{incolor}In [{\color{incolor}125}]:} \PY{k+kn}{import} \PY{n+nn}{math}
          \PY{k}{def} \PY{n+nf}{b\PYZus{}nf}\PY{p}{(}\PY{n}{n}\PY{p}{)}\PY{p}{:} 
              \PY{k}{return} \PY{l+m+mf}{1.0}\PY{o}{/}\PY{n}{math}\PY{o}{.}\PY{n}{factorial}\PY{p}{(}\PY{n}{n}\PY{p}{)}
          \PY{n+nb}{sum}\PY{p}{(} \PY{p}{[}\PY{n}{b\PYZus{}nf}\PY{p}{(}\PY{n}{n}\PY{p}{)} \PY{k}{for} \PY{n}{n} \PY{o+ow}{in} \PY{n+nb}{range}\PY{p}{(}\PY{l+m+mi}{0}\PY{p}{,}\PY{l+m+mi}{10}\PY{p}{)}\PY{p}{]} \PY{p}{)}
\end{Verbatim}
\texttt{\color{outcolor}Out[{\color{outcolor}125}]:}
    
    
        \begin{equation*}\adjustbox{max width=\hsize}{$
        2.7182815255731922
        $}\end{equation*}

    

    \begin{Verbatim}[commandchars=\\\{\}]
{\color{incolor}In [{\color{incolor}126}]:} \PY{n}{E}\PY{o}{.}\PY{n}{evalf}\PY{p}{(}\PY{p}{)}  \PY{c}{# true value}
\end{Verbatim}
\texttt{\color{outcolor}Out[{\color{outcolor}126}]:}
    
    
        \begin{equation*}\adjustbox{max width=\hsize}{$
        2.71828182845905
        $}\end{equation*}

    

    \subsubsection{Taylor series}\label{taylor-series}

    Wait, there's more! Not only can we use series to approximate numbers,
we can also use them to approximate functions.

A \emph{power series} is a series whose terms contain different powers
of the variable $x$. The $n^\mathrm{th}$ term in a power series is a
function of both the sequence index $n$ and the input variable $x$.

For example, the power series of the function $\exp(x)=e^x$ is

\[
 \exp(x) \equiv  1 + x + \frac{x^2}{2} + \frac{x^3}{3!} + \frac{x^4}{4!} + \frac{x^5}{5!} + \cdots         
  =       \sum_{n=0}^\infty \frac{x^n}{n!}.
\]

This is, IMHO, one of the most important ideas in calculus: you can
compute the value of $\exp(5)$ by taking the infinite sum of the terms
in the power series with $x=5$:

    \begin{Verbatim}[commandchars=\\\{\}]
{\color{incolor}In [{\color{incolor}127}]:} \PY{n}{exp\PYZus{}xn} \PY{o}{=} \PY{n}{x}\PY{o}{*}\PY{o}{*}\PY{n}{n}\PY{o}{/}\PY{n}{factorial}\PY{p}{(}\PY{n}{n}\PY{p}{)}
          \PY{n}{summation}\PY{p}{(} \PY{n}{exp\PYZus{}xn}\PY{o}{.}\PY{n}{subs}\PY{p}{(}\PY{p}{\PYZob{}}\PY{n}{x}\PY{p}{:}\PY{l+m+mi}{5}\PY{p}{\PYZcb{}}\PY{p}{)}\PY{p}{,} \PY{p}{[}\PY{n}{n}\PY{p}{,} \PY{l+m+mi}{0}\PY{p}{,} \PY{n}{oo}\PY{p}{]} \PY{p}{)}\PY{o}{.}\PY{n}{evalf}\PY{p}{(}\PY{p}{)}
\end{Verbatim}
\texttt{\color{outcolor}Out[{\color{outcolor}127}]:}
    
    
        \begin{equation*}\adjustbox{max width=\hsize}{$
        148.413159102577
        $}\end{equation*}

    

    \begin{Verbatim}[commandchars=\\\{\}]
{\color{incolor}In [{\color{incolor}128}]:} \PY{n}{exp}\PY{p}{(}\PY{l+m+mi}{5}\PY{p}{)}\PY{o}{.}\PY{n}{evalf}\PY{p}{(}\PY{p}{)}  \PY{c}{# the true value}
\end{Verbatim}
\texttt{\color{outcolor}Out[{\color{outcolor}128}]:}
    
    
        \begin{equation*}\adjustbox{max width=\hsize}{$
        148.413159102577
        $}\end{equation*}

    

    Note that \texttt{SymPy} is actually smart enough to recognize that the
infinite series you're computing corresponds to the closed-form
expression $e^5$:

    \begin{Verbatim}[commandchars=\\\{\}]
{\color{incolor}In [{\color{incolor}129}]:} \PY{n}{summation}\PY{p}{(} \PY{n}{exp\PYZus{}xn}\PY{o}{.}\PY{n}{subs}\PY{p}{(}\PY{p}{\PYZob{}}\PY{n}{x}\PY{p}{:}\PY{l+m+mi}{5}\PY{p}{\PYZcb{}}\PY{p}{)}\PY{p}{,} \PY{p}{[}\PY{n}{n}\PY{p}{,} \PY{l+m+mi}{0}\PY{p}{,} \PY{n}{oo}\PY{p}{]}\PY{p}{)}
\end{Verbatim}
\texttt{\color{outcolor}Out[{\color{outcolor}129}]:}
    
    
        \begin{equation*}\adjustbox{max width=\hsize}{$
        e^{5}
        $}\end{equation*}

    

    Taking as few as 35 terms in the series is sufficient to obtain an
approximation to $e$ that is accurate to 16 decimals:

    \begin{Verbatim}[commandchars=\\\{\}]
{\color{incolor}In [{\color{incolor}130}]:} \PY{k+kn}{import} \PY{n+nn}{math}  \PY{c}{# redo using only python }
          \PY{k}{def} \PY{n+nf}{exp\PYZus{}xnf}\PY{p}{(}\PY{n}{x}\PY{p}{,}\PY{n}{n}\PY{p}{)}\PY{p}{:} 
              \PY{k}{return} \PY{n}{x}\PY{o}{*}\PY{o}{*}\PY{n}{n}\PY{o}{/}\PY{n}{math}\PY{o}{.}\PY{n}{factorial}\PY{p}{(}\PY{n}{n}\PY{p}{)}
          \PY{n+nb}{sum}\PY{p}{(} \PY{p}{[}\PY{n}{exp\PYZus{}xnf}\PY{p}{(}\PY{l+m+mf}{5.0}\PY{p}{,}\PY{n}{i}\PY{p}{)} \PY{k}{for} \PY{n}{i} \PY{o+ow}{in} \PY{n+nb}{range}\PY{p}{(}\PY{l+m+mi}{0}\PY{p}{,}\PY{l+m+mi}{35}\PY{p}{)}\PY{p}{]} \PY{p}{)}
\end{Verbatim}
\texttt{\color{outcolor}Out[{\color{outcolor}130}]:}
    
    
        \begin{equation*}\adjustbox{max width=\hsize}{$
        148.41315910257657
        $}\end{equation*}

    

    The coefficients in the power series of a function (also known as the
\emph{Taylor series}) The formula for the $n^\mathrm{th}$ term in the
Taylor series of $f(x)$ expanded at $x=c$ is
$a_n(x) = \frac{f^{(n)}(c)}{n!}(x-c)^n$, where $f^{(n)}(c)$ is the value
of the $n^\mathrm{th}$ derivative of $f(x)$ evaluated at $x=c$. The term
\emph{Maclaurin series} refers to Taylor series expansions at $x=0$.

The \texttt{SymPy} function \texttt{series} is a convenient way to
obtain the series of any function. Calling
\texttt{series(expr,var,at,nmax)} will show you the series expansion of
\texttt{expr} near \texttt{var}=\texttt{at} up to power \texttt{nmax}:

    \begin{Verbatim}[commandchars=\\\{\}]
{\color{incolor}In [{\color{incolor}131}]:} \PY{n}{series}\PY{p}{(} \PY{n}{sin}\PY{p}{(}\PY{n}{x}\PY{p}{)}\PY{p}{,} \PY{n}{x}\PY{p}{,} \PY{l+m+mi}{0}\PY{p}{,} \PY{l+m+mi}{8}\PY{p}{)}
\end{Verbatim}
\texttt{\color{outcolor}Out[{\color{outcolor}131}]:}
    
    
        \begin{equation*}\adjustbox{max width=\hsize}{$
        x - \frac{x^{3}}{6} + \frac{x^{5}}{120} - \frac{x^{7}}{5040} + \mathcal{O}\left(x^{8}\right)
        $}\end{equation*}

    

    \begin{Verbatim}[commandchars=\\\{\}]
{\color{incolor}In [{\color{incolor}132}]:} \PY{n}{series}\PY{p}{(} \PY{n}{cos}\PY{p}{(}\PY{n}{x}\PY{p}{)}\PY{p}{,} \PY{n}{x}\PY{p}{,} \PY{l+m+mi}{0}\PY{p}{,} \PY{l+m+mi}{8}\PY{p}{)}
\end{Verbatim}
\texttt{\color{outcolor}Out[{\color{outcolor}132}]:}
    
    
        \begin{equation*}\adjustbox{max width=\hsize}{$
        1 - \frac{x^{2}}{2} + \frac{x^{4}}{24} - \frac{x^{6}}{720} + \mathcal{O}\left(x^{8}\right)
        $}\end{equation*}

    

    \begin{Verbatim}[commandchars=\\\{\}]
{\color{incolor}In [{\color{incolor}133}]:} \PY{n}{series}\PY{p}{(} \PY{n}{sinh}\PY{p}{(}\PY{n}{x}\PY{p}{)}\PY{p}{,} \PY{n}{x}\PY{p}{,} \PY{l+m+mi}{0}\PY{p}{,} \PY{l+m+mi}{8}\PY{p}{)}
\end{Verbatim}
\texttt{\color{outcolor}Out[{\color{outcolor}133}]:}
    
    
        \begin{equation*}\adjustbox{max width=\hsize}{$
        x + \frac{x^{3}}{6} + \frac{x^{5}}{120} + \frac{x^{7}}{5040} + \mathcal{O}\left(x^{8}\right)
        $}\end{equation*}

    

    \begin{Verbatim}[commandchars=\\\{\}]
{\color{incolor}In [{\color{incolor}134}]:} \PY{n}{series}\PY{p}{(} \PY{n}{cosh}\PY{p}{(}\PY{n}{x}\PY{p}{)}\PY{p}{,} \PY{n}{x}\PY{p}{,} \PY{l+m+mi}{0}\PY{p}{,} \PY{l+m+mi}{8}\PY{p}{)}
\end{Verbatim}
\texttt{\color{outcolor}Out[{\color{outcolor}134}]:}
    
    
        \begin{equation*}\adjustbox{max width=\hsize}{$
        1 + \frac{x^{2}}{2} + \frac{x^{4}}{24} + \frac{x^{6}}{720} + \mathcal{O}\left(x^{8}\right)
        $}\end{equation*}

    

    Some functions are not defined at $x=0$, so we expand them at a
different value of $x$. For example, the power series of $\ln(x)$
expanded at $x=1$ is

    \begin{Verbatim}[commandchars=\\\{\}]
{\color{incolor}In [{\color{incolor}135}]:} \PY{n}{series}\PY{p}{(}\PY{n}{ln}\PY{p}{(}\PY{n}{x}\PY{p}{)}\PY{p}{,} \PY{n}{x}\PY{p}{,} \PY{l+m+mi}{1}\PY{p}{,} \PY{l+m+mi}{6}\PY{p}{)}  \PY{c}{# Taylor series of ln(x) at x=1}
\end{Verbatim}
\texttt{\color{outcolor}Out[{\color{outcolor}135}]:}
    
    
        \begin{equation*}\adjustbox{max width=\hsize}{$
        -1 - \frac{1}{2} \left(x - 1\right)^{2} + \frac{1}{3} \left(x - 1\right)^{3} - \frac{1}{4} \left(x - 1\right)^{4} + \frac{1}{5} \left(x - 1\right)^{5} + x + \mathcal{O}\left(\left(x - 1\right)^{6}; x\rightarrow1\right)
        $}\end{equation*}

    

    Here, the result \texttt{SymPy} returns is misleading. The Taylor series
of $\ln(x)$ expanded at $x=1$ has terms of the form $(x-1)^n$:

\[
  \ln(x) = (x-1) - \frac{(x-1)^2}{2} + \frac{(x-1)^3}{3} - \frac{(x-1)^4}{4} + \frac{(x-1)^5}{5} + \cdots.
\]

Verify this is the correct formula by substituting $x=1$. \texttt{SymPy}
returns an answer in terms of coordinates \texttt{relative} to $x=1$.

Instead of expanding $\ln(x)$ around $x=1$, we can obtain an equivalent
expression if we expand $\ln(x+1)$ around $x=0$:

    \begin{Verbatim}[commandchars=\\\{\}]
{\color{incolor}In [{\color{incolor}136}]:} \PY{n}{series}\PY{p}{(}\PY{n}{ln}\PY{p}{(}\PY{n}{x}\PY{o}{+}\PY{l+m+mi}{1}\PY{p}{)}\PY{p}{,} \PY{n}{x}\PY{p}{,} \PY{l+m+mi}{0}\PY{p}{,} \PY{l+m+mi}{6}\PY{p}{)}  \PY{c}{# Maclaurin series of ln(x+1)}
\end{Verbatim}
\texttt{\color{outcolor}Out[{\color{outcolor}136}]:}
    
    
        \begin{equation*}\adjustbox{max width=\hsize}{$
        x - \frac{x^{2}}{2} + \frac{x^{3}}{3} - \frac{x^{4}}{4} + \frac{x^{5}}{5} + \mathcal{O}\left(x^{6}\right)
        $}\end{equation*}

    

    \subsection{Vectors}\label{vectors}

    A vector $\vec{v} \in \mathbb{R}^n$ is an $n$-tuple of real numbers. For
example, consider a vector that has three components:

\[
 \vec{v} = (v_1,v_2,v_3) \  \in \  (\mathbb{R},\mathbb{R},\mathbb{R}) \equiv \mathbb{R}^3.
\]

To specify the vector $\vec{v}$, we specify the values for its three
components $v_1$, $v_2$, and $v_3$.

A matrix $A \in \mathbb{R}^{m\times n}$ is a rectangular array of real
numbers with $m$ rows and $n$ columns. A vector is a special type of
matrix; we can think of a vector $\vec{v}\in \mathbb{R}^n$ either as a
row vector ($1\times n$ matrix) or a column vector ($n \times 1$
matrix). Because of this equivalence between vectors and matrices, there
is no need for a special vector object in \texttt{SymPy}, and
\texttt{Matrix} objects are used for vectors as well.

This is how we define vectors and compute their properties:

    \begin{Verbatim}[commandchars=\\\{\}]
{\color{incolor}In [{\color{incolor}137}]:} \PY{n}{u} \PY{o}{=} \PY{n}{Matrix}\PY{p}{(}\PY{p}{[}\PY{p}{[}\PY{l+m+mi}{4}\PY{p}{,}\PY{l+m+mi}{5}\PY{p}{,}\PY{l+m+mi}{6}\PY{p}{]}\PY{p}{]}\PY{p}{)}  \PY{c}{# a row vector = 1x3 matrix}
          \PY{n}{v} \PY{o}{=} \PY{n}{Matrix}\PY{p}{(}\PY{p}{[}\PY{p}{[}\PY{l+m+mi}{7}\PY{p}{]}\PY{p}{,}
                      \PY{p}{[}\PY{l+m+mi}{8}\PY{p}{]}\PY{p}{,}       \PY{c}{# a col vector = 3x1 matrix }
                      \PY{p}{[}\PY{l+m+mi}{9}\PY{p}{]}\PY{p}{]}\PY{p}{)}
\end{Verbatim}

    \begin{Verbatim}[commandchars=\\\{\}]
{\color{incolor}In [{\color{incolor}138}]:} \PY{n}{v}\PY{o}{.}\PY{n}{T}                    \PY{c}{# use the transpose operation to convert a col vec to a row vec}
\end{Verbatim}
\texttt{\color{outcolor}Out[{\color{outcolor}138}]:}
    
    
        \begin{equation*}\adjustbox{max width=\hsize}{$
        \left[\begin{matrix}7 & 8 & 9\end{matrix}\right]
        $}\end{equation*}

    

    \begin{Verbatim}[commandchars=\\\{\}]
{\color{incolor}In [{\color{incolor}139}]:} \PY{n}{u}\PY{p}{[}\PY{l+m+mi}{0}\PY{p}{]}                   \PY{c}{# 0-based indexing for entries}
\end{Verbatim}
\texttt{\color{outcolor}Out[{\color{outcolor}139}]:}
    
    
        \begin{equation*}\adjustbox{max width=\hsize}{$
        4
        $}\end{equation*}

    

    \begin{Verbatim}[commandchars=\\\{\}]
{\color{incolor}In [{\color{incolor}140}]:} \PY{n}{u}\PY{o}{.}\PY{n}{norm}\PY{p}{(}\PY{p}{)}               \PY{c}{# length of u}
\end{Verbatim}
\texttt{\color{outcolor}Out[{\color{outcolor}140}]:}
    
    
        \begin{equation*}\adjustbox{max width=\hsize}{$
        \sqrt{77}
        $}\end{equation*}

    

    \begin{Verbatim}[commandchars=\\\{\}]
{\color{incolor}In [{\color{incolor}141}]:} \PY{n}{uhat} \PY{o}{=} \PY{n}{u}\PY{o}{/}\PY{n}{u}\PY{o}{.}\PY{n}{norm}\PY{p}{(}\PY{p}{)}      \PY{c}{# unit-length vec in same dir as u}
          \PY{n}{uhat}
\end{Verbatim}
\texttt{\color{outcolor}Out[{\color{outcolor}141}]:}
    
    
        \begin{equation*}\adjustbox{max width=\hsize}{$
        \left[\begin{matrix}\frac{4 \sqrt{77}}{77} & \frac{5 \sqrt{77}}{77} & \frac{6 \sqrt{77}}{77}\end{matrix}\right]
        $}\end{equation*}

    

    \begin{Verbatim}[commandchars=\\\{\}]
{\color{incolor}In [{\color{incolor}142}]:} \PY{n}{uhat}\PY{o}{.}\PY{n}{norm}\PY{p}{(}\PY{p}{)}
\end{Verbatim}
\texttt{\color{outcolor}Out[{\color{outcolor}142}]:}
    
    
        \begin{equation*}\adjustbox{max width=\hsize}{$
        1
        $}\end{equation*}

    

    \subsubsection{Dot product}\label{dot-product}

    The dot product of the 3-vectors $\vec{u}$ and $\vec{v}$ can be defined
two ways:

\[
  \vec{u}\cdot\vec{v}
    \equiv 
    \underbrace{u_xv_x+u_yv_y+u_zv_z}_{\textrm{algebraic def.}} 
    \equiv 
    \underbrace{\|\vec{u}\|\|\vec{v}\|\cos(\varphi)}_{\textrm{geometric def.}} 
    \quad \in \mathbb{R},
\]

where $\varphi$ is the angle between the vectors $\vec{u}$ and
$\vec{v}$. In \texttt{SymPy},

    \begin{Verbatim}[commandchars=\\\{\}]
{\color{incolor}In [{\color{incolor}143}]:} \PY{n}{u} \PY{o}{=} \PY{n}{Matrix}\PY{p}{(}\PY{p}{[} \PY{l+m+mi}{4}\PY{p}{,}\PY{l+m+mi}{5}\PY{p}{,}\PY{l+m+mi}{6}\PY{p}{]}\PY{p}{)}
          \PY{n}{v} \PY{o}{=} \PY{n}{Matrix}\PY{p}{(}\PY{p}{[}\PY{o}{-}\PY{l+m+mi}{1}\PY{p}{,}\PY{l+m+mi}{1}\PY{p}{,}\PY{l+m+mi}{2}\PY{p}{]}\PY{p}{)}
          \PY{n}{u}\PY{o}{.}\PY{n}{dot}\PY{p}{(}\PY{n}{v}\PY{p}{)}
\end{Verbatim}
\texttt{\color{outcolor}Out[{\color{outcolor}143}]:}
    
    
        \begin{equation*}\adjustbox{max width=\hsize}{$
        13
        $}\end{equation*}

    

    We can combine the algebraic and geometric formulas for the dot product
to obtain the cosine of the angle between the vectors

\[
    \cos(\varphi)
        = \frac{ \vec{u}\cdot\vec{v} }{  \|\vec{u}\|\|\vec{v}\| }
        = \frac{ u_xv_x+u_yv_y+u_zv_z  }{  \|\vec{u}\|\|\vec{v}\| },
\]

and use the \texttt{acos} function to find the angle measure:

    \begin{Verbatim}[commandchars=\\\{\}]
{\color{incolor}In [{\color{incolor}144}]:} \PY{n}{acos}\PY{p}{(}\PY{n}{u}\PY{o}{.}\PY{n}{dot}\PY{p}{(}\PY{n}{v}\PY{p}{)}\PY{o}{/}\PY{p}{(}\PY{n}{u}\PY{o}{.}\PY{n}{norm}\PY{p}{(}\PY{p}{)}\PY{o}{*}\PY{n}{v}\PY{o}{.}\PY{n}{norm}\PY{p}{(}\PY{p}{)}\PY{p}{)}\PY{p}{)}\PY{o}{.}\PY{n}{evalf}\PY{p}{(}\PY{p}{)}  \PY{c}{# in radians = 52.76 degrees}
\end{Verbatim}
\texttt{\color{outcolor}Out[{\color{outcolor}144}]:}
    
    
        \begin{equation*}\adjustbox{max width=\hsize}{$
        0.921263115666387
        $}\end{equation*}

    

    Just by looking at the coordinates of the vectors $\vec{u}$ and
$\vec{v}$, it's difficult to determine their relative direction. Thanks
to the dot product, however, we know the angle between the vectors is
$52.76^\circ$, which means they \emph{kind of} point in the same
direction. Vectors that are at an angle $\varphi=90^\circ$ are called
\emph{orthogonal}, meaning at right angles with each other. The dot
product of vectors for which $\varphi > 90^\circ$ is negative because
they point \emph{mostly} in opposite directions.

The notion of the ``angle between vectors'' applies more generally to
vectors with any number of dimensions. The dot product for
$n$-dimensional vectors is $\vec{u}\cdot\vec{v}=\sum_{i=1}^n u_iv_i$.
This means we can talk about ``the angle between'' 1000-dimensional
vectors. That's pretty crazy if you think about it---there is no way we
could possibly ``visualize'' 1000-dimensional vectors, yet given two
such vectors we can tell if they point mostly in the same direction, in
perpendicular directions, or mostly in opposite directions.

The dot product is a commutative operation
$\vec{u}\cdot\vec{v} = \vec{v}\cdot\vec{u}$:

    \begin{Verbatim}[commandchars=\\\{\}]
{\color{incolor}In [{\color{incolor}145}]:} \PY{n}{u}\PY{o}{.}\PY{n}{dot}\PY{p}{(}\PY{n}{v}\PY{p}{)} \PY{o}{==} \PY{n}{v}\PY{o}{.}\PY{n}{dot}\PY{p}{(}\PY{n}{u}\PY{p}{)}
\end{Verbatim}

            \begin{Verbatim}[commandchars=\\\{\}]
{\color{outcolor}Out[{\color{outcolor}145}]:} True
\end{Verbatim}
        
    \subsubsection{Projections}\label{projections}

    Dot products are used for computing projections. Assume you're given two
vectors $\vec{u}$ and $\vec{n}$ and you want to find the component of
$\vec{u}$ that points in the $\vec{n}$ direction. The following formula
based on the dot product will give you the answer:

\[
 \Pi_{\vec{n}}( \vec{u} ) \equiv \frac{  \vec{u} \cdot \vec{n}  }{ \| \vec{n} \|^2 } \vec{n}.
\]

This is how to implement this formula in \texttt{SymPy}:

    \begin{Verbatim}[commandchars=\\\{\}]
{\color{incolor}In [{\color{incolor}146}]:} \PY{n}{u} \PY{o}{=} \PY{n}{Matrix}\PY{p}{(}\PY{p}{[}\PY{l+m+mi}{4}\PY{p}{,}\PY{l+m+mi}{5}\PY{p}{,}\PY{l+m+mi}{6}\PY{p}{]}\PY{p}{)}
          \PY{n}{n} \PY{o}{=} \PY{n}{Matrix}\PY{p}{(}\PY{p}{[}\PY{l+m+mi}{1}\PY{p}{,}\PY{l+m+mi}{1}\PY{p}{,}\PY{l+m+mi}{1}\PY{p}{]}\PY{p}{)}
          \PY{p}{(}\PY{n}{u}\PY{o}{.}\PY{n}{dot}\PY{p}{(}\PY{n}{n}\PY{p}{)} \PY{o}{/} \PY{n}{n}\PY{o}{.}\PY{n}{norm}\PY{p}{(}\PY{p}{)}\PY{o}{*}\PY{o}{*}\PY{l+m+mi}{2}\PY{p}{)}\PY{o}{*}\PY{n}{n}  \PY{c}{# projection of v in the n dir}
\end{Verbatim}
\texttt{\color{outcolor}Out[{\color{outcolor}146}]:}
    
    
        \begin{equation*}\adjustbox{max width=\hsize}{$
        \left[\begin{matrix}5\\5\\5\end{matrix}\right]
        $}\end{equation*}

    

    In the case where the direction vector $\hat{n}$ is of unit length
$\|\hat{n}\| = 1$, the projection formula simplifies to
$\Pi_{\hat{n}}( \vec{u} ) \equiv (\vec{u}\cdot\hat{n})\hat{n}$.

Consider now the plane $P$ defined by $(1,1,1)\cdot[(x,y,z)-(0,0,0)]=0$.
A plane is a two dimensional subspace of $\mathbb{R}^3$. We can
decompose any vector $\vec{u} \in \mathbb{R}^3$ into two parts
$\vec{u}=\vec{v} + \vec{w}$ such that $\vec{v}$ lies inside the plane
and $\vec{w}$ is perpendicular to the plane (parallel to
$\vec{n}=(1,1,1)$).

To obtain the perpendicular-to-$P$ component of $\vec{u}$, compute the
projection of $\vec{u}$ in the direction $\vec{n}$:

    \begin{Verbatim}[commandchars=\\\{\}]
{\color{incolor}In [{\color{incolor}147}]:} \PY{n}{w} \PY{o}{=} \PY{p}{(}\PY{n}{u}\PY{o}{.}\PY{n}{dot}\PY{p}{(}\PY{n}{n}\PY{p}{)} \PY{o}{/} \PY{n}{n}\PY{o}{.}\PY{n}{norm}\PY{p}{(}\PY{p}{)}\PY{o}{*}\PY{o}{*}\PY{l+m+mi}{2}\PY{p}{)}\PY{o}{*}\PY{n}{n}
          \PY{n}{w}
\end{Verbatim}
\texttt{\color{outcolor}Out[{\color{outcolor}147}]:}
    
    
        \begin{equation*}\adjustbox{max width=\hsize}{$
        \left[\begin{matrix}5\\5\\5\end{matrix}\right]
        $}\end{equation*}

    

    To obtain the in-the-plane-$P$ component of $\vec{u}$, start with
$\vec{u}$ and subtract the perpendicular-to-$P$ part:

    \begin{Verbatim}[commandchars=\\\{\}]
{\color{incolor}In [{\color{incolor}148}]:} \PY{n}{v} \PY{o}{=} \PY{n}{u} \PY{o}{-} \PY{p}{(}\PY{n}{u}\PY{o}{.}\PY{n}{dot}\PY{p}{(}\PY{n}{n}\PY{p}{)}\PY{o}{/}\PY{n}{n}\PY{o}{.}\PY{n}{norm}\PY{p}{(}\PY{p}{)}\PY{o}{*}\PY{o}{*}\PY{l+m+mi}{2}\PY{p}{)}\PY{o}{*}\PY{n}{n}  \PY{c}{# same as u - w}
          \PY{n}{v}
\end{Verbatim}
\texttt{\color{outcolor}Out[{\color{outcolor}148}]:}
    
    
        \begin{equation*}\adjustbox{max width=\hsize}{$
        \left[\begin{matrix}-1\\0\\1\end{matrix}\right]
        $}\end{equation*}

    

    You should check on your own that $\vec{v}+\vec{w}=\vec{u}$ as claimed.

    \subsubsection{Cross product}\label{cross-product}

    The \emph{cross product}, denoted $\times$, takes two vectors as inputs
and produces a vector as output. The cross products of individual basis
elements are defined as follows:

\[
 \hat{\imath}\times\hat{\jmath} =\hat{k}, \qquad
 \hat{\jmath}\times\hat{k} =\hat{\imath}, \qquad
 \hat{k}\times \hat{\imath}= \hat{\jmath}.
\]

Here is how to compute the cross product of two vectors in
\texttt{SymPy}:

    \begin{Verbatim}[commandchars=\\\{\}]
{\color{incolor}In [{\color{incolor}149}]:} \PY{n}{u} \PY{o}{=} \PY{n}{Matrix}\PY{p}{(}\PY{p}{[} \PY{l+m+mi}{4}\PY{p}{,}\PY{l+m+mi}{5}\PY{p}{,}\PY{l+m+mi}{6}\PY{p}{]}\PY{p}{)}
          \PY{n}{v} \PY{o}{=} \PY{n}{Matrix}\PY{p}{(}\PY{p}{[}\PY{o}{-}\PY{l+m+mi}{1}\PY{p}{,}\PY{l+m+mi}{1}\PY{p}{,}\PY{l+m+mi}{2}\PY{p}{]}\PY{p}{)}
          \PY{n}{u}\PY{o}{.}\PY{n}{cross}\PY{p}{(}\PY{n}{v}\PY{p}{)}
\end{Verbatim}
\texttt{\color{outcolor}Out[{\color{outcolor}149}]:}
    
    
        \begin{equation*}\adjustbox{max width=\hsize}{$
        \left[\begin{matrix}4\\-14\\9\end{matrix}\right]
        $}\end{equation*}

    

    The vector $\vec{u}\times \vec{v}$ is orthogonal to both $\vec{u}$ and
$\vec{v}$. The norm of the cross product $\|\vec{u}\times \vec{v}\|$ is
proportional to the lengths of the vectors and the sine of the angle
between them:

    \begin{Verbatim}[commandchars=\\\{\}]
{\color{incolor}In [{\color{incolor}150}]:} \PY{p}{(}\PY{n}{u}\PY{o}{.}\PY{n}{cross}\PY{p}{(}\PY{n}{v}\PY{p}{)}\PY{o}{.}\PY{n}{norm}\PY{p}{(}\PY{p}{)}\PY{o}{/}\PY{p}{(}\PY{n}{u}\PY{o}{.}\PY{n}{norm}\PY{p}{(}\PY{p}{)}\PY{o}{*}\PY{n}{v}\PY{o}{.}\PY{n}{norm}\PY{p}{(}\PY{p}{)}\PY{p}{)}\PY{p}{)}\PY{o}{.}\PY{n}{n}\PY{p}{(}\PY{p}{)}
\end{Verbatim}
\texttt{\color{outcolor}Out[{\color{outcolor}150}]:}
    
    
        \begin{equation*}\adjustbox{max width=\hsize}{$
        0.796366206088088
        $}\end{equation*}

    

    The name ``cross product'' is well-suited for this operation since it is
calculated by ``cross-multiplying'' the coefficients of the vectors:

\[
   \vec{u}\times\vec{v}=
   \left( 
     u_yv_z-u_zv_y, \ u_zv_x-u_xv_z, \ u_xv_y-u_yv_x 
    \right).
\]

By defining individual symbols for the entries of two vectors, we can
make \texttt{SymPy} show us the cross-product formula:

    \begin{Verbatim}[commandchars=\\\{\}]
{\color{incolor}In [{\color{incolor}151}]:} \PY{n}{u1}\PY{p}{,}\PY{n}{u2}\PY{p}{,}\PY{n}{u3} \PY{o}{=} \PY{n}{symbols}\PY{p}{(}\PY{l+s}{'}\PY{l+s}{u1:4}\PY{l+s}{'}\PY{p}{)}
          \PY{n}{v1}\PY{p}{,}\PY{n}{v2}\PY{p}{,}\PY{n}{v3} \PY{o}{=} \PY{n}{symbols}\PY{p}{(}\PY{l+s}{'}\PY{l+s}{v1:4}\PY{l+s}{'}\PY{p}{)}
          \PY{n}{Matrix}\PY{p}{(}\PY{p}{[}\PY{n}{u1}\PY{p}{,}\PY{n}{u2}\PY{p}{,}\PY{n}{u3}\PY{p}{]}\PY{p}{)}\PY{o}{.}\PY{n}{cross}\PY{p}{(}\PY{n}{Matrix}\PY{p}{(}\PY{p}{[}\PY{n}{v1}\PY{p}{,}\PY{n}{v2}\PY{p}{,}\PY{n}{v3}\PY{p}{]}\PY{p}{)}\PY{p}{)}
\end{Verbatim}
\texttt{\color{outcolor}Out[{\color{outcolor}151}]:}
    
    
        \begin{equation*}\adjustbox{max width=\hsize}{$
        \left[\begin{matrix}u_{2} v_{3} - u_{3} v_{2}\\- u_{1} v_{3} + u_{3} v_{1}\\u_{1} v_{2} - u_{2} v_{1}\end{matrix}\right]
        $}\end{equation*}

    

    The dot product is anti-commutative
$\vec{u}\times\vec{v} = -\vec{v}\times\vec{u}$:

    \begin{Verbatim}[commandchars=\\\{\}]
{\color{incolor}In [{\color{incolor}152}]:} \PY{n}{u}\PY{o}{.}\PY{n}{cross}\PY{p}{(}\PY{n}{v}\PY{p}{)}
\end{Verbatim}
\texttt{\color{outcolor}Out[{\color{outcolor}152}]:}
    
    
        \begin{equation*}\adjustbox{max width=\hsize}{$
        \left[\begin{matrix}4\\-14\\9\end{matrix}\right]
        $}\end{equation*}

    

    \begin{Verbatim}[commandchars=\\\{\}]
{\color{incolor}In [{\color{incolor}153}]:} \PY{n}{v}\PY{o}{.}\PY{n}{cross}\PY{p}{(}\PY{n}{u}\PY{p}{)}
\end{Verbatim}
\texttt{\color{outcolor}Out[{\color{outcolor}153}]:}
    
    
        \begin{equation*}\adjustbox{max width=\hsize}{$
        \left[\begin{matrix}-4\\14\\-9\end{matrix}\right]
        $}\end{equation*}

    

    The product of two numbers and the dot product of two vectors are
commutative operations. The cross product, however, is not commutative:
$\vec{u}\times\vec{v} \neq \vec{v}\times\vec{u}$.

    \subsection{Mechanics}\label{mechanics}

    The module called
\href{http://pyvideo.org/video/2653/dynamics-and-control-with-python}{\texttt{sympy.physics.mechanics}}
contains elaborate tools for describing mechanical systems, manipulating
reference frames, forces, and torques. These specialized functions are
not necessary for a first-year mechanics course. The basic
\texttt{SymPy} functions like \texttt{solve}, and the vector operations
you learned in the previous sections are powerful enough for basic
Newtonian mechanics.

    \subsubsection{Dynamics}\label{dynamics}

    The net force acting on an object is the sum of all the external forces
acting on it $\vec{F}_{\textrm{net}} = \sum \vec{F}$. Since forces are
vectors, we need to use vector addition to compute the net force.

Compute $\vec{F}_{\textrm{net}}=\vec{F}_1 + \vec{F}_2$, where
$\vec{F}_1=4\hat{\imath}[\mathrm{N}]$ and
$\vec{F}_2 = 5\angle 30^\circ[\mathrm{N}]$:

    \begin{Verbatim}[commandchars=\\\{\}]
{\color{incolor}In [{\color{incolor}154}]:} \PY{n}{F\PYZus{}1} \PY{o}{=}  \PY{n}{Matrix}\PY{p}{(} \PY{p}{[}\PY{l+m+mi}{4}\PY{p}{,}\PY{l+m+mi}{0}\PY{p}{]} \PY{p}{)} 
          \PY{n}{F\PYZus{}2} \PY{o}{=}  \PY{n}{Matrix}\PY{p}{(} \PY{p}{[}\PY{l+m+mi}{5}\PY{o}{*}\PY{n}{cos}\PY{p}{(}\PY{l+m+mi}{30}\PY{o}{*}\PY{n}{pi}\PY{o}{/}\PY{l+m+mi}{180}\PY{p}{)}\PY{p}{,} \PY{l+m+mi}{5}\PY{o}{*}\PY{n}{sin}\PY{p}{(}\PY{l+m+mi}{30}\PY{o}{*}\PY{n}{pi}\PY{o}{/}\PY{l+m+mi}{180}\PY{p}{)} \PY{p}{]} \PY{p}{)}
          \PY{n}{F\PYZus{}net} \PY{o}{=} \PY{n}{F\PYZus{}1} \PY{o}{+} \PY{n}{F\PYZus{}2}
          \PY{n}{F\PYZus{}net}                                    \PY{c}{# in Newtons}
\end{Verbatim}
\texttt{\color{outcolor}Out[{\color{outcolor}154}]:}
    
    
        \begin{equation*}\adjustbox{max width=\hsize}{$
        \left[\begin{matrix}4 + \frac{5 \sqrt{3}}{2}\\\frac{5}{2}\end{matrix}\right]
        $}\end{equation*}

    

    \begin{Verbatim}[commandchars=\\\{\}]
{\color{incolor}In [{\color{incolor}155}]:} \PY{n}{F\PYZus{}net}\PY{o}{.}\PY{n}{evalf}\PY{p}{(}\PY{p}{)}                            \PY{c}{# in Newtons}
\end{Verbatim}
\texttt{\color{outcolor}Out[{\color{outcolor}155}]:}
    
    
        \begin{equation*}\adjustbox{max width=\hsize}{$
        \left[\begin{matrix}8.33012701892219\\2.5\end{matrix}\right]
        $}\end{equation*}

    

    To express the answer in length-and-direction notation, use
\texttt{norm} to find the length of $\vec{F}_{\textrm{net}}$ and
\texttt{atan2} (The function \texttt{atan2(y,x)} computes the correct
direction for all vectors $(x,y)$, unlike \texttt{atan(y/x)} which
requires corrections for angles in the range
$[\frac{\pi}{2}, \frac{3\pi}{2}]$.) to find its direction:

    \begin{Verbatim}[commandchars=\\\{\}]
{\color{incolor}In [{\color{incolor}156}]:} \PY{n}{F\PYZus{}net}\PY{o}{.}\PY{n}{norm}\PY{p}{(}\PY{p}{)}\PY{o}{.}\PY{n}{evalf}\PY{p}{(}\PY{p}{)}                     \PY{c}{# |F\PYZus{}net| in [N]}
\end{Verbatim}
\texttt{\color{outcolor}Out[{\color{outcolor}156}]:}
    
    
        \begin{equation*}\adjustbox{max width=\hsize}{$
        8.69718438067042
        $}\end{equation*}

    

    \begin{Verbatim}[commandchars=\\\{\}]
{\color{incolor}In [{\color{incolor}157}]:} \PY{p}{(}\PY{n}{atan2}\PY{p}{(} \PY{n}{F\PYZus{}net}\PY{p}{[}\PY{l+m+mi}{1}\PY{p}{]}\PY{p}{,}\PY{n}{F\PYZus{}net}\PY{p}{[}\PY{l+m+mi}{0}\PY{p}{]} \PY{p}{)}\PY{o}{*}\PY{l+m+mi}{180}\PY{o}{/}\PY{n}{pi}\PY{p}{)}\PY{o}{.}\PY{n}{n}\PY{p}{(}\PY{p}{)}  \PY{c}{# angle in degrees}
\end{Verbatim}
\texttt{\color{outcolor}Out[{\color{outcolor}157}]:}
    
    
        \begin{equation*}\adjustbox{max width=\hsize}{$
        16.70531380601
        $}\end{equation*}

    

    The net force on the object is
$\vec{F}_{\textrm{net}}= 8.697\angle 16.7^\circ${[}N{]}.

    \subsubsection{Kinematics}\label{kinematics}

    Let $x(t)$ denote the position of an object, $v(t)$ denote its velocity,
and $a(t)$ denote its acceleration. Together $x(t)$, $v(t)$, and $a(t)$
are known as the \emph{equations of motion} of the object.

The equations of motion are related by the derivative operation:

\[
  a(t) \overset{\frac{d}{dt} }{\longleftarrow} v(t) \overset{\frac{d}{dt} }{\longleftarrow} x(t).
\]

Assume we know the initial position $x_i\equiv x(0)$ and the initial
velocity $v_i\equiv v(0)$ of the object and we want to find $x(t)$ for
all later times. We can do this starting from the dynamics of the
problem---the forces acting on the object.

Newton's second law $\vec{F}_{\textrm{net}} = m\vec{a}$ states that a
net force $\vec{F}_{\textrm{net}}$ applied on an object of mass $m$
produces acceleration $\vec{a}$. Thus, we can obtain an objects
acceleration if we know the net force acting on it. Starting from the
knowledge of $a(t)$, we can obtain $v(t)$ by integrating then find
$x(t)$ by integrating $v(t)$:

\[
a(t) \ \ \ \overset{v_i+ \int\!dt }{\longrightarrow} \ \ \ v(t) \ \ \ \overset{x_i+ \int\!dt }{\longrightarrow} \ \ \ x(t).
\]

The reasoning follows from the fundamental theorem of calculus: if
$a(t)$ represents the change in $v(t)$, then the total of $a(t)$
accumulated between $t=t_1$ and $t=t_2$ is equal to the total change in
$v(t)$ between these times: $\Delta v = v(t_2) - v(t_1)$. Similarly, the
integral of $v(t)$ from $t=0$ until $t=\tau$ is equal to
$x(\tau) - x(0)$.

    \subsubsection{Uniform acceleration motion
(UAM)}\label{uniform-acceleration-motion-uam}

    Let's analyze the case where the net force on the object is constant. A
constant force causes a constant acceleration
$a = \frac{F}{m} = \textrm{constant}$. If the acceleration function is
constant over time $a(t)=a$. We find $v(t)$ and $x(t)$ as follows:

    \begin{Verbatim}[commandchars=\\\{\}]
{\color{incolor}In [{\color{incolor}158}]:} \PY{n}{t}\PY{p}{,} \PY{n}{a}\PY{p}{,} \PY{n}{v\PYZus{}i}\PY{p}{,} \PY{n}{x\PYZus{}i} \PY{o}{=} \PY{n}{symbols}\PY{p}{(}\PY{l+s}{'}\PY{l+s}{t a v\PYZus{}i x\PYZus{}i}\PY{l+s}{'}\PY{p}{)}
          \PY{n}{v} \PY{o}{=} \PY{n}{v\PYZus{}i} \PY{o}{+} \PY{n}{integrate}\PY{p}{(}\PY{n}{a}\PY{p}{,} \PY{p}{(}\PY{n}{t}\PY{p}{,} \PY{l+m+mi}{0}\PY{p}{,}\PY{n}{t}\PY{p}{)} \PY{p}{)}
          \PY{n}{v}
\end{Verbatim}
\texttt{\color{outcolor}Out[{\color{outcolor}158}]:}
    
    
        \begin{equation*}\adjustbox{max width=\hsize}{$
        a t + v_{i}
        $}\end{equation*}

    

    \begin{Verbatim}[commandchars=\\\{\}]
{\color{incolor}In [{\color{incolor}159}]:} \PY{n}{x} \PY{o}{=} \PY{n}{x\PYZus{}i} \PY{o}{+} \PY{n}{integrate}\PY{p}{(}\PY{n}{v}\PY{p}{,} \PY{p}{(}\PY{n}{t}\PY{p}{,} \PY{l+m+mi}{0}\PY{p}{,}\PY{n}{t}\PY{p}{)} \PY{p}{)}
          \PY{n}{x}
\end{Verbatim}
\texttt{\color{outcolor}Out[{\color{outcolor}159}]:}
    
    
        \begin{equation*}\adjustbox{max width=\hsize}{$
        \frac{a t^{2}}{2} + t v_{i} + x_{i}
        $}\end{equation*}

    

    You may remember these equations from your high school physics class.
They are the \emph{uniform accelerated motion} (UAM) equations:

\begin{align*}
 a(t) &= a,                                  \\ 
 v(t) &= v_i  + at,                          \\[-2mm] 
 x(t) &= x_i + v_it + \frac{1}{2}at^2.
\end{align*}

In high school, you probably had to memorize these equations. Now you
know how to derive them yourself starting from first principles.

For the sake of completeness, we'll now derive the fourth UAM equation,
which relates the object's final velocity to the initial velocity, the
displacement, and the acceleration, without reference to time:

    \begin{Verbatim}[commandchars=\\\{\}]
{\color{incolor}In [{\color{incolor}160}]:} \PY{p}{(}\PY{n}{v}\PY{o}{*}\PY{n}{v}\PY{p}{)}\PY{o}{.}\PY{n}{expand}\PY{p}{(}\PY{p}{)}
\end{Verbatim}
\texttt{\color{outcolor}Out[{\color{outcolor}160}]:}
    
    
        \begin{equation*}\adjustbox{max width=\hsize}{$
        a^{2} t^{2} + 2 a t v_{i} + v_{i}^{2}
        $}\end{equation*}

    

    \begin{Verbatim}[commandchars=\\\{\}]
{\color{incolor}In [{\color{incolor}161}]:} \PY{p}{(}\PY{p}{(}\PY{n}{v}\PY{o}{*}\PY{n}{v}\PY{p}{)}\PY{o}{.}\PY{n}{expand}\PY{p}{(}\PY{p}{)} \PY{o}{-} \PY{l+m+mi}{2}\PY{o}{*}\PY{n}{a}\PY{o}{*}\PY{n}{x}\PY{p}{)}\PY{o}{.}\PY{n}{simplify}\PY{p}{(}\PY{p}{)}
\end{Verbatim}
\texttt{\color{outcolor}Out[{\color{outcolor}161}]:}
    
    
        \begin{equation*}\adjustbox{max width=\hsize}{$
        - 2 a x_{i} + v_{i}^{2}
        $}\end{equation*}

    

    The above calculation shows $v_f^2 - 2ax_f = -2ax_i + v_i^2$. After
moving the term $2ax_f$ to the other side of the equation, we obtain

\begin{align*}
 (v(t))^2 \ = \ v_f^2 =  v_i^2  + 2a\Delta x \ = \  v_i^2  + 2a(x_f-x_i).
\end{align*}

The fourth equation is important for practical purposes because it
allows us to solve physics problems in a time-less manner.

    \paragraph{Example}\label{example}

    Find the position function of an object at time $t=3[\mathrm{s}]$, if it
starts from $x_i=20[\mathrm{m}]$ with $v_i=10[\mathrm{m/s}]$ and
undergoes a constant acceleration of $a=5[\mathrm{m/s^2}]$. What is the
object's velocity at $t=3[\mathrm{s}]$?

    \begin{Verbatim}[commandchars=\\\{\}]
{\color{incolor}In [{\color{incolor}162}]:} \PY{n}{x\PYZus{}i} \PY{o}{=} \PY{l+m+mi}{20}  \PY{c}{# initial position}
          \PY{n}{v\PYZus{}i} \PY{o}{=} \PY{l+m+mi}{10}  \PY{c}{# initial velocity}
          \PY{n}{a}   \PY{o}{=} \PY{l+m+mi}{5}   \PY{c}{# acceleration (constant during motion)}
          \PY{n}{x} \PY{o}{=} \PY{n}{x\PYZus{}i} \PY{o}{+} \PY{n}{integrate}\PY{p}{(} \PY{n}{v\PYZus{}i}\PY{o}{+}\PY{n}{integrate}\PY{p}{(}\PY{n}{a}\PY{p}{,}\PY{p}{(}\PY{n}{t}\PY{p}{,}\PY{l+m+mi}{0}\PY{p}{,}\PY{n}{t}\PY{p}{)}\PY{p}{)}\PY{p}{,} \PY{p}{(}\PY{n}{t}\PY{p}{,}\PY{l+m+mi}{0}\PY{p}{,}\PY{n}{t}\PY{p}{)} \PY{p}{)}   
          \PY{n}{x}
\end{Verbatim}
\texttt{\color{outcolor}Out[{\color{outcolor}162}]:}
    
    
        \begin{equation*}\adjustbox{max width=\hsize}{$
        \frac{5 t^{2}}{2} + 10 t + 20
        $}\end{equation*}

    

    \begin{Verbatim}[commandchars=\\\{\}]
{\color{incolor}In [{\color{incolor}163}]:} \PY{n}{x}\PY{o}{.}\PY{n}{subs}\PY{p}{(}\PY{p}{\PYZob{}}\PY{n}{t}\PY{p}{:}\PY{l+m+mi}{3}\PY{p}{\PYZcb{}}\PY{p}{)}\PY{o}{.}\PY{n}{n}\PY{p}{(}\PY{p}{)}          \PY{c}{# x(3) in [m]}
\end{Verbatim}
\texttt{\color{outcolor}Out[{\color{outcolor}163}]:}
    
    
        \begin{equation*}\adjustbox{max width=\hsize}{$
        72.5
        $}\end{equation*}

    

    \begin{Verbatim}[commandchars=\\\{\}]
{\color{incolor}In [{\color{incolor}164}]:} \PY{n}{diff}\PY{p}{(}\PY{n}{x}\PY{p}{,}\PY{n}{t}\PY{p}{)}\PY{o}{.}\PY{n}{subs}\PY{p}{(}\PY{p}{\PYZob{}}\PY{n}{t}\PY{p}{:}\PY{l+m+mi}{3}\PY{p}{\PYZcb{}}\PY{p}{)}\PY{o}{.}\PY{n}{n}\PY{p}{(}\PY{p}{)}  \PY{c}{# v(3) in [m/s]}
\end{Verbatim}
\texttt{\color{outcolor}Out[{\color{outcolor}164}]:}
    
    
        \begin{equation*}\adjustbox{max width=\hsize}{$
        25.0
        $}\end{equation*}

    

    If you think about it, physics knowledge combined with computer skills
is like a superpower!

    \subsubsection{General equations of
motion}\label{general-equations-of-motion}

    The procedure
$a(t) \ \overset{v_i+ \int\!dt }{\longrightarrow} \ v(t) \ \overset{x_i+ \int\!dt }{\longrightarrow} \ x(t)$
can be used to obtain the position function $x(t)$ even when the
acceleration is not constant. Suppose the acceleration of an object is
$a(t)=\sqrt{k t}$; what is its $x(t)$?

    \begin{Verbatim}[commandchars=\\\{\}]
{\color{incolor}In [{\color{incolor}165}]:} \PY{n}{t}\PY{p}{,} \PY{n}{v\PYZus{}i}\PY{p}{,} \PY{n}{x\PYZus{}i}\PY{p}{,} \PY{n}{k} \PY{o}{=} \PY{n}{symbols}\PY{p}{(}\PY{l+s}{'}\PY{l+s}{t v\PYZus{}i x\PYZus{}i k}\PY{l+s}{'}\PY{p}{)}
          \PY{n}{a} \PY{o}{=} \PY{n}{sqrt}\PY{p}{(}\PY{n}{k}\PY{o}{*}\PY{n}{t}\PY{p}{)}
          \PY{n}{x} \PY{o}{=} \PY{n}{x\PYZus{}i} \PY{o}{+} \PY{n}{integrate}\PY{p}{(} \PY{n}{v\PYZus{}i}\PY{o}{+}\PY{n}{integrate}\PY{p}{(}\PY{n}{a}\PY{p}{,}\PY{p}{(}\PY{n}{t}\PY{p}{,}\PY{l+m+mi}{0}\PY{p}{,}\PY{n}{t}\PY{p}{)}\PY{p}{)}\PY{p}{,} \PY{p}{(}\PY{n}{t}\PY{p}{,} \PY{l+m+mi}{0}\PY{p}{,}\PY{n}{t}\PY{p}{)} \PY{p}{)}
          \PY{n}{x}
\end{Verbatim}
\texttt{\color{outcolor}Out[{\color{outcolor}165}]:}
    
    
        \begin{equation*}\adjustbox{max width=\hsize}{$
        t v_{i} + x_{i} + \frac{4 \left(k t\right)^{\frac{5}{2}}}{15 k^{2}}
        $}\end{equation*}

    

    \subsubsection{Potential energy}\label{potential-energy}

    Instead of working with the kinematic equations of motion $x(t)$,
$v(t)$, and $a(t)$ which depend on time, we can solve physics problems
using \emph{energy} calculations. A key connection between the world of
forces and the world of energy is the concept of \emph{potential
energy}. If you move an object against a conservative force (think
raising a ball in the air against the force of gravity), you can think
of the work you do agains the force as being stored in the potential
energy of the object.

For each force $\vec{F}(x)$ there is a corresponding potential energy
$U_F(x)$. The change in potential energy associated with the force
$\vec{F}(x)$ and displacement $\vec{d}$ is defined as the negative of
the work done by the force during the displacement:
$U_F(x) = - W = - \int_{\vec{d}} \vec{F}(x)\cdot d\vec{x}$.

The potential energies associated with gravity
$\vec{F}_g = -mg\hat{\jmath}$ and the force of a spring
$\vec{F}_s = -k\vec{x}$ are calculated as follows:

    \begin{Verbatim}[commandchars=\\\{\}]
{\color{incolor}In [{\color{incolor}166}]:} \PY{n}{x}\PY{p}{,} \PY{n}{y} \PY{o}{=} \PY{n}{symbols}\PY{p}{(}\PY{l+s}{'}\PY{l+s}{x y}\PY{l+s}{'}\PY{p}{)}
          \PY{n}{m}\PY{p}{,} \PY{n}{g}\PY{p}{,} \PY{n}{k}\PY{p}{,} \PY{n}{h} \PY{o}{=} \PY{n}{symbols}\PY{p}{(}\PY{l+s}{'}\PY{l+s}{m g k h}\PY{l+s}{'}\PY{p}{)}
          \PY{n}{F\PYZus{}g} \PY{o}{=} \PY{o}{-}\PY{n}{m}\PY{o}{*}\PY{n}{g}  \PY{c}{# Force of gravity on mass m }
          \PY{n}{U\PYZus{}g} \PY{o}{=} \PY{o}{-} \PY{n}{integrate}\PY{p}{(} \PY{n}{F\PYZus{}g}\PY{p}{,} \PY{p}{(}\PY{n}{y}\PY{p}{,}\PY{l+m+mi}{0}\PY{p}{,}\PY{n}{h}\PY{p}{)} \PY{p}{)}
          \PY{n}{U\PYZus{}g}         \PY{c}{# Grav. potential energy}
\end{Verbatim}
\texttt{\color{outcolor}Out[{\color{outcolor}166}]:}
    
    
        \begin{equation*}\adjustbox{max width=\hsize}{$
        g h m
        $}\end{equation*}

    

    \begin{Verbatim}[commandchars=\\\{\}]
{\color{incolor}In [{\color{incolor}167}]:} \PY{n}{F\PYZus{}s} \PY{o}{=} \PY{o}{-}\PY{n}{k}\PY{o}{*}\PY{n}{x}  \PY{c}{# Spring force for displacement x }
          \PY{n}{U\PYZus{}s} \PY{o}{=} \PY{o}{-} \PY{n}{integrate}\PY{p}{(} \PY{n}{F\PYZus{}s}\PY{p}{,} \PY{p}{(}\PY{n}{x}\PY{p}{,}\PY{l+m+mi}{0}\PY{p}{,}\PY{n}{x}\PY{p}{)} \PY{p}{)}
          \PY{n}{U\PYZus{}s}         \PY{c}{# Spring potential energy}
\end{Verbatim}
\texttt{\color{outcolor}Out[{\color{outcolor}167}]:}
    
    
        \begin{equation*}\adjustbox{max width=\hsize}{$
        \frac{k x^{2}}{2}
        $}\end{equation*}

    

    Note the negative sign in the formula defining the potential energy.
This negative is canceled by the negative sign of the dot product
$\vec{F}\cdot d\vec{x}$: when the force acts in the direction opposite
to the displacement, the work done by the force is negative.

    \subsubsection{Simple harmonic motion}\label{simple-harmonic-motion}

    The force exerted by a spring is given by the formula $F=-kx$. If the
only force acting on a mass $m$ is the force of a spring, we can use
Newton's second law to obtain the following equation:

\[
  F=ma  
  \quad \Rightarrow \quad
  -kx = ma   
  \quad \Rightarrow \quad
  -kx(t) = m\frac{d^2}{dt^2}\Big[x(t)\Big].
\]

The motion of a mass-spring system is described by the
\emph{differential equation} $\frac{d^2}{dt^2}x(t) + \omega^2 x(t)=0$,
where the constant $\omega = \sqrt{\frac{k}{m}}$ is called the angular
frequency. We can find the position function $x(t)$ using the
\texttt{dsolve} method:

    \begin{Verbatim}[commandchars=\\\{\}]
{\color{incolor}In [{\color{incolor}168}]:} \PY{n}{t} \PY{o}{=} \PY{n}{Symbol}\PY{p}{(}\PY{l+s}{'}\PY{l+s}{t}\PY{l+s}{'}\PY{p}{)}                 \PY{c}{# time t}
          \PY{n}{x} \PY{o}{=} \PY{n}{Function}\PY{p}{(}\PY{l+s}{'}\PY{l+s}{x}\PY{l+s}{'}\PY{p}{)}               \PY{c}{# position function x(t)}
          \PY{n}{w} \PY{o}{=} \PY{n}{Symbol}\PY{p}{(}\PY{l+s}{'}\PY{l+s}{w}\PY{l+s}{'}\PY{p}{,} \PY{n}{positive}\PY{o}{=}\PY{k}{True}\PY{p}{)}  \PY{c}{# angular frequency w}
          \PY{n}{sol} \PY{o}{=} \PY{n}{dsolve}\PY{p}{(} \PY{n}{diff}\PY{p}{(}\PY{n}{x}\PY{p}{(}\PY{n}{t}\PY{p}{)}\PY{p}{,}\PY{n}{t}\PY{p}{,}\PY{n}{t}\PY{p}{)} \PY{o}{+} \PY{n}{w}\PY{o}{*}\PY{o}{*}\PY{l+m+mi}{2}\PY{o}{*}\PY{n}{x}\PY{p}{(}\PY{n}{t}\PY{p}{)}\PY{p}{,} \PY{n}{x}\PY{p}{(}\PY{n}{t}\PY{p}{)} \PY{p}{)}
          \PY{n}{sol}
\end{Verbatim}
\texttt{\color{outcolor}Out[{\color{outcolor}168}]:}
    
    
        \begin{equation*}\adjustbox{max width=\hsize}{$
        x{\left (t \right )} = C_{1} \sin{\left (t w \right )} + C_{2} \cos{\left (t w \right )}
        $}\end{equation*}

    

    \begin{Verbatim}[commandchars=\\\{\}]
{\color{incolor}In [{\color{incolor}169}]:} \PY{n}{x} \PY{o}{=} \PY{n}{sol}\PY{o}{.}\PY{n}{rhs}           
          \PY{n}{x}
\end{Verbatim}
\texttt{\color{outcolor}Out[{\color{outcolor}169}]:}
    
    
        \begin{equation*}\adjustbox{max width=\hsize}{$
        C_{1} \sin{\left (t w \right )} + C_{2} \cos{\left (t w \right )}
        $}\end{equation*}

    

    Note the solution $x(t)=C_1\sin(\omega t)+C_2 \cos(\omega t)$ is
equivalent to $x(t) = A\cos(\omega t + \phi)$, which is more commonly
used to describe simple harmonic motion. We can use the \texttt{expand}
function with the argument \texttt{trig=True} to convince ourselves of
this equivalence:

    \begin{Verbatim}[commandchars=\\\{\}]
{\color{incolor}In [{\color{incolor}170}]:} \PY{n}{A}\PY{p}{,} \PY{n}{phi} \PY{o}{=} \PY{n}{symbols}\PY{p}{(}\PY{l+s}{"}\PY{l+s}{A phi}\PY{l+s}{"}\PY{p}{)}
          \PY{p}{(}\PY{n}{A}\PY{o}{*}\PY{n}{cos}\PY{p}{(}\PY{n}{w}\PY{o}{*}\PY{n}{t} \PY{o}{-} \PY{n}{phi}\PY{p}{)}\PY{p}{)}\PY{o}{.}\PY{n}{expand}\PY{p}{(}\PY{n}{trig}\PY{o}{=}\PY{k}{True}\PY{p}{)}
\end{Verbatim}
\texttt{\color{outcolor}Out[{\color{outcolor}170}]:}
    
    
        \begin{equation*}\adjustbox{max width=\hsize}{$
        A \sin{\left (\phi \right )} \sin{\left (t w \right )} + A \cos{\left (\phi \right )} \cos{\left (t w \right )}
        $}\end{equation*}

    

    If we define $C_1=A\sin(\phi)$ and $C_2=A\cos(\phi)$, we obtain the form
$x(t)=C_1\sin(\omega t)+C_2 \cos(\omega t)$ that \texttt{SymPy} found.

    \subsubsection{Conservation of energy}\label{conservation-of-energy}

    We can verify that the total energy of the mass-spring system is
conserved by showing $E_T(t) = U_s(t) + K(t) = \textrm{constant}$:

    \begin{Verbatim}[commandchars=\\\{\}]
{\color{incolor}In [{\color{incolor}171}]:} \PY{n}{x} \PY{o}{=} \PY{n}{sol}\PY{o}{.}\PY{n}{rhs}\PY{o}{.}\PY{n}{subs}\PY{p}{(}\PY{p}{\PYZob{}}\PY{l+s}{"}\PY{l+s}{C1}\PY{l+s}{"}\PY{p}{:}\PY{l+m+mi}{0}\PY{p}{,}\PY{l+s}{"}\PY{l+s}{C2}\PY{l+s}{"}\PY{p}{:}\PY{n}{A}\PY{p}{\PYZcb{}}\PY{p}{)} 
          \PY{n}{x}
\end{Verbatim}
\texttt{\color{outcolor}Out[{\color{outcolor}171}]:}
    
    
        \begin{equation*}\adjustbox{max width=\hsize}{$
        A \cos{\left (t w \right )}
        $}\end{equation*}

    

    \begin{Verbatim}[commandchars=\\\{\}]
{\color{incolor}In [{\color{incolor}172}]:} \PY{n}{v} \PY{o}{=} \PY{n}{diff}\PY{p}{(}\PY{n}{x}\PY{p}{,} \PY{n}{t}\PY{p}{)}
          \PY{n}{v}
\end{Verbatim}
\texttt{\color{outcolor}Out[{\color{outcolor}172}]:}
    
    
        \begin{equation*}\adjustbox{max width=\hsize}{$
        - A w \sin{\left (t w \right )}
        $}\end{equation*}

    

    \begin{Verbatim}[commandchars=\\\{\}]
{\color{incolor}In [{\color{incolor}173}]:} \PY{n}{E\PYZus{}T} \PY{o}{=} \PY{p}{(}\PY{l+m+mf}{0.5}\PY{o}{*}\PY{n}{k}\PY{o}{*}\PY{n}{x}\PY{o}{*}\PY{o}{*}\PY{l+m+mi}{2} \PY{o}{+} \PY{l+m+mf}{0.5}\PY{o}{*}\PY{n}{m}\PY{o}{*}\PY{n}{v}\PY{o}{*}\PY{o}{*}\PY{l+m+mi}{2}\PY{p}{)}\PY{o}{.}\PY{n}{simplify}\PY{p}{(}\PY{p}{)}
          \PY{n}{E\PYZus{}T}
\end{Verbatim}
\texttt{\color{outcolor}Out[{\color{outcolor}173}]:}
    
    
        \begin{equation*}\adjustbox{max width=\hsize}{$
        0.5 A^{2} \left(k \cos^{2}{\left (t w \right )} + m w^{2} \sin^{2}{\left (t w \right )}\right)
        $}\end{equation*}

    

    \begin{Verbatim}[commandchars=\\\{\}]
{\color{incolor}In [{\color{incolor}174}]:} \PY{n}{E\PYZus{}T}\PY{o}{.}\PY{n}{subs}\PY{p}{(}\PY{p}{\PYZob{}}\PY{n}{k}\PY{p}{:}\PY{n}{m}\PY{o}{*}\PY{n}{w}\PY{o}{*}\PY{o}{*}\PY{l+m+mi}{2}\PY{p}{\PYZcb{}}\PY{p}{)}\PY{o}{.}\PY{n}{simplify}\PY{p}{(}\PY{p}{)}     \PY{c}{# = K\PYZus{}max}
\end{Verbatim}
\texttt{\color{outcolor}Out[{\color{outcolor}174}]:}
    
    
        \begin{equation*}\adjustbox{max width=\hsize}{$
        0.5 A^{2} m w^{2}
        $}\end{equation*}

    

    \begin{Verbatim}[commandchars=\\\{\}]
{\color{incolor}In [{\color{incolor}175}]:} \PY{n}{E\PYZus{}T}\PY{o}{.}\PY{n}{subs}\PY{p}{(}\PY{p}{\PYZob{}}\PY{n}{w}\PY{p}{:}\PY{n}{sqrt}\PY{p}{(}\PY{n}{k}\PY{o}{/}\PY{n}{m}\PY{p}{)}\PY{p}{\PYZcb{}}\PY{p}{)}\PY{o}{.}\PY{n}{simplify}\PY{p}{(}\PY{p}{)}  \PY{c}{# = U\PYZus{}max}
\end{Verbatim}
\texttt{\color{outcolor}Out[{\color{outcolor}175}]:}
    
    
        \begin{equation*}\adjustbox{max width=\hsize}{$
        0.5 A^{2} k
        $}\end{equation*}

    

    \subsection{Linear algebra}\label{linear-algebra}

    A matrix $A \in \mathbb{R}^{m\times n}$ is a rectangular array of real
numbers with $m$ rows and $n$ columns. To specify a matrix $A$, we
specify the values for its $mn$ components
$a_{11}, a_{12}, \ldots, a_{mn}$ as a list of lists:

    \begin{Verbatim}[commandchars=\\\{\}]
{\color{incolor}In [{\color{incolor}176}]:} \PY{n}{A} \PY{o}{=} \PY{n}{Matrix}\PY{p}{(} \PY{p}{[}\PY{p}{[} \PY{l+m+mi}{2}\PY{p}{,}\PY{o}{-}\PY{l+m+mi}{3}\PY{p}{,}\PY{o}{-}\PY{l+m+mi}{8}\PY{p}{,} \PY{l+m+mi}{7}\PY{p}{]}\PY{p}{,}
                       \PY{p}{[}\PY{o}{-}\PY{l+m+mi}{2}\PY{p}{,}\PY{o}{-}\PY{l+m+mi}{1}\PY{p}{,} \PY{l+m+mi}{2}\PY{p}{,}\PY{o}{-}\PY{l+m+mi}{7}\PY{p}{]}\PY{p}{,}
                       \PY{p}{[} \PY{l+m+mi}{1}\PY{p}{,} \PY{l+m+mi}{0}\PY{p}{,}\PY{o}{-}\PY{l+m+mi}{3}\PY{p}{,} \PY{l+m+mi}{6}\PY{p}{]}\PY{p}{]} \PY{p}{)}
\end{Verbatim}

    Use the square brackets to access the matrix elements or to obtain a
submatrix:

    \begin{Verbatim}[commandchars=\\\{\}]
{\color{incolor}In [{\color{incolor}177}]:} \PY{n}{A}\PY{p}{[}\PY{l+m+mi}{0}\PY{p}{,}\PY{l+m+mi}{1}\PY{p}{]}         \PY{c}{# row 0, col 1 of A}
\end{Verbatim}
\texttt{\color{outcolor}Out[{\color{outcolor}177}]:}
    
    
        \begin{equation*}\adjustbox{max width=\hsize}{$
        -3
        $}\end{equation*}

    

    \begin{Verbatim}[commandchars=\\\{\}]
{\color{incolor}In [{\color{incolor}178}]:} \PY{n}{A}\PY{p}{[}\PY{l+m+mi}{0}\PY{p}{:}\PY{l+m+mi}{2}\PY{p}{,}\PY{l+m+mi}{0}\PY{p}{:}\PY{l+m+mi}{3}\PY{p}{]}     \PY{c}{# top-left 2x3 submatrix of A}
\end{Verbatim}
\texttt{\color{outcolor}Out[{\color{outcolor}178}]:}
    
    
        \begin{equation*}\adjustbox{max width=\hsize}{$
        \left[\begin{matrix}2 & -3 & -8\\-2 & -1 & 2\end{matrix}\right]
        $}\end{equation*}

    

    Some commonly used matrices can be created with shortcut methods:

    \begin{Verbatim}[commandchars=\\\{\}]
{\color{incolor}In [{\color{incolor}179}]:} \PY{n}{eye}\PY{p}{(}\PY{l+m+mi}{2}\PY{p}{)}         \PY{c}{# 2x2 identity matrix}
\end{Verbatim}
\texttt{\color{outcolor}Out[{\color{outcolor}179}]:}
    
    
        \begin{equation*}\adjustbox{max width=\hsize}{$
        \left[\begin{matrix}1 & 0\\0 & 1\end{matrix}\right]
        $}\end{equation*}

    

    \begin{Verbatim}[commandchars=\\\{\}]
{\color{incolor}In [{\color{incolor}180}]:} \PY{n}{zeros}\PY{p}{(}\PY{l+m+mi}{2}\PY{p}{,} \PY{l+m+mi}{3}\PY{p}{)}
\end{Verbatim}
\texttt{\color{outcolor}Out[{\color{outcolor}180}]:}
    
    
        \begin{equation*}\adjustbox{max width=\hsize}{$
        \left[\begin{matrix}0 & 0 & 0\\0 & 0 & 0\end{matrix}\right]
        $}\end{equation*}

    

    Standard algebraic operations like addition \texttt{+}, subtraction
\texttt{-}, multiplication \texttt{*}, and exponentiation \texttt{**}
work as expected for \texttt{Matrix} objects. The \texttt{transpose}
operation flips the matrix through its diagonal:

    \begin{Verbatim}[commandchars=\\\{\}]
{\color{incolor}In [{\color{incolor}181}]:} \PY{n}{A}\PY{o}{.}\PY{n}{transpose}\PY{p}{(}\PY{p}{)}  \PY{c}{# the same as A.T}
\end{Verbatim}
\texttt{\color{outcolor}Out[{\color{outcolor}181}]:}
    
    
        \begin{equation*}\adjustbox{max width=\hsize}{$
        \left[\begin{matrix}2 & -2 & 1\\-3 & -1 & 0\\-8 & 2 & -3\\7 & -7 & 6\end{matrix}\right]
        $}\end{equation*}

    

    Recall that the transpose is also used to convert row vectors into
column vectors and vice versa.

    \subsubsection{Row operations}\label{row-operations}

    \begin{Verbatim}[commandchars=\\\{\}]
{\color{incolor}In [{\color{incolor}182}]:} \PY{n}{M} \PY{o}{=} \PY{n}{eye}\PY{p}{(}\PY{l+m+mi}{3}\PY{p}{)}
          \PY{n}{M}\PY{o}{.}\PY{n}{row\PYZus{}op}\PY{p}{(}\PY{l+m+mi}{1}\PY{p}{,} \PY{k}{lambda} \PY{n}{v}\PY{p}{,}\PY{n}{j}\PY{p}{:} \PY{n}{v}\PY{o}{+}\PY{l+m+mi}{3}\PY{o}{*}\PY{n}{M}\PY{p}{[}\PY{l+m+mi}{0}\PY{p}{,}\PY{n}{j}\PY{p}{]} \PY{p}{)}
          \PY{n}{M}
\end{Verbatim}
\texttt{\color{outcolor}Out[{\color{outcolor}182}]:}
    
    
        \begin{equation*}\adjustbox{max width=\hsize}{$
        \left[\begin{matrix}1 & 0 & 0\\3 & 1 & 0\\0 & 0 & 1\end{matrix}\right]
        $}\end{equation*}

    

    The method \texttt{row\_op} takes two arguments as inputs: the first
argument specifies the 0-based index of the row you want to act on,
while the second argument is a function of the form \texttt{f(val,j)}
that describes how you want the value \texttt{val=M{[}i,j{]}} to be
transformed. The above call to \texttt{row\_op} implements the row
operation $R_2 \gets R_2 + 3R_1$.

    \subsubsection{Reduced row echelon form}\label{reduced-row-echelon-form}

    The Gauss---Jordan elimination procedure is a sequence of row operations
you can perform on any matrix to bring it to its \emph{reduced row
echelon form} (RREF). In \texttt{SymPy}, matrices have a \texttt{rref}
method that computes their RREF:

    \begin{Verbatim}[commandchars=\\\{\}]
{\color{incolor}In [{\color{incolor}183}]:} \PY{n}{A} \PY{o}{=} \PY{n}{Matrix}\PY{p}{(} \PY{p}{[}\PY{p}{[}\PY{l+m+mi}{2}\PY{p}{,}\PY{o}{-}\PY{l+m+mi}{3}\PY{p}{,}\PY{o}{-}\PY{l+m+mi}{8}\PY{p}{,} \PY{l+m+mi}{7}\PY{p}{]}\PY{p}{,}
                       \PY{p}{[}\PY{o}{-}\PY{l+m+mi}{2}\PY{p}{,}\PY{o}{-}\PY{l+m+mi}{1}\PY{p}{,}\PY{l+m+mi}{2}\PY{p}{,}\PY{o}{-}\PY{l+m+mi}{7}\PY{p}{]}\PY{p}{,}
                       \PY{p}{[}\PY{l+m+mi}{1}\PY{p}{,} \PY{l+m+mi}{0}\PY{p}{,}\PY{o}{-}\PY{l+m+mi}{3}\PY{p}{,} \PY{l+m+mi}{6}\PY{p}{]}\PY{p}{]}\PY{p}{)}
          \PY{n}{A}\PY{o}{.}\PY{n}{rref}\PY{p}{(}\PY{p}{)}  \PY{c}{# RREF of A, location of pivots}
\end{Verbatim}
\texttt{\color{outcolor}Out[{\color{outcolor}183}]:}
    
    
        \begin{equation*}\adjustbox{max width=\hsize}{$
        \left ( \left[\begin{matrix}1 & 0 & 0 & 0\\0 & 1 & 0 & 3\\0 & 0 & 1 & -2\end{matrix}\right], \quad \left [ 0, \quad 1, \quad 2\right ]\right )
        $}\end{equation*}

    

    Note the \texttt{rref} method returns a tuple of values: the first value
is the RREF of $A$, while the second tells you the indices of the
leading ones (also known as pivots) in the RREF of $A$. To get just the
RREF of $A$, select the $0^\mathrm{th}$ entry form the tuple:
\texttt{A.rref(){[}0{]}}.

    \subsubsection{Matrix fundamental
spaces}\label{matrix-fundamental-spaces}

    Consider the matrix $A \in \mathbb{R}^{m\times n}$. The fundamental
spaces of a matrix are its column space $\mathcal{C}(A)$, its null space
$\mathcal{N}(A)$, and its row space $\mathcal{R}(A)$. These vector
spaces are important when you consider the matrix product
$A\vec{x}=\vec{y}$ as ``applying'' the linear transformation
$T_A:\mathbb{R}^n \to \mathbb{R}^m$ to an input vector
$\vec{x} \in \mathbb{R}^n$ to produce the output vector
$\vec{y} \in \mathbb{R}^m$.

\textbf{Linear transformations} $T_A:\mathbb{R}^n \to \mathbb{R}^m$
(vector functions) \textbf{are equivalent to $m\times n$ matrices}. This
is one of the fundamental ideas in linear algebra. You can think of
$T_A$ as the abstract description of the transformation and
$A \in \mathbb{R}^{m\times n}$ as a concrete implementation of $T_A$. By
this equivalence, the fundamental spaces of a matrix $A$ tell us facts
about the domain and image of the linear transformation $T_A$. The
columns space $\mathcal{C}(A)$ is the same as the image space space
$\textrm{Im}(T_A)$ (the set of all possible outputs). The null space
$\mathcal{N}(A)$ is the same as the kernel $\textrm{Ker}(T_A)$ (the set
of inputs that $T_A$ maps to the zero vector). The row space
$\mathcal{R}(A)$ is the orthogonal complement of the null space. Input
vectors in the row space of $A$ are in one-to-one correspondence with
the output vectors in the column space of $A$.

Okay, enough theory! Let's see how to compute the fundamental spaces of
the matrix $A$ defined above. The non-zero rows in the reduced row
echelon form of $A$ are a basis for its row space:

    \begin{Verbatim}[commandchars=\\\{\}]
{\color{incolor}In [{\color{incolor}184}]:} \PY{p}{[} \PY{n}{A}\PY{o}{.}\PY{n}{rref}\PY{p}{(}\PY{p}{)}\PY{p}{[}\PY{l+m+mi}{0}\PY{p}{]}\PY{p}{[}\PY{n}{r}\PY{p}{,}\PY{p}{:}\PY{p}{]} \PY{k}{for} \PY{n}{r} \PY{o+ow}{in} \PY{n}{A}\PY{o}{.}\PY{n}{rref}\PY{p}{(}\PY{p}{)}\PY{p}{[}\PY{l+m+mi}{1}\PY{p}{]} \PY{p}{]}  \PY{c}{# R(A)}
\end{Verbatim}
\texttt{\color{outcolor}Out[{\color{outcolor}184}]:}
    
    
        \begin{equation*}\adjustbox{max width=\hsize}{$
        \left [ \left[\begin{matrix}1 & 0 & 0 & 0\end{matrix}\right], \quad \left[\begin{matrix}0 & 1 & 0 & 3\end{matrix}\right], \quad \left[\begin{matrix}0 & 0 & 1 & -2\end{matrix}\right]\right ]
        $}\end{equation*}

    

    The column space of $A$ is the span of the columns of $A$ that contain
the pivots in the reduced row echelon form of $A$:

    \begin{Verbatim}[commandchars=\\\{\}]
{\color{incolor}In [{\color{incolor}185}]:} \PY{p}{[} \PY{n}{A}\PY{p}{[}\PY{p}{:}\PY{p}{,}\PY{n}{c}\PY{p}{]} \PY{k}{for} \PY{n}{c} \PY{o+ow}{in}  \PY{n}{A}\PY{o}{.}\PY{n}{rref}\PY{p}{(}\PY{p}{)}\PY{p}{[}\PY{l+m+mi}{1}\PY{p}{]} \PY{p}{]}           \PY{c}{# C(A)}
\end{Verbatim}
\texttt{\color{outcolor}Out[{\color{outcolor}185}]:}
    
    
        \begin{equation*}\adjustbox{max width=\hsize}{$
        \left [ \left[\begin{matrix}2\\-2\\1\end{matrix}\right], \quad \left[\begin{matrix}-3\\-1\\0\end{matrix}\right], \quad \left[\begin{matrix}-8\\2\\-3\end{matrix}\right]\right ]
        $}\end{equation*}

    

    Note we took columns from the original matrix $A$ and not its RREF.

To find the null space of $A$, call its \texttt{nullspace} method:

    \begin{Verbatim}[commandchars=\\\{\}]
{\color{incolor}In [{\color{incolor}186}]:} \PY{n}{A}\PY{o}{.}\PY{n}{nullspace}\PY{p}{(}\PY{p}{)}                              \PY{c}{# N(A)}
\end{Verbatim}
\texttt{\color{outcolor}Out[{\color{outcolor}186}]:}
    
    
        \begin{equation*}\adjustbox{max width=\hsize}{$
        \left [ \left[\begin{matrix}0\\-3\\2\\1\end{matrix}\right]\right ]
        $}\end{equation*}

    

    \subsubsection{Determinants}\label{determinants}

    The determinant of a matrix, denoted $\det(A)$ or $|A|$, is a particular
way to multiply the entries of the matrix to produce a single number.

    \begin{Verbatim}[commandchars=\\\{\}]
{\color{incolor}In [{\color{incolor}187}]:} \PY{n}{M} \PY{o}{=} \PY{n}{Matrix}\PY{p}{(} \PY{p}{[}\PY{p}{[}\PY{l+m+mi}{1}\PY{p}{,} \PY{l+m+mi}{2}\PY{p}{,} \PY{l+m+mi}{3}\PY{p}{]}\PY{p}{,} 
                       \PY{p}{[}\PY{l+m+mi}{2}\PY{p}{,}\PY{o}{-}\PY{l+m+mi}{2}\PY{p}{,} \PY{l+m+mi}{4}\PY{p}{]}\PY{p}{,}
                       \PY{p}{[}\PY{l+m+mi}{2}\PY{p}{,} \PY{l+m+mi}{2}\PY{p}{,} \PY{l+m+mi}{5}\PY{p}{]}\PY{p}{]} \PY{p}{)}
          \PY{n}{M}\PY{o}{.}\PY{n}{det}\PY{p}{(}\PY{p}{)}
\end{Verbatim}
\texttt{\color{outcolor}Out[{\color{outcolor}187}]:}
    
    
        \begin{equation*}\adjustbox{max width=\hsize}{$
        2
        $}\end{equation*}

    

    Determinants are used for all kinds of tasks: to compute areas and
volumes, to solve systems of equations, and to check whether a matrix is
invertible or not.

    \subsubsection{Matrix inverse}\label{matrix-inverse}

    For every invertible matrix $A$, there exists an inverse matrix $A^{-1}$
which \emph{undoes} the effect of $A$. The cumulative effect of the
product of $A$ and $A^{-1}$ (in any order) is the identity matrix:
$AA^{-1}= A^{-1}A=\mathbb{1}$.

    \begin{Verbatim}[commandchars=\\\{\}]
{\color{incolor}In [{\color{incolor}188}]:} \PY{n}{A} \PY{o}{=} \PY{n}{Matrix}\PY{p}{(} \PY{p}{[}\PY{p}{[}\PY{l+m+mi}{1}\PY{p}{,}\PY{l+m+mi}{2}\PY{p}{]}\PY{p}{,} 
                       \PY{p}{[}\PY{l+m+mi}{3}\PY{p}{,}\PY{l+m+mi}{9}\PY{p}{]}\PY{p}{]} \PY{p}{)} 
          \PY{n}{A}\PY{o}{.}\PY{n}{inv}\PY{p}{(}\PY{p}{)}
\end{Verbatim}
\texttt{\color{outcolor}Out[{\color{outcolor}188}]:}
    
    
        \begin{equation*}\adjustbox{max width=\hsize}{$
        \left[\begin{matrix}3 & - \frac{2}{3}\\-1 & \frac{1}{3}\end{matrix}\right]
        $}\end{equation*}

    

    \begin{Verbatim}[commandchars=\\\{\}]
{\color{incolor}In [{\color{incolor}189}]:} \PY{n}{A}\PY{o}{.}\PY{n}{inv}\PY{p}{(}\PY{p}{)}\PY{o}{*}\PY{n}{A}
\end{Verbatim}
\texttt{\color{outcolor}Out[{\color{outcolor}189}]:}
    
    
        \begin{equation*}\adjustbox{max width=\hsize}{$
        \left[\begin{matrix}1 & 0\\0 & 1\end{matrix}\right]
        $}\end{equation*}

    

    \begin{Verbatim}[commandchars=\\\{\}]
{\color{incolor}In [{\color{incolor}190}]:} \PY{n}{A}\PY{o}{*}\PY{n}{A}\PY{o}{.}\PY{n}{inv}\PY{p}{(}\PY{p}{)}
\end{Verbatim}
\texttt{\color{outcolor}Out[{\color{outcolor}190}]:}
    
    
        \begin{equation*}\adjustbox{max width=\hsize}{$
        \left[\begin{matrix}1 & 0\\0 & 1\end{matrix}\right]
        $}\end{equation*}

    

    The matrix inverse $A^{-1}$ plays the role of division by $A$.

    \subsubsection{Eigenvectors and
eigenvalues}\label{eigenvectors-and-eigenvalues}

    When a matrix is multiplied by one of its eigenvectors the output is the
same eigenvector multiplied by a constant
$A\vec{e}_\lambda =\lambda\vec{e}_\lambda$. The constant $\lambda$ (the
Greek letter \emph{lambda}) is called an \emph{eigenvalue} of $A$.

To find the eigenvalues of a matrix, start from the definition
$A\vec{e}_\lambda =\lambda\vec{e}_\lambda$, insert the identity
$\mathbb{1}$, and rewrite it as a null-space problem:

\[
A\vec{e}_\lambda =\lambda\mathbb{1}\vec{e}_\lambda
\qquad
\Rightarrow
\qquad
\left(A - \lambda\mathbb{1}\right)\vec{e}_\lambda = \vec{0}.
\]

This equation will have a solution whenever
$|A - \lambda\mathbb{1}|=0$.(The invertible matrix theorem states that a
matrix has a non-empty null space if and only if its determinant is
zero.) The eigenvalues of $A \in \mathbb{R}^{n \times n}$, denoted
$\{ \lambda_1, \lambda_2, \ldots, \lambda_n \}$,\\are the roots of the
\emph{characteristic polynomial} $p(\lambda)=|A - \lambda \mathbb{1}|$.

    \begin{Verbatim}[commandchars=\\\{\}]
{\color{incolor}In [{\color{incolor}191}]:} \PY{n}{A} \PY{o}{=} \PY{n}{Matrix}\PY{p}{(} \PY{p}{[}\PY{p}{[} \PY{l+m+mi}{9}\PY{p}{,} \PY{o}{-}\PY{l+m+mi}{2}\PY{p}{]}\PY{p}{,}
                       \PY{p}{[}\PY{o}{-}\PY{l+m+mi}{2}\PY{p}{,}  \PY{l+m+mi}{6}\PY{p}{]}\PY{p}{]} \PY{p}{)}
          \PY{n}{A}\PY{o}{.}\PY{n}{eigenvals}\PY{p}{(}\PY{p}{)}  \PY{c}{# same as solve(det(A-eye(2)*x), x)}
                         \PY{c}{# return eigenvalues with their multiplicity}
\end{Verbatim}
\texttt{\color{outcolor}Out[{\color{outcolor}191}]:}
    
    
        \begin{equation*}\adjustbox{max width=\hsize}{$
        \left \{ 5 : 1, \quad 10 : 1\right \}
        $}\end{equation*}

    

    \begin{Verbatim}[commandchars=\\\{\}]
{\color{incolor}In [{\color{incolor}192}]:} \PY{n}{A}\PY{o}{.}\PY{n}{eigenvects}\PY{p}{(}\PY{p}{)}
\end{Verbatim}
\texttt{\color{outcolor}Out[{\color{outcolor}192}]:}
    
    
        \begin{equation*}\adjustbox{max width=\hsize}{$
        \left [ \left ( 5, \quad 1, \quad \left [ \left[\begin{matrix}\frac{1}{2}\\1\end{matrix}\right]\right ]\right ), \quad \left ( 10, \quad 1, \quad \left [ \left[\begin{matrix}-2\\1\end{matrix}\right]\right ]\right )\right ]
        $}\end{equation*}

    

    Certain matrices can be written entirely in terms of their eigenvectors
and their eigenvalues. Consider the matrix $\Lambda$ (capital Greek
\emph{L}) that has the eigenvalues of the matrix $A$ on the diagonal,
and the matrix $Q$ constructed from the eigenvectors of $A$ as columns:

\[
\Lambda = 
\begin{bmatrix}
\lambda_1   &  \cdots  &  0 \\
\vdots  &  \ddots  &  0  \\
0   &   0      &  \lambda_n
\end{bmatrix}\!,
\ \ 
Q \: = 
\begin{bmatrix}
|  &  & | \\
\vec{e}_{\lambda_1}  & \!  \cdots \! &  \large\vec{e}_{\lambda_n} \\
|  &  & | 
\end{bmatrix}\!,
\ \ 
\textrm{then}
\ \ 
A = Q \Lambda Q^{-1}.
\]

Matrices that can be written this way are called \emph{diagonalizable}.
To \emph{diagonalize} a matrix $A$ is to find its $Q$ and $\Lambda$
matrices:

    \begin{Verbatim}[commandchars=\\\{\}]
{\color{incolor}In [{\color{incolor}193}]:} \PY{n}{Q}\PY{p}{,} \PY{n}{L} \PY{o}{=} \PY{n}{A}\PY{o}{.}\PY{n}{diagonalize}\PY{p}{(}\PY{p}{)}
          \PY{n}{Q}            \PY{c}{# the matrix of eigenvectors as columns }
\end{Verbatim}
\texttt{\color{outcolor}Out[{\color{outcolor}193}]:}
    
    
        \begin{equation*}\adjustbox{max width=\hsize}{$
        \left[\begin{matrix}1 & -2\\2 & 1\end{matrix}\right]
        $}\end{equation*}

    

    \begin{Verbatim}[commandchars=\\\{\}]
{\color{incolor}In [{\color{incolor}194}]:} \PY{n}{Q}\PY{o}{.}\PY{n}{inv}\PY{p}{(}\PY{p}{)}
\end{Verbatim}
\texttt{\color{outcolor}Out[{\color{outcolor}194}]:}
    
    
        \begin{equation*}\adjustbox{max width=\hsize}{$
        \left[\begin{matrix}\frac{1}{5} & \frac{2}{5}\\- \frac{2}{5} & \frac{1}{5}\end{matrix}\right]
        $}\end{equation*}

    

    \begin{Verbatim}[commandchars=\\\{\}]
{\color{incolor}In [{\color{incolor}195}]:} \PY{n}{L}            \PY{c}{# the matrix of eigenvalues}
\end{Verbatim}
\texttt{\color{outcolor}Out[{\color{outcolor}195}]:}
    
    
        \begin{equation*}\adjustbox{max width=\hsize}{$
        \left[\begin{matrix}5 & 0\\0 & 10\end{matrix}\right]
        $}\end{equation*}

    

    \begin{Verbatim}[commandchars=\\\{\}]
{\color{incolor}In [{\color{incolor}196}]:} \PY{n}{Q}\PY{o}{*}\PY{n}{L}\PY{o}{*}\PY{n}{Q}\PY{o}{.}\PY{n}{inv}\PY{p}{(}\PY{p}{)}  \PY{c}{# eigendecomposition of A}
\end{Verbatim}
\texttt{\color{outcolor}Out[{\color{outcolor}196}]:}
    
    
        \begin{equation*}\adjustbox{max width=\hsize}{$
        \left[\begin{matrix}9 & -2\\-2 & 6\end{matrix}\right]
        $}\end{equation*}

    

    \begin{Verbatim}[commandchars=\\\{\}]
{\color{incolor}In [{\color{incolor}197}]:} \PY{n}{Q}\PY{o}{.}\PY{n}{inv}\PY{p}{(}\PY{p}{)}\PY{o}{*}\PY{n}{A}\PY{o}{*}\PY{n}{Q}  \PY{c}{# obtain L from A and Q}
\end{Verbatim}
\texttt{\color{outcolor}Out[{\color{outcolor}197}]:}
    
    
        \begin{equation*}\adjustbox{max width=\hsize}{$
        \left[\begin{matrix}5 & 0\\0 & 10\end{matrix}\right]
        $}\end{equation*}

    

    Not all matrices are diagonalizable. You can check if a matrix is
diagonalizable by calling its \texttt{is\_diagonalizable} method:

    \begin{Verbatim}[commandchars=\\\{\}]
{\color{incolor}In [{\color{incolor}198}]:} \PY{n}{A}\PY{o}{.}\PY{n}{is\PYZus{}diagonalizable}\PY{p}{(}\PY{p}{)}
\end{Verbatim}

            \begin{Verbatim}[commandchars=\\\{\}]
{\color{outcolor}Out[{\color{outcolor}198}]:} True
\end{Verbatim}
        
    \begin{Verbatim}[commandchars=\\\{\}]
{\color{incolor}In [{\color{incolor}199}]:} \PY{n}{B} \PY{o}{=} \PY{n}{Matrix}\PY{p}{(} \PY{p}{[}\PY{p}{[}\PY{l+m+mi}{1}\PY{p}{,} \PY{l+m+mi}{3}\PY{p}{]}\PY{p}{,}
                      \PY{p}{[}\PY{l+m+mi}{0}\PY{p}{,} \PY{l+m+mi}{1}\PY{p}{]}\PY{p}{]} \PY{p}{)}
          \PY{n}{B}\PY{o}{.}\PY{n}{is\PYZus{}diagonalizable}\PY{p}{(}\PY{p}{)}
\end{Verbatim}

            \begin{Verbatim}[commandchars=\\\{\}]
{\color{outcolor}Out[{\color{outcolor}199}]:} False
\end{Verbatim}
        
    \begin{Verbatim}[commandchars=\\\{\}]
{\color{incolor}In [{\color{incolor}200}]:} \PY{n}{B}\PY{o}{.}\PY{n}{eigenvals}\PY{p}{(}\PY{p}{)}  \PY{c}{# eigenvalue 1 with multiplicity 2}
\end{Verbatim}
\texttt{\color{outcolor}Out[{\color{outcolor}200}]:}
    
    
        \begin{equation*}\adjustbox{max width=\hsize}{$
        \left \{ 1 : 2\right \}
        $}\end{equation*}

    

    \begin{Verbatim}[commandchars=\\\{\}]
{\color{incolor}In [{\color{incolor}201}]:} \PY{n}{B}\PY{o}{.}\PY{n}{eigenvects}\PY{p}{(}\PY{p}{)}
\end{Verbatim}
\texttt{\color{outcolor}Out[{\color{outcolor}201}]:}
    
    
        \begin{equation*}\adjustbox{max width=\hsize}{$
        \left [ \left ( 1, \quad 2, \quad \left [ \left[\begin{matrix}1\\0\end{matrix}\right]\right ]\right )\right ]
        $}\end{equation*}

    

    The matrix $B$ is not diagonalizable because it doesn't have a full set
of eigenvectors. To diagonalize a $2\times 2$ matrix, we need two
orthogonal eigenvectors but $B$ has only a single eigenvector.
Therefore, we can't construct the matrix of eigenvectors $Q$ (we're
missing a column!) and so $B$ is not diagonalizable.

Non-square matrices don't have eigenvectors and therefore don't have an
eigendecomposition. Instead, we can use the \emph{singular value
decomposition} to break up a non-square matrix $A$ into left singular
vectors, right singular vectors, and a diagonal matrix of singular
values. Use the \texttt{singular\_values} method on any matrix to find
its singular values.

    \subsection{Conclusion}\label{conclusion}

    I would like to conclude with some words of caution about the overuse of
computers. Computer technology is very powerful and is everywhere around
us, but let's not forget that computers are actually very dumb:
computers are mere calculators and they depend on your knowledge to
direct them. It's important that you learn how to do complicated math by
hand in order to be able to instruct computers to do math for you and to
check the results of your computer calculations. I don't want you to use
the tricks you learned in this tutorial to avoid math problems from now
on and simply rely blindly on \texttt{SymPy} for all your math needs. I
want both you and the computer to become math powerhouses! The computer
will help you with tedious calculations (they're good at that) and
you'll help the computer by guiding it when it gets stuck (humans are
good at that).

    \subsection{Links}\label{links}

    \begin{itemize}
\itemsep1pt\parskip0pt\parsep0pt
\item
  \href{http://ipython.org/install.html}{Installation instructions for
  \texttt{ipython notebook}}
\item
  \href{http://docs.sympy.org/latest/tutorial/intro.html}{The official
  \texttt{SymPy} tutorial}
\item
  \href{http://docs.sympy.org/dev/gotchas.html}{A list of \texttt{SymPy}
  gotchas}
\item
  \href{http://pyvideo.org/speaker/583/matthew-rocklin}{\texttt{SymPy}
  video tutorials by Matthew Rocklin}
\end{itemize}

    \subsection{Book plug}\label{book-plug}

    \begin{figure}[htbp]
\centering
\includegraphics{http://minireference.com/miniref/lib/tpl/miniref/dist/images/productshots/noBSguide_math_physics_softcover.png}
\caption{Cover}
\end{figure}

The examples and math explanations in this tutorial are sourced from the
\emph{No bullshit guide} series of books published by Minireference~Co.
We publish textbooks that make math and physics accessible and
affordable for everyone. If you're interested in learning more about the
math, physics, and calculus topics discussed in this tutorial, check out
the \textbf{No bullshit guide to math and physics}. The book contains
the distilled information that normally comes in two first-year
university books: the introductory physics book (1000+ pages) and the
first-year calculus book (1000+ pages). Would you believe me if I told
you that you can learn the same material from a single book that is
1/7th the size and 1/10th of the price of mainstream textbooks?

This book contains short lessons on math and physics, calculus. Often
calculus and mechanics are taught as separate subjects. It shouldn't be
like that. If you learn calculus without mechanics, it will be boring.
If you learn mechanics without calculus, you won't truly understand what
is going on. This textbook covers both subjects in an integrated manner.

Contents:

\begin{itemize}
\itemsep1pt\parskip0pt\parsep0pt
\item
  High school math
\item
  Vectors
\item
  Mechanics
\item
  Differential calculus
\item
  Integral calculus
\item
  250+ practice problems
\end{itemize}

For more information, see the book's website at
\href{http://minireference.com/}{minireference.com}

The presented linear algebra examples are sourced from the
\href{https://gum.co/noBSLA}{\textbf{No bullshit guide to linear
algebra}}. Check out the book if you're taking a linear algebra course
of if you're missing the prerequisites for learning machine learning,
computer graphics, or quantum mechanics.

I'll close on a note for potential readers who suffer from math-phobia.
Both books start with an introductory chapter that reviews all high
school math concepts needed to make math and physics accessible to
everyone. Don't worry, we'll fix this math-phobia thing right up for
you; \textbf{when you've got \texttt{SymPy} skills, math fears
\emph{you}!}

To stay informed about upcoming titles, follow
{[}@minireference{]}(https://twitter.com/minireference) on twitter and
check out the facebook page at
\href{http://fb.me/noBSguide}{fb.me/noBSguide}.


    % Add a bibliography block to the postdoc
    
    
    
    \end{document}
