
% Default to the notebook output style

    


% Inherit from the specified cell style.




    
\documentclass{article}

    
    
    \usepackage{graphicx} % Used to insert images
    \usepackage{adjustbox} % Used to constrain images to a maximum size 
    \usepackage{color} % Allow colors to be defined
    \usepackage{enumerate} % Needed for markdown enumerations to work
    \usepackage{geometry} % Used to adjust the document margins
    \usepackage{amsmath} % Equations
    \usepackage{amssymb} % Equations
    \usepackage{eurosym} % defines \euro
    \usepackage[mathletters]{ucs} % Extended unicode (utf-8) support
    \usepackage[utf8x]{inputenc} % Allow utf-8 characters in the tex document
    \usepackage{fancyvrb} % verbatim replacement that allows latex
    \usepackage{grffile} % extends the file name processing of package graphics 
                         % to support a larger range 
    % The hyperref package gives us a pdf with properly built
    % internal navigation ('pdf bookmarks' for the table of contents,
    % internal cross-reference links, web links for URLs, etc.)
    \usepackage{hyperref}
    \usepackage{longtable} % longtable support required by pandoc >1.10
    \usepackage{booktabs}  % table support for pandoc > 1.12.2
    

    
    
    \definecolor{orange}{cmyk}{0,0.4,0.8,0.2}
    \definecolor{darkorange}{rgb}{.71,0.21,0.01}
    \definecolor{darkgreen}{rgb}{.12,.54,.11}
    \definecolor{myteal}{rgb}{.26, .44, .56}
    \definecolor{gray}{gray}{0.45}
    \definecolor{lightgray}{gray}{.95}
    \definecolor{mediumgray}{gray}{.8}
    \definecolor{inputbackground}{rgb}{.95, .95, .85}
    \definecolor{outputbackground}{rgb}{.95, .95, .95}
    \definecolor{traceback}{rgb}{1, .95, .95}
    % ansi colors
    \definecolor{red}{rgb}{.6,0,0}
    \definecolor{green}{rgb}{0,.65,0}
    \definecolor{brown}{rgb}{0.6,0.6,0}
    \definecolor{blue}{rgb}{0,.145,.698}
    \definecolor{purple}{rgb}{.698,.145,.698}
    \definecolor{cyan}{rgb}{0,.698,.698}
    \definecolor{lightgray}{gray}{0.5}
    
    % bright ansi colors
    \definecolor{darkgray}{gray}{0.25}
    \definecolor{lightred}{rgb}{1.0,0.39,0.28}
    \definecolor{lightgreen}{rgb}{0.48,0.99,0.0}
    \definecolor{lightblue}{rgb}{0.53,0.81,0.92}
    \definecolor{lightpurple}{rgb}{0.87,0.63,0.87}
    \definecolor{lightcyan}{rgb}{0.5,1.0,0.83}
    
    % commands and environments needed by pandoc snippets
    % extracted from the output of `pandoc -s`
    \DefineVerbatimEnvironment{Highlighting}{Verbatim}{commandchars=\\\{\}}
    % Add ',fontsize=\small' for more characters per line
    \newenvironment{Shaded}{}{}
    \newcommand{\KeywordTok}[1]{\textcolor[rgb]{0.00,0.44,0.13}{\textbf{{#1}}}}
    \newcommand{\DataTypeTok}[1]{\textcolor[rgb]{0.56,0.13,0.00}{{#1}}}
    \newcommand{\DecValTok}[1]{\textcolor[rgb]{0.25,0.63,0.44}{{#1}}}
    \newcommand{\BaseNTok}[1]{\textcolor[rgb]{0.25,0.63,0.44}{{#1}}}
    \newcommand{\FloatTok}[1]{\textcolor[rgb]{0.25,0.63,0.44}{{#1}}}
    \newcommand{\CharTok}[1]{\textcolor[rgb]{0.25,0.44,0.63}{{#1}}}
    \newcommand{\StringTok}[1]{\textcolor[rgb]{0.25,0.44,0.63}{{#1}}}
    \newcommand{\CommentTok}[1]{\textcolor[rgb]{0.38,0.63,0.69}{\textit{{#1}}}}
    \newcommand{\OtherTok}[1]{\textcolor[rgb]{0.00,0.44,0.13}{{#1}}}
    \newcommand{\AlertTok}[1]{\textcolor[rgb]{1.00,0.00,0.00}{\textbf{{#1}}}}
    \newcommand{\FunctionTok}[1]{\textcolor[rgb]{0.02,0.16,0.49}{{#1}}}
    \newcommand{\RegionMarkerTok}[1]{{#1}}
    \newcommand{\ErrorTok}[1]{\textcolor[rgb]{1.00,0.00,0.00}{\textbf{{#1}}}}
    \newcommand{\NormalTok}[1]{{#1}}
    
    % Define a nice break command that doesn't care if a line doesn't already
    % exist.
    \def\br{\hspace*{\fill} \\* }
    % Math Jax compatability definitions
    \def\gt{>}
    \def\lt{<}
    % Document parameters
    \title{Mechanics}
    
    
    

    % Pygments definitions
    
\makeatletter
\def\PY@reset{\let\PY@it=\relax \let\PY@bf=\relax%
    \let\PY@ul=\relax \let\PY@tc=\relax%
    \let\PY@bc=\relax \let\PY@ff=\relax}
\def\PY@tok#1{\csname PY@tok@#1\endcsname}
\def\PY@toks#1+{\ifx\relax#1\empty\else%
    \PY@tok{#1}\expandafter\PY@toks\fi}
\def\PY@do#1{\PY@bc{\PY@tc{\PY@ul{%
    \PY@it{\PY@bf{\PY@ff{#1}}}}}}}
\def\PY#1#2{\PY@reset\PY@toks#1+\relax+\PY@do{#2}}

\def\PY@tok@gd{\def\PY@tc##1{\textcolor[rgb]{0.63,0.00,0.00}{##1}}}
\def\PY@tok@gu{\let\PY@bf=\textbf\def\PY@tc##1{\textcolor[rgb]{0.50,0.00,0.50}{##1}}}
\def\PY@tok@gt{\def\PY@tc##1{\textcolor[rgb]{0.00,0.25,0.82}{##1}}}
\def\PY@tok@gs{\let\PY@bf=\textbf}
\def\PY@tok@gr{\def\PY@tc##1{\textcolor[rgb]{1.00,0.00,0.00}{##1}}}
\def\PY@tok@cm{\let\PY@it=\textit\def\PY@tc##1{\textcolor[rgb]{0.25,0.50,0.50}{##1}}}
\def\PY@tok@vg{\def\PY@tc##1{\textcolor[rgb]{0.10,0.09,0.49}{##1}}}
\def\PY@tok@m{\def\PY@tc##1{\textcolor[rgb]{0.40,0.40,0.40}{##1}}}
\def\PY@tok@mh{\def\PY@tc##1{\textcolor[rgb]{0.40,0.40,0.40}{##1}}}
\def\PY@tok@go{\def\PY@tc##1{\textcolor[rgb]{0.50,0.50,0.50}{##1}}}
\def\PY@tok@ge{\let\PY@it=\textit}
\def\PY@tok@vc{\def\PY@tc##1{\textcolor[rgb]{0.10,0.09,0.49}{##1}}}
\def\PY@tok@il{\def\PY@tc##1{\textcolor[rgb]{0.40,0.40,0.40}{##1}}}
\def\PY@tok@cs{\let\PY@it=\textit\def\PY@tc##1{\textcolor[rgb]{0.25,0.50,0.50}{##1}}}
\def\PY@tok@cp{\def\PY@tc##1{\textcolor[rgb]{0.74,0.48,0.00}{##1}}}
\def\PY@tok@gi{\def\PY@tc##1{\textcolor[rgb]{0.00,0.63,0.00}{##1}}}
\def\PY@tok@gh{\let\PY@bf=\textbf\def\PY@tc##1{\textcolor[rgb]{0.00,0.00,0.50}{##1}}}
\def\PY@tok@ni{\let\PY@bf=\textbf\def\PY@tc##1{\textcolor[rgb]{0.60,0.60,0.60}{##1}}}
\def\PY@tok@nl{\def\PY@tc##1{\textcolor[rgb]{0.63,0.63,0.00}{##1}}}
\def\PY@tok@nn{\let\PY@bf=\textbf\def\PY@tc##1{\textcolor[rgb]{0.00,0.00,1.00}{##1}}}
\def\PY@tok@no{\def\PY@tc##1{\textcolor[rgb]{0.53,0.00,0.00}{##1}}}
\def\PY@tok@na{\def\PY@tc##1{\textcolor[rgb]{0.49,0.56,0.16}{##1}}}
\def\PY@tok@nb{\def\PY@tc##1{\textcolor[rgb]{0.00,0.50,0.00}{##1}}}
\def\PY@tok@nc{\let\PY@bf=\textbf\def\PY@tc##1{\textcolor[rgb]{0.00,0.00,1.00}{##1}}}
\def\PY@tok@nd{\def\PY@tc##1{\textcolor[rgb]{0.67,0.13,1.00}{##1}}}
\def\PY@tok@ne{\let\PY@bf=\textbf\def\PY@tc##1{\textcolor[rgb]{0.82,0.25,0.23}{##1}}}
\def\PY@tok@nf{\def\PY@tc##1{\textcolor[rgb]{0.00,0.00,1.00}{##1}}}
\def\PY@tok@si{\let\PY@bf=\textbf\def\PY@tc##1{\textcolor[rgb]{0.73,0.40,0.53}{##1}}}
\def\PY@tok@s2{\def\PY@tc##1{\textcolor[rgb]{0.73,0.13,0.13}{##1}}}
\def\PY@tok@vi{\def\PY@tc##1{\textcolor[rgb]{0.10,0.09,0.49}{##1}}}
\def\PY@tok@nt{\let\PY@bf=\textbf\def\PY@tc##1{\textcolor[rgb]{0.00,0.50,0.00}{##1}}}
\def\PY@tok@nv{\def\PY@tc##1{\textcolor[rgb]{0.10,0.09,0.49}{##1}}}
\def\PY@tok@s1{\def\PY@tc##1{\textcolor[rgb]{0.73,0.13,0.13}{##1}}}
\def\PY@tok@sh{\def\PY@tc##1{\textcolor[rgb]{0.73,0.13,0.13}{##1}}}
\def\PY@tok@sc{\def\PY@tc##1{\textcolor[rgb]{0.73,0.13,0.13}{##1}}}
\def\PY@tok@sx{\def\PY@tc##1{\textcolor[rgb]{0.00,0.50,0.00}{##1}}}
\def\PY@tok@bp{\def\PY@tc##1{\textcolor[rgb]{0.00,0.50,0.00}{##1}}}
\def\PY@tok@c1{\let\PY@it=\textit\def\PY@tc##1{\textcolor[rgb]{0.25,0.50,0.50}{##1}}}
\def\PY@tok@kc{\let\PY@bf=\textbf\def\PY@tc##1{\textcolor[rgb]{0.00,0.50,0.00}{##1}}}
\def\PY@tok@c{\let\PY@it=\textit\def\PY@tc##1{\textcolor[rgb]{0.25,0.50,0.50}{##1}}}
\def\PY@tok@mf{\def\PY@tc##1{\textcolor[rgb]{0.40,0.40,0.40}{##1}}}
\def\PY@tok@err{\def\PY@bc##1{\fcolorbox[rgb]{1.00,0.00,0.00}{1,1,1}{##1}}}
\def\PY@tok@kd{\let\PY@bf=\textbf\def\PY@tc##1{\textcolor[rgb]{0.00,0.50,0.00}{##1}}}
\def\PY@tok@ss{\def\PY@tc##1{\textcolor[rgb]{0.10,0.09,0.49}{##1}}}
\def\PY@tok@sr{\def\PY@tc##1{\textcolor[rgb]{0.73,0.40,0.53}{##1}}}
\def\PY@tok@mo{\def\PY@tc##1{\textcolor[rgb]{0.40,0.40,0.40}{##1}}}
\def\PY@tok@kn{\let\PY@bf=\textbf\def\PY@tc##1{\textcolor[rgb]{0.00,0.50,0.00}{##1}}}
\def\PY@tok@mi{\def\PY@tc##1{\textcolor[rgb]{0.40,0.40,0.40}{##1}}}
\def\PY@tok@gp{\let\PY@bf=\textbf\def\PY@tc##1{\textcolor[rgb]{0.00,0.00,0.50}{##1}}}
\def\PY@tok@o{\def\PY@tc##1{\textcolor[rgb]{0.40,0.40,0.40}{##1}}}
\def\PY@tok@kr{\let\PY@bf=\textbf\def\PY@tc##1{\textcolor[rgb]{0.00,0.50,0.00}{##1}}}
\def\PY@tok@s{\def\PY@tc##1{\textcolor[rgb]{0.73,0.13,0.13}{##1}}}
\def\PY@tok@kp{\def\PY@tc##1{\textcolor[rgb]{0.00,0.50,0.00}{##1}}}
\def\PY@tok@w{\def\PY@tc##1{\textcolor[rgb]{0.73,0.73,0.73}{##1}}}
\def\PY@tok@kt{\def\PY@tc##1{\textcolor[rgb]{0.69,0.00,0.25}{##1}}}
\def\PY@tok@ow{\let\PY@bf=\textbf\def\PY@tc##1{\textcolor[rgb]{0.67,0.13,1.00}{##1}}}
\def\PY@tok@sb{\def\PY@tc##1{\textcolor[rgb]{0.73,0.13,0.13}{##1}}}
\def\PY@tok@k{\let\PY@bf=\textbf\def\PY@tc##1{\textcolor[rgb]{0.00,0.50,0.00}{##1}}}
\def\PY@tok@se{\let\PY@bf=\textbf\def\PY@tc##1{\textcolor[rgb]{0.73,0.40,0.13}{##1}}}
\def\PY@tok@sd{\let\PY@it=\textit\def\PY@tc##1{\textcolor[rgb]{0.73,0.13,0.13}{##1}}}

\def\PYZbs{\char`\\}
\def\PYZus{\char`\_}
\def\PYZob{\char`\{}
\def\PYZcb{\char`\}}
\def\PYZca{\char`\^}
% for compatibility with earlier versions
\def\PYZat{@}
\def\PYZlb{[}
\def\PYZrb{]}
\makeatother


    % Exact colors from NB
    \definecolor{incolor}{rgb}{0.0, 0.0, 0.5}
    \definecolor{outcolor}{rgb}{0.545, 0.0, 0.0}



    
    % Prevent overflowing lines due to hard-to-break entities
    \sloppy 
    % Setup hyperref package
    \hypersetup{
      breaklinks=true,  % so long urls are correctly broken across lines
      colorlinks=true,
      urlcolor=blue,
      linkcolor=darkorange,
      citecolor=darkgreen,
      }
    % Slightly bigger margins than the latex defaults
    
    \geometry{verbose,tmargin=1in,bmargin=1in,lmargin=1in,rmargin=1in}
    
    

    \begin{document}
    
    
    \maketitle
    
    

    
    \subsection{Mechanics}\label{mechanics}

    The module called
\href{http://pyvideo.org/video/2653/dynamics-and-control-with-python}{\texttt{sympy.physics.mechanics}}
contains elaborate tools for describing mechanical systems, manipulating
reference frames, forces, and torques. These specialized functions are
not necessary for a first-year mechanics course. The basic
\texttt{SymPy} functions like \texttt{solve}, and the vector operations
you learned in the previous sections are powerful enough for basic
Newtonian mechanics.

    \subsubsection{Dynamics}\label{dynamics}

    The net force acting on an object is the sum of all the external forces
acting on it $\vec{F}_{\textrm{net}} = \sum \vec{F}$. Since forces are
vectors, we need to use vector addition to compute the net force.

Compute $\vec{F}_{\textrm{net}}=\vec{F}_1 + \vec{F}_2$, where
$\vec{F}_1=4\hat{\imath}[\mathrm{N}]$ and
$\vec{F}_2 = 5\angle 30^\circ[\mathrm{N}]$:

    \begin{Verbatim}[commandchars=\\\{\}]
{\color{incolor}In [{\color{incolor}154}]:} \PY{n}{F\PYZus{}1} \PY{o}{=}  \PY{n}{Matrix}\PY{p}{(} \PY{p}{[}\PY{l+m+mi}{4}\PY{p}{,}\PY{l+m+mi}{0}\PY{p}{]} \PY{p}{)} 
          \PY{n}{F\PYZus{}2} \PY{o}{=}  \PY{n}{Matrix}\PY{p}{(} \PY{p}{[}\PY{l+m+mi}{5}\PY{o}{*}\PY{n}{cos}\PY{p}{(}\PY{l+m+mi}{30}\PY{o}{*}\PY{n}{pi}\PY{o}{/}\PY{l+m+mi}{180}\PY{p}{)}\PY{p}{,} \PY{l+m+mi}{5}\PY{o}{*}\PY{n}{sin}\PY{p}{(}\PY{l+m+mi}{30}\PY{o}{*}\PY{n}{pi}\PY{o}{/}\PY{l+m+mi}{180}\PY{p}{)} \PY{p}{]} \PY{p}{)}
          \PY{n}{F\PYZus{}net} \PY{o}{=} \PY{n}{F\PYZus{}1} \PY{o}{+} \PY{n}{F\PYZus{}2}
          \PY{n}{F\PYZus{}net}                                    \PY{c}{# in Newtons}
\end{Verbatim}
\texttt{\color{outcolor}Out[{\color{outcolor}154}]:}
    
    
        \begin{equation*}\adjustbox{max width=\hsize}{$
        \left[\begin{matrix}4 + \frac{5 \sqrt{3}}{2}\\\frac{5}{2}\end{matrix}\right]
        $}\end{equation*}

    

    \begin{Verbatim}[commandchars=\\\{\}]
{\color{incolor}In [{\color{incolor}155}]:} \PY{n}{F\PYZus{}net}\PY{o}{.}\PY{n}{evalf}\PY{p}{(}\PY{p}{)}                            \PY{c}{# in Newtons}
\end{Verbatim}
\texttt{\color{outcolor}Out[{\color{outcolor}155}]:}
    
    
        \begin{equation*}\adjustbox{max width=\hsize}{$
        \left[\begin{matrix}8.33012701892219\\2.5\end{matrix}\right]
        $}\end{equation*}

    

    To express the answer in length-and-direction notation, use
\texttt{norm} to find the length of $\vec{F}_{\textrm{net}}$ and
\texttt{atan2} (The function \texttt{atan2(y,x)} computes the correct
direction for all vectors $(x,y)$, unlike \texttt{atan(y/x)} which
requires corrections for angles in the range
$[\frac{\pi}{2}, \frac{3\pi}{2}]$.) to find its direction:

    \begin{Verbatim}[commandchars=\\\{\}]
{\color{incolor}In [{\color{incolor}156}]:} \PY{n}{F\PYZus{}net}\PY{o}{.}\PY{n}{norm}\PY{p}{(}\PY{p}{)}\PY{o}{.}\PY{n}{evalf}\PY{p}{(}\PY{p}{)}                     \PY{c}{# |F\PYZus{}net| in [N]}
\end{Verbatim}
\texttt{\color{outcolor}Out[{\color{outcolor}156}]:}
    
    
        \begin{equation*}\adjustbox{max width=\hsize}{$
        8.69718438067042
        $}\end{equation*}

    

    \begin{Verbatim}[commandchars=\\\{\}]
{\color{incolor}In [{\color{incolor}157}]:} \PY{p}{(}\PY{n}{atan2}\PY{p}{(} \PY{n}{F\PYZus{}net}\PY{p}{[}\PY{l+m+mi}{1}\PY{p}{]}\PY{p}{,}\PY{n}{F\PYZus{}net}\PY{p}{[}\PY{l+m+mi}{0}\PY{p}{]} \PY{p}{)}\PY{o}{*}\PY{l+m+mi}{180}\PY{o}{/}\PY{n}{pi}\PY{p}{)}\PY{o}{.}\PY{n}{n}\PY{p}{(}\PY{p}{)}  \PY{c}{# angle in degrees}
\end{Verbatim}
\texttt{\color{outcolor}Out[{\color{outcolor}157}]:}
    
    
        \begin{equation*}\adjustbox{max width=\hsize}{$
        16.70531380601
        $}\end{equation*}

    

    The net force on the object is
$\vec{F}_{\textrm{net}}= 8.697\angle 16.7^\circ${[}N{]}.

    \subsubsection{Kinematics}\label{kinematics}

    Let $x(t)$ denote the position of an object, $v(t)$ denote its velocity,
and $a(t)$ denote its acceleration. Together $x(t)$, $v(t)$, and $a(t)$
are known as the \emph{equations of motion} of the object.

The equations of motion are related by the derivative operation:

\[
  a(t) \overset{\frac{d}{dt} }{\longleftarrow} v(t) \overset{\frac{d}{dt} }{\longleftarrow} x(t).
\]

Assume we know the initial position $x_i\equiv x(0)$ and the initial
velocity $v_i\equiv v(0)$ of the object and we want to find $x(t)$ for
all later times. We can do this starting from the dynamics of the
problem---the forces acting on the object.

Newton's second law $\vec{F}_{\textrm{net}} = m\vec{a}$ states that a
net force $\vec{F}_{\textrm{net}}$ applied on an object of mass $m$
produces acceleration $\vec{a}$. Thus, we can obtain an objects
acceleration if we know the net force acting on it. Starting from the
knowledge of $a(t)$, we can obtain $v(t)$ by integrating then find
$x(t)$ by integrating $v(t)$:

\[
a(t) \ \ \ \overset{v_i+ \int\!dt }{\longrightarrow} \ \ \ v(t) \ \ \ \overset{x_i+ \int\!dt }{\longrightarrow} \ \ \ x(t).
\]

The reasoning follows from the fundamental theorem of calculus: if
$a(t)$ represents the change in $v(t)$, then the total of $a(t)$
accumulated between $t=t_1$ and $t=t_2$ is equal to the total change in
$v(t)$ between these times: $\Delta v = v(t_2) - v(t_1)$. Similarly, the
integral of $v(t)$ from $t=0$ until $t=\tau$ is equal to
$x(\tau) - x(0)$.

    \subsubsection{Uniform acceleration motion
(UAM)}\label{uniform-acceleration-motion-uam}

    Let's analyze the case where the net force on the object is constant. A
constant force causes a constant acceleration
$a = \frac{F}{m} = \textrm{constant}$. If the acceleration function is
constant over time $a(t)=a$. We find $v(t)$ and $x(t)$ as follows:

    \begin{Verbatim}[commandchars=\\\{\}]
{\color{incolor}In [{\color{incolor}158}]:} \PY{n}{t}\PY{p}{,} \PY{n}{a}\PY{p}{,} \PY{n}{v\PYZus{}i}\PY{p}{,} \PY{n}{x\PYZus{}i} \PY{o}{=} \PY{n}{symbols}\PY{p}{(}\PY{l+s}{'}\PY{l+s}{t a v\PYZus{}i x\PYZus{}i}\PY{l+s}{'}\PY{p}{)}
          \PY{n}{v} \PY{o}{=} \PY{n}{v\PYZus{}i} \PY{o}{+} \PY{n}{integrate}\PY{p}{(}\PY{n}{a}\PY{p}{,} \PY{p}{(}\PY{n}{t}\PY{p}{,} \PY{l+m+mi}{0}\PY{p}{,}\PY{n}{t}\PY{p}{)} \PY{p}{)}
          \PY{n}{v}
\end{Verbatim}
\texttt{\color{outcolor}Out[{\color{outcolor}158}]:}
    
    
        \begin{equation*}\adjustbox{max width=\hsize}{$
        a t + v_{i}
        $}\end{equation*}

    

    \begin{Verbatim}[commandchars=\\\{\}]
{\color{incolor}In [{\color{incolor}159}]:} \PY{n}{x} \PY{o}{=} \PY{n}{x\PYZus{}i} \PY{o}{+} \PY{n}{integrate}\PY{p}{(}\PY{n}{v}\PY{p}{,} \PY{p}{(}\PY{n}{t}\PY{p}{,} \PY{l+m+mi}{0}\PY{p}{,}\PY{n}{t}\PY{p}{)} \PY{p}{)}
          \PY{n}{x}
\end{Verbatim}
\texttt{\color{outcolor}Out[{\color{outcolor}159}]:}
    
    
        \begin{equation*}\adjustbox{max width=\hsize}{$
        \frac{a t^{2}}{2} + t v_{i} + x_{i}
        $}\end{equation*}

    

    You may remember these equations from your high school physics class.
They are the \emph{uniform accelerated motion} (UAM) equations:

\begin{align*}
 a(t) &= a,                                  \\ 
 v(t) &= v_i  + at,                          \\[-2mm] 
 x(t) &= x_i + v_it + \frac{1}{2}at^2.
\end{align*}

In high school, you probably had to memorize these equations. Now you
know how to derive them yourself starting from first principles.

For the sake of completeness, we'll now derive the fourth UAM equation,
which relates the object's final velocity to the initial velocity, the
displacement, and the acceleration, without reference to time:

    \begin{Verbatim}[commandchars=\\\{\}]
{\color{incolor}In [{\color{incolor}160}]:} \PY{p}{(}\PY{n}{v}\PY{o}{*}\PY{n}{v}\PY{p}{)}\PY{o}{.}\PY{n}{expand}\PY{p}{(}\PY{p}{)}
\end{Verbatim}
\texttt{\color{outcolor}Out[{\color{outcolor}160}]:}
    
    
        \begin{equation*}\adjustbox{max width=\hsize}{$
        a^{2} t^{2} + 2 a t v_{i} + v_{i}^{2}
        $}\end{equation*}

    

    \begin{Verbatim}[commandchars=\\\{\}]
{\color{incolor}In [{\color{incolor}161}]:} \PY{p}{(}\PY{p}{(}\PY{n}{v}\PY{o}{*}\PY{n}{v}\PY{p}{)}\PY{o}{.}\PY{n}{expand}\PY{p}{(}\PY{p}{)} \PY{o}{-} \PY{l+m+mi}{2}\PY{o}{*}\PY{n}{a}\PY{o}{*}\PY{n}{x}\PY{p}{)}\PY{o}{.}\PY{n}{simplify}\PY{p}{(}\PY{p}{)}
\end{Verbatim}
\texttt{\color{outcolor}Out[{\color{outcolor}161}]:}
    
    
        \begin{equation*}\adjustbox{max width=\hsize}{$
        - 2 a x_{i} + v_{i}^{2}
        $}\end{equation*}

    

    The above calculation shows $v_f^2 - 2ax_f = -2ax_i + v_i^2$. After
moving the term $2ax_f$ to the other side of the equation, we obtain

\begin{align*}
 (v(t))^2 \ = \ v_f^2 =  v_i^2  + 2a\Delta x \ = \  v_i^2  + 2a(x_f-x_i).
\end{align*}

The fourth equation is important for practical purposes because it
allows us to solve physics problems in a time-less manner.

    \paragraph{Example}\label{example}

    Find the position function of an object at time $t=3[\mathrm{s}]$, if it
starts from $x_i=20[\mathrm{m}]$ with $v_i=10[\mathrm{m/s}]$ and
undergoes a constant acceleration of $a=5[\mathrm{m/s^2}]$. What is the
object's velocity at $t=3[\mathrm{s}]$?

    \begin{Verbatim}[commandchars=\\\{\}]
{\color{incolor}In [{\color{incolor}162}]:} \PY{n}{x\PYZus{}i} \PY{o}{=} \PY{l+m+mi}{20}  \PY{c}{# initial position}
          \PY{n}{v\PYZus{}i} \PY{o}{=} \PY{l+m+mi}{10}  \PY{c}{# initial velocity}
          \PY{n}{a}   \PY{o}{=} \PY{l+m+mi}{5}   \PY{c}{# acceleration (constant during motion)}
          \PY{n}{x} \PY{o}{=} \PY{n}{x\PYZus{}i} \PY{o}{+} \PY{n}{integrate}\PY{p}{(} \PY{n}{v\PYZus{}i}\PY{o}{+}\PY{n}{integrate}\PY{p}{(}\PY{n}{a}\PY{p}{,}\PY{p}{(}\PY{n}{t}\PY{p}{,}\PY{l+m+mi}{0}\PY{p}{,}\PY{n}{t}\PY{p}{)}\PY{p}{)}\PY{p}{,} \PY{p}{(}\PY{n}{t}\PY{p}{,}\PY{l+m+mi}{0}\PY{p}{,}\PY{n}{t}\PY{p}{)} \PY{p}{)}   
          \PY{n}{x}
\end{Verbatim}
\texttt{\color{outcolor}Out[{\color{outcolor}162}]:}
    
    
        \begin{equation*}\adjustbox{max width=\hsize}{$
        \frac{5 t^{2}}{2} + 10 t + 20
        $}\end{equation*}

    

    \begin{Verbatim}[commandchars=\\\{\}]
{\color{incolor}In [{\color{incolor}163}]:} \PY{n}{x}\PY{o}{.}\PY{n}{subs}\PY{p}{(}\PY{p}{\PYZob{}}\PY{n}{t}\PY{p}{:}\PY{l+m+mi}{3}\PY{p}{\PYZcb{}}\PY{p}{)}\PY{o}{.}\PY{n}{n}\PY{p}{(}\PY{p}{)}          \PY{c}{# x(3) in [m]}
\end{Verbatim}
\texttt{\color{outcolor}Out[{\color{outcolor}163}]:}
    
    
        \begin{equation*}\adjustbox{max width=\hsize}{$
        72.5
        $}\end{equation*}

    

    \begin{Verbatim}[commandchars=\\\{\}]
{\color{incolor}In [{\color{incolor}164}]:} \PY{n}{diff}\PY{p}{(}\PY{n}{x}\PY{p}{,}\PY{n}{t}\PY{p}{)}\PY{o}{.}\PY{n}{subs}\PY{p}{(}\PY{p}{\PYZob{}}\PY{n}{t}\PY{p}{:}\PY{l+m+mi}{3}\PY{p}{\PYZcb{}}\PY{p}{)}\PY{o}{.}\PY{n}{n}\PY{p}{(}\PY{p}{)}  \PY{c}{# v(3) in [m/s]}
\end{Verbatim}
\texttt{\color{outcolor}Out[{\color{outcolor}164}]:}
    
    
        \begin{equation*}\adjustbox{max width=\hsize}{$
        25.0
        $}\end{equation*}

    

    If you think about it, physics knowledge combined with computer skills
is like a superpower!

    \subsubsection{General equations of
motion}\label{general-equations-of-motion}

    The procedure
$a(t) \ \overset{v_i+ \int\!dt }{\longrightarrow} \ v(t) \ \overset{x_i+ \int\!dt }{\longrightarrow} \ x(t)$
can be used to obtain the position function $x(t)$ even when the
acceleration is not constant. Suppose the acceleration of an object is
$a(t)=\sqrt{k t}$; what is its $x(t)$?

    \begin{Verbatim}[commandchars=\\\{\}]
{\color{incolor}In [{\color{incolor}165}]:} \PY{n}{t}\PY{p}{,} \PY{n}{v\PYZus{}i}\PY{p}{,} \PY{n}{x\PYZus{}i}\PY{p}{,} \PY{n}{k} \PY{o}{=} \PY{n}{symbols}\PY{p}{(}\PY{l+s}{'}\PY{l+s}{t v\PYZus{}i x\PYZus{}i k}\PY{l+s}{'}\PY{p}{)}
          \PY{n}{a} \PY{o}{=} \PY{n}{sqrt}\PY{p}{(}\PY{n}{k}\PY{o}{*}\PY{n}{t}\PY{p}{)}
          \PY{n}{x} \PY{o}{=} \PY{n}{x\PYZus{}i} \PY{o}{+} \PY{n}{integrate}\PY{p}{(} \PY{n}{v\PYZus{}i}\PY{o}{+}\PY{n}{integrate}\PY{p}{(}\PY{n}{a}\PY{p}{,}\PY{p}{(}\PY{n}{t}\PY{p}{,}\PY{l+m+mi}{0}\PY{p}{,}\PY{n}{t}\PY{p}{)}\PY{p}{)}\PY{p}{,} \PY{p}{(}\PY{n}{t}\PY{p}{,} \PY{l+m+mi}{0}\PY{p}{,}\PY{n}{t}\PY{p}{)} \PY{p}{)}
          \PY{n}{x}
\end{Verbatim}
\texttt{\color{outcolor}Out[{\color{outcolor}165}]:}
    
    
        \begin{equation*}\adjustbox{max width=\hsize}{$
        t v_{i} + x_{i} + \frac{4 \left(k t\right)^{\frac{5}{2}}}{15 k^{2}}
        $}\end{equation*}

    

    \subsubsection{Potential energy}\label{potential-energy}

    Instead of working with the kinematic equations of motion $x(t)$,
$v(t)$, and $a(t)$ which depend on time, we can solve physics problems
using \emph{energy} calculations. A key connection between the world of
forces and the world of energy is the concept of \emph{potential
energy}. If you move an object against a conservative force (think
raising a ball in the air against the force of gravity), you can think
of the work you do agains the force as being stored in the potential
energy of the object.

For each force $\vec{F}(x)$ there is a corresponding potential energy
$U_F(x)$. The change in potential energy associated with the force
$\vec{F}(x)$ and displacement $\vec{d}$ is defined as the negative of
the work done by the force during the displacement:
$U_F(x) = - W = - \int_{\vec{d}} \vec{F}(x)\cdot d\vec{x}$.

The potential energies associated with gravity
$\vec{F}_g = -mg\hat{\jmath}$ and the force of a spring
$\vec{F}_s = -k\vec{x}$ are calculated as follows:

    \begin{Verbatim}[commandchars=\\\{\}]
{\color{incolor}In [{\color{incolor}166}]:} \PY{n}{x}\PY{p}{,} \PY{n}{y} \PY{o}{=} \PY{n}{symbols}\PY{p}{(}\PY{l+s}{'}\PY{l+s}{x y}\PY{l+s}{'}\PY{p}{)}
          \PY{n}{m}\PY{p}{,} \PY{n}{g}\PY{p}{,} \PY{n}{k}\PY{p}{,} \PY{n}{h} \PY{o}{=} \PY{n}{symbols}\PY{p}{(}\PY{l+s}{'}\PY{l+s}{m g k h}\PY{l+s}{'}\PY{p}{)}
          \PY{n}{F\PYZus{}g} \PY{o}{=} \PY{o}{-}\PY{n}{m}\PY{o}{*}\PY{n}{g}  \PY{c}{# Force of gravity on mass m }
          \PY{n}{U\PYZus{}g} \PY{o}{=} \PY{o}{-} \PY{n}{integrate}\PY{p}{(} \PY{n}{F\PYZus{}g}\PY{p}{,} \PY{p}{(}\PY{n}{y}\PY{p}{,}\PY{l+m+mi}{0}\PY{p}{,}\PY{n}{h}\PY{p}{)} \PY{p}{)}
          \PY{n}{U\PYZus{}g}         \PY{c}{# Grav. potential energy}
\end{Verbatim}
\texttt{\color{outcolor}Out[{\color{outcolor}166}]:}
    
    
        \begin{equation*}\adjustbox{max width=\hsize}{$
        g h m
        $}\end{equation*}

    

    \begin{Verbatim}[commandchars=\\\{\}]
{\color{incolor}In [{\color{incolor}167}]:} \PY{n}{F\PYZus{}s} \PY{o}{=} \PY{o}{-}\PY{n}{k}\PY{o}{*}\PY{n}{x}  \PY{c}{# Spring force for displacement x }
          \PY{n}{U\PYZus{}s} \PY{o}{=} \PY{o}{-} \PY{n}{integrate}\PY{p}{(} \PY{n}{F\PYZus{}s}\PY{p}{,} \PY{p}{(}\PY{n}{x}\PY{p}{,}\PY{l+m+mi}{0}\PY{p}{,}\PY{n}{x}\PY{p}{)} \PY{p}{)}
          \PY{n}{U\PYZus{}s}         \PY{c}{# Spring potential energy}
\end{Verbatim}
\texttt{\color{outcolor}Out[{\color{outcolor}167}]:}
    
    
        \begin{equation*}\adjustbox{max width=\hsize}{$
        \frac{k x^{2}}{2}
        $}\end{equation*}

    

    Note the negative sign in the formula defining the potential energy.
This negative is canceled by the negative sign of the dot product
$\vec{F}\cdot d\vec{x}$: when the force acts in the direction opposite
to the displacement, the work done by the force is negative.

    \subsubsection{Simple harmonic motion}\label{simple-harmonic-motion}

    The force exerted by a spring is given by the formula $F=-kx$. If the
only force acting on a mass $m$ is the force of a spring, we can use
Newton's second law to obtain the following equation:

\[
  F=ma  
  \quad \Rightarrow \quad
  -kx = ma   
  \quad \Rightarrow \quad
  -kx(t) = m\frac{d^2}{dt^2}\Big[x(t)\Big].
\]

The motion of a mass-spring system is described by the
\emph{differential equation} $\frac{d^2}{dt^2}x(t) + \omega^2 x(t)=0$,
where the constant $\omega = \sqrt{\frac{k}{m}}$ is called the angular
frequency. We can find the position function $x(t)$ using the
\texttt{dsolve} method:

    \begin{Verbatim}[commandchars=\\\{\}]
{\color{incolor}In [{\color{incolor}168}]:} \PY{n}{t} \PY{o}{=} \PY{n}{Symbol}\PY{p}{(}\PY{l+s}{'}\PY{l+s}{t}\PY{l+s}{'}\PY{p}{)}                 \PY{c}{# time t}
          \PY{n}{x} \PY{o}{=} \PY{n}{Function}\PY{p}{(}\PY{l+s}{'}\PY{l+s}{x}\PY{l+s}{'}\PY{p}{)}               \PY{c}{# position function x(t)}
          \PY{n}{w} \PY{o}{=} \PY{n}{Symbol}\PY{p}{(}\PY{l+s}{'}\PY{l+s}{w}\PY{l+s}{'}\PY{p}{,} \PY{n}{positive}\PY{o}{=}\PY{k}{True}\PY{p}{)}  \PY{c}{# angular frequency w}
          \PY{n}{sol} \PY{o}{=} \PY{n}{dsolve}\PY{p}{(} \PY{n}{diff}\PY{p}{(}\PY{n}{x}\PY{p}{(}\PY{n}{t}\PY{p}{)}\PY{p}{,}\PY{n}{t}\PY{p}{,}\PY{n}{t}\PY{p}{)} \PY{o}{+} \PY{n}{w}\PY{o}{*}\PY{o}{*}\PY{l+m+mi}{2}\PY{o}{*}\PY{n}{x}\PY{p}{(}\PY{n}{t}\PY{p}{)}\PY{p}{,} \PY{n}{x}\PY{p}{(}\PY{n}{t}\PY{p}{)} \PY{p}{)}
          \PY{n}{sol}
\end{Verbatim}
\texttt{\color{outcolor}Out[{\color{outcolor}168}]:}
    
    
        \begin{equation*}\adjustbox{max width=\hsize}{$
        x{\left (t \right )} = C_{1} \sin{\left (t w \right )} + C_{2} \cos{\left (t w \right )}
        $}\end{equation*}

    

    \begin{Verbatim}[commandchars=\\\{\}]
{\color{incolor}In [{\color{incolor}169}]:} \PY{n}{x} \PY{o}{=} \PY{n}{sol}\PY{o}{.}\PY{n}{rhs}           
          \PY{n}{x}
\end{Verbatim}
\texttt{\color{outcolor}Out[{\color{outcolor}169}]:}
    
    
        \begin{equation*}\adjustbox{max width=\hsize}{$
        C_{1} \sin{\left (t w \right )} + C_{2} \cos{\left (t w \right )}
        $}\end{equation*}

    

    Note the solution $x(t)=C_1\sin(\omega t)+C_2 \cos(\omega t)$ is
equivalent to $x(t) = A\cos(\omega t + \phi)$, which is more commonly
used to describe simple harmonic motion. We can use the \texttt{expand}
function with the argument \texttt{trig=True} to convince ourselves of
this equivalence:

    \begin{Verbatim}[commandchars=\\\{\}]
{\color{incolor}In [{\color{incolor}170}]:} \PY{n}{A}\PY{p}{,} \PY{n}{phi} \PY{o}{=} \PY{n}{symbols}\PY{p}{(}\PY{l+s}{"}\PY{l+s}{A phi}\PY{l+s}{"}\PY{p}{)}
          \PY{p}{(}\PY{n}{A}\PY{o}{*}\PY{n}{cos}\PY{p}{(}\PY{n}{w}\PY{o}{*}\PY{n}{t} \PY{o}{-} \PY{n}{phi}\PY{p}{)}\PY{p}{)}\PY{o}{.}\PY{n}{expand}\PY{p}{(}\PY{n}{trig}\PY{o}{=}\PY{k}{True}\PY{p}{)}
\end{Verbatim}
\texttt{\color{outcolor}Out[{\color{outcolor}170}]:}
    
    
        \begin{equation*}\adjustbox{max width=\hsize}{$
        A \sin{\left (\phi \right )} \sin{\left (t w \right )} + A \cos{\left (\phi \right )} \cos{\left (t w \right )}
        $}\end{equation*}

    

    If we define $C_1=A\sin(\phi)$ and $C_2=A\cos(\phi)$, we obtain the form
$x(t)=C_1\sin(\omega t)+C_2 \cos(\omega t)$ that \texttt{SymPy} found.

    \subsubsection{Conservation of energy}\label{conservation-of-energy}

    We can verify that the total energy of the mass-spring system is
conserved by showing $E_T(t) = U_s(t) + K(t) = \textrm{constant}$:

    \begin{Verbatim}[commandchars=\\\{\}]
{\color{incolor}In [{\color{incolor}171}]:} \PY{n}{x} \PY{o}{=} \PY{n}{sol}\PY{o}{.}\PY{n}{rhs}\PY{o}{.}\PY{n}{subs}\PY{p}{(}\PY{p}{\PYZob{}}\PY{l+s}{"}\PY{l+s}{C1}\PY{l+s}{"}\PY{p}{:}\PY{l+m+mi}{0}\PY{p}{,}\PY{l+s}{"}\PY{l+s}{C2}\PY{l+s}{"}\PY{p}{:}\PY{n}{A}\PY{p}{\PYZcb{}}\PY{p}{)} 
          \PY{n}{x}
\end{Verbatim}
\texttt{\color{outcolor}Out[{\color{outcolor}171}]:}
    
    
        \begin{equation*}\adjustbox{max width=\hsize}{$
        A \cos{\left (t w \right )}
        $}\end{equation*}

    

    \begin{Verbatim}[commandchars=\\\{\}]
{\color{incolor}In [{\color{incolor}172}]:} \PY{n}{v} \PY{o}{=} \PY{n}{diff}\PY{p}{(}\PY{n}{x}\PY{p}{,} \PY{n}{t}\PY{p}{)}
          \PY{n}{v}
\end{Verbatim}
\texttt{\color{outcolor}Out[{\color{outcolor}172}]:}
    
    
        \begin{equation*}\adjustbox{max width=\hsize}{$
        - A w \sin{\left (t w \right )}
        $}\end{equation*}

    

    \begin{Verbatim}[commandchars=\\\{\}]
{\color{incolor}In [{\color{incolor}173}]:} \PY{n}{E\PYZus{}T} \PY{o}{=} \PY{p}{(}\PY{l+m+mf}{0.5}\PY{o}{*}\PY{n}{k}\PY{o}{*}\PY{n}{x}\PY{o}{*}\PY{o}{*}\PY{l+m+mi}{2} \PY{o}{+} \PY{l+m+mf}{0.5}\PY{o}{*}\PY{n}{m}\PY{o}{*}\PY{n}{v}\PY{o}{*}\PY{o}{*}\PY{l+m+mi}{2}\PY{p}{)}\PY{o}{.}\PY{n}{simplify}\PY{p}{(}\PY{p}{)}
          \PY{n}{E\PYZus{}T}
\end{Verbatim}
\texttt{\color{outcolor}Out[{\color{outcolor}173}]:}
    
    
        \begin{equation*}\adjustbox{max width=\hsize}{$
        0.5 A^{2} \left(k \cos^{2}{\left (t w \right )} + m w^{2} \sin^{2}{\left (t w \right )}\right)
        $}\end{equation*}

    

    \begin{Verbatim}[commandchars=\\\{\}]
{\color{incolor}In [{\color{incolor}174}]:} \PY{n}{E\PYZus{}T}\PY{o}{.}\PY{n}{subs}\PY{p}{(}\PY{p}{\PYZob{}}\PY{n}{k}\PY{p}{:}\PY{n}{m}\PY{o}{*}\PY{n}{w}\PY{o}{*}\PY{o}{*}\PY{l+m+mi}{2}\PY{p}{\PYZcb{}}\PY{p}{)}\PY{o}{.}\PY{n}{simplify}\PY{p}{(}\PY{p}{)}     \PY{c}{# = K\PYZus{}max}
\end{Verbatim}
\texttt{\color{outcolor}Out[{\color{outcolor}174}]:}
    
    
        \begin{equation*}\adjustbox{max width=\hsize}{$
        0.5 A^{2} m w^{2}
        $}\end{equation*}

    

    \begin{Verbatim}[commandchars=\\\{\}]
{\color{incolor}In [{\color{incolor}175}]:} \PY{n}{E\PYZus{}T}\PY{o}{.}\PY{n}{subs}\PY{p}{(}\PY{p}{\PYZob{}}\PY{n}{w}\PY{p}{:}\PY{n}{sqrt}\PY{p}{(}\PY{n}{k}\PY{o}{/}\PY{n}{m}\PY{p}{)}\PY{p}{\PYZcb{}}\PY{p}{)}\PY{o}{.}\PY{n}{simplify}\PY{p}{(}\PY{p}{)}  \PY{c}{# = U\PYZus{}max}
\end{Verbatim}
\texttt{\color{outcolor}Out[{\color{outcolor}175}]:}
    
    
        \begin{equation*}\adjustbox{max width=\hsize}{$
        0.5 A^{2} k
        $}\end{equation*}

    


    % Add a bibliography block to the postdoc
    
    
    
    \end{document}
