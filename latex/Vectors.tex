
% Default to the notebook output style

    


% Inherit from the specified cell style.




    
\documentclass{article}

    
    
    \usepackage{graphicx} % Used to insert images
    \usepackage{adjustbox} % Used to constrain images to a maximum size 
    \usepackage{color} % Allow colors to be defined
    \usepackage{enumerate} % Needed for markdown enumerations to work
    \usepackage{geometry} % Used to adjust the document margins
    \usepackage{amsmath} % Equations
    \usepackage{amssymb} % Equations
    \usepackage{eurosym} % defines \euro
    \usepackage[mathletters]{ucs} % Extended unicode (utf-8) support
    \usepackage[utf8x]{inputenc} % Allow utf-8 characters in the tex document
    \usepackage{fancyvrb} % verbatim replacement that allows latex
    \usepackage{grffile} % extends the file name processing of package graphics 
                         % to support a larger range 
    % The hyperref package gives us a pdf with properly built
    % internal navigation ('pdf bookmarks' for the table of contents,
    % internal cross-reference links, web links for URLs, etc.)
    \usepackage{hyperref}
    \usepackage{longtable} % longtable support required by pandoc >1.10
    \usepackage{booktabs}  % table support for pandoc > 1.12.2
    

    
    
    \definecolor{orange}{cmyk}{0,0.4,0.8,0.2}
    \definecolor{darkorange}{rgb}{.71,0.21,0.01}
    \definecolor{darkgreen}{rgb}{.12,.54,.11}
    \definecolor{myteal}{rgb}{.26, .44, .56}
    \definecolor{gray}{gray}{0.45}
    \definecolor{lightgray}{gray}{.95}
    \definecolor{mediumgray}{gray}{.8}
    \definecolor{inputbackground}{rgb}{.95, .95, .85}
    \definecolor{outputbackground}{rgb}{.95, .95, .95}
    \definecolor{traceback}{rgb}{1, .95, .95}
    % ansi colors
    \definecolor{red}{rgb}{.6,0,0}
    \definecolor{green}{rgb}{0,.65,0}
    \definecolor{brown}{rgb}{0.6,0.6,0}
    \definecolor{blue}{rgb}{0,.145,.698}
    \definecolor{purple}{rgb}{.698,.145,.698}
    \definecolor{cyan}{rgb}{0,.698,.698}
    \definecolor{lightgray}{gray}{0.5}
    
    % bright ansi colors
    \definecolor{darkgray}{gray}{0.25}
    \definecolor{lightred}{rgb}{1.0,0.39,0.28}
    \definecolor{lightgreen}{rgb}{0.48,0.99,0.0}
    \definecolor{lightblue}{rgb}{0.53,0.81,0.92}
    \definecolor{lightpurple}{rgb}{0.87,0.63,0.87}
    \definecolor{lightcyan}{rgb}{0.5,1.0,0.83}
    
    % commands and environments needed by pandoc snippets
    % extracted from the output of `pandoc -s`
    \DefineVerbatimEnvironment{Highlighting}{Verbatim}{commandchars=\\\{\}}
    % Add ',fontsize=\small' for more characters per line
    \newenvironment{Shaded}{}{}
    \newcommand{\KeywordTok}[1]{\textcolor[rgb]{0.00,0.44,0.13}{\textbf{{#1}}}}
    \newcommand{\DataTypeTok}[1]{\textcolor[rgb]{0.56,0.13,0.00}{{#1}}}
    \newcommand{\DecValTok}[1]{\textcolor[rgb]{0.25,0.63,0.44}{{#1}}}
    \newcommand{\BaseNTok}[1]{\textcolor[rgb]{0.25,0.63,0.44}{{#1}}}
    \newcommand{\FloatTok}[1]{\textcolor[rgb]{0.25,0.63,0.44}{{#1}}}
    \newcommand{\CharTok}[1]{\textcolor[rgb]{0.25,0.44,0.63}{{#1}}}
    \newcommand{\StringTok}[1]{\textcolor[rgb]{0.25,0.44,0.63}{{#1}}}
    \newcommand{\CommentTok}[1]{\textcolor[rgb]{0.38,0.63,0.69}{\textit{{#1}}}}
    \newcommand{\OtherTok}[1]{\textcolor[rgb]{0.00,0.44,0.13}{{#1}}}
    \newcommand{\AlertTok}[1]{\textcolor[rgb]{1.00,0.00,0.00}{\textbf{{#1}}}}
    \newcommand{\FunctionTok}[1]{\textcolor[rgb]{0.02,0.16,0.49}{{#1}}}
    \newcommand{\RegionMarkerTok}[1]{{#1}}
    \newcommand{\ErrorTok}[1]{\textcolor[rgb]{1.00,0.00,0.00}{\textbf{{#1}}}}
    \newcommand{\NormalTok}[1]{{#1}}
    
    % Define a nice break command that doesn't care if a line doesn't already
    % exist.
    \def\br{\hspace*{\fill} \\* }
    % Math Jax compatability definitions
    \def\gt{>}
    \def\lt{<}
    % Document parameters
    \title{Vectors}
    
    
    

    % Pygments definitions
    
\makeatletter
\def\PY@reset{\let\PY@it=\relax \let\PY@bf=\relax%
    \let\PY@ul=\relax \let\PY@tc=\relax%
    \let\PY@bc=\relax \let\PY@ff=\relax}
\def\PY@tok#1{\csname PY@tok@#1\endcsname}
\def\PY@toks#1+{\ifx\relax#1\empty\else%
    \PY@tok{#1}\expandafter\PY@toks\fi}
\def\PY@do#1{\PY@bc{\PY@tc{\PY@ul{%
    \PY@it{\PY@bf{\PY@ff{#1}}}}}}}
\def\PY#1#2{\PY@reset\PY@toks#1+\relax+\PY@do{#2}}

\def\PY@tok@gd{\def\PY@tc##1{\textcolor[rgb]{0.63,0.00,0.00}{##1}}}
\def\PY@tok@gu{\let\PY@bf=\textbf\def\PY@tc##1{\textcolor[rgb]{0.50,0.00,0.50}{##1}}}
\def\PY@tok@gt{\def\PY@tc##1{\textcolor[rgb]{0.00,0.25,0.82}{##1}}}
\def\PY@tok@gs{\let\PY@bf=\textbf}
\def\PY@tok@gr{\def\PY@tc##1{\textcolor[rgb]{1.00,0.00,0.00}{##1}}}
\def\PY@tok@cm{\let\PY@it=\textit\def\PY@tc##1{\textcolor[rgb]{0.25,0.50,0.50}{##1}}}
\def\PY@tok@vg{\def\PY@tc##1{\textcolor[rgb]{0.10,0.09,0.49}{##1}}}
\def\PY@tok@m{\def\PY@tc##1{\textcolor[rgb]{0.40,0.40,0.40}{##1}}}
\def\PY@tok@mh{\def\PY@tc##1{\textcolor[rgb]{0.40,0.40,0.40}{##1}}}
\def\PY@tok@go{\def\PY@tc##1{\textcolor[rgb]{0.50,0.50,0.50}{##1}}}
\def\PY@tok@ge{\let\PY@it=\textit}
\def\PY@tok@vc{\def\PY@tc##1{\textcolor[rgb]{0.10,0.09,0.49}{##1}}}
\def\PY@tok@il{\def\PY@tc##1{\textcolor[rgb]{0.40,0.40,0.40}{##1}}}
\def\PY@tok@cs{\let\PY@it=\textit\def\PY@tc##1{\textcolor[rgb]{0.25,0.50,0.50}{##1}}}
\def\PY@tok@cp{\def\PY@tc##1{\textcolor[rgb]{0.74,0.48,0.00}{##1}}}
\def\PY@tok@gi{\def\PY@tc##1{\textcolor[rgb]{0.00,0.63,0.00}{##1}}}
\def\PY@tok@gh{\let\PY@bf=\textbf\def\PY@tc##1{\textcolor[rgb]{0.00,0.00,0.50}{##1}}}
\def\PY@tok@ni{\let\PY@bf=\textbf\def\PY@tc##1{\textcolor[rgb]{0.60,0.60,0.60}{##1}}}
\def\PY@tok@nl{\def\PY@tc##1{\textcolor[rgb]{0.63,0.63,0.00}{##1}}}
\def\PY@tok@nn{\let\PY@bf=\textbf\def\PY@tc##1{\textcolor[rgb]{0.00,0.00,1.00}{##1}}}
\def\PY@tok@no{\def\PY@tc##1{\textcolor[rgb]{0.53,0.00,0.00}{##1}}}
\def\PY@tok@na{\def\PY@tc##1{\textcolor[rgb]{0.49,0.56,0.16}{##1}}}
\def\PY@tok@nb{\def\PY@tc##1{\textcolor[rgb]{0.00,0.50,0.00}{##1}}}
\def\PY@tok@nc{\let\PY@bf=\textbf\def\PY@tc##1{\textcolor[rgb]{0.00,0.00,1.00}{##1}}}
\def\PY@tok@nd{\def\PY@tc##1{\textcolor[rgb]{0.67,0.13,1.00}{##1}}}
\def\PY@tok@ne{\let\PY@bf=\textbf\def\PY@tc##1{\textcolor[rgb]{0.82,0.25,0.23}{##1}}}
\def\PY@tok@nf{\def\PY@tc##1{\textcolor[rgb]{0.00,0.00,1.00}{##1}}}
\def\PY@tok@si{\let\PY@bf=\textbf\def\PY@tc##1{\textcolor[rgb]{0.73,0.40,0.53}{##1}}}
\def\PY@tok@s2{\def\PY@tc##1{\textcolor[rgb]{0.73,0.13,0.13}{##1}}}
\def\PY@tok@vi{\def\PY@tc##1{\textcolor[rgb]{0.10,0.09,0.49}{##1}}}
\def\PY@tok@nt{\let\PY@bf=\textbf\def\PY@tc##1{\textcolor[rgb]{0.00,0.50,0.00}{##1}}}
\def\PY@tok@nv{\def\PY@tc##1{\textcolor[rgb]{0.10,0.09,0.49}{##1}}}
\def\PY@tok@s1{\def\PY@tc##1{\textcolor[rgb]{0.73,0.13,0.13}{##1}}}
\def\PY@tok@sh{\def\PY@tc##1{\textcolor[rgb]{0.73,0.13,0.13}{##1}}}
\def\PY@tok@sc{\def\PY@tc##1{\textcolor[rgb]{0.73,0.13,0.13}{##1}}}
\def\PY@tok@sx{\def\PY@tc##1{\textcolor[rgb]{0.00,0.50,0.00}{##1}}}
\def\PY@tok@bp{\def\PY@tc##1{\textcolor[rgb]{0.00,0.50,0.00}{##1}}}
\def\PY@tok@c1{\let\PY@it=\textit\def\PY@tc##1{\textcolor[rgb]{0.25,0.50,0.50}{##1}}}
\def\PY@tok@kc{\let\PY@bf=\textbf\def\PY@tc##1{\textcolor[rgb]{0.00,0.50,0.00}{##1}}}
\def\PY@tok@c{\let\PY@it=\textit\def\PY@tc##1{\textcolor[rgb]{0.25,0.50,0.50}{##1}}}
\def\PY@tok@mf{\def\PY@tc##1{\textcolor[rgb]{0.40,0.40,0.40}{##1}}}
\def\PY@tok@err{\def\PY@bc##1{\fcolorbox[rgb]{1.00,0.00,0.00}{1,1,1}{##1}}}
\def\PY@tok@kd{\let\PY@bf=\textbf\def\PY@tc##1{\textcolor[rgb]{0.00,0.50,0.00}{##1}}}
\def\PY@tok@ss{\def\PY@tc##1{\textcolor[rgb]{0.10,0.09,0.49}{##1}}}
\def\PY@tok@sr{\def\PY@tc##1{\textcolor[rgb]{0.73,0.40,0.53}{##1}}}
\def\PY@tok@mo{\def\PY@tc##1{\textcolor[rgb]{0.40,0.40,0.40}{##1}}}
\def\PY@tok@kn{\let\PY@bf=\textbf\def\PY@tc##1{\textcolor[rgb]{0.00,0.50,0.00}{##1}}}
\def\PY@tok@mi{\def\PY@tc##1{\textcolor[rgb]{0.40,0.40,0.40}{##1}}}
\def\PY@tok@gp{\let\PY@bf=\textbf\def\PY@tc##1{\textcolor[rgb]{0.00,0.00,0.50}{##1}}}
\def\PY@tok@o{\def\PY@tc##1{\textcolor[rgb]{0.40,0.40,0.40}{##1}}}
\def\PY@tok@kr{\let\PY@bf=\textbf\def\PY@tc##1{\textcolor[rgb]{0.00,0.50,0.00}{##1}}}
\def\PY@tok@s{\def\PY@tc##1{\textcolor[rgb]{0.73,0.13,0.13}{##1}}}
\def\PY@tok@kp{\def\PY@tc##1{\textcolor[rgb]{0.00,0.50,0.00}{##1}}}
\def\PY@tok@w{\def\PY@tc##1{\textcolor[rgb]{0.73,0.73,0.73}{##1}}}
\def\PY@tok@kt{\def\PY@tc##1{\textcolor[rgb]{0.69,0.00,0.25}{##1}}}
\def\PY@tok@ow{\let\PY@bf=\textbf\def\PY@tc##1{\textcolor[rgb]{0.67,0.13,1.00}{##1}}}
\def\PY@tok@sb{\def\PY@tc##1{\textcolor[rgb]{0.73,0.13,0.13}{##1}}}
\def\PY@tok@k{\let\PY@bf=\textbf\def\PY@tc##1{\textcolor[rgb]{0.00,0.50,0.00}{##1}}}
\def\PY@tok@se{\let\PY@bf=\textbf\def\PY@tc##1{\textcolor[rgb]{0.73,0.40,0.13}{##1}}}
\def\PY@tok@sd{\let\PY@it=\textit\def\PY@tc##1{\textcolor[rgb]{0.73,0.13,0.13}{##1}}}

\def\PYZbs{\char`\\}
\def\PYZus{\char`\_}
\def\PYZob{\char`\{}
\def\PYZcb{\char`\}}
\def\PYZca{\char`\^}
% for compatibility with earlier versions
\def\PYZat{@}
\def\PYZlb{[}
\def\PYZrb{]}
\makeatother


    % Exact colors from NB
    \definecolor{incolor}{rgb}{0.0, 0.0, 0.5}
    \definecolor{outcolor}{rgb}{0.545, 0.0, 0.0}



    
    % Prevent overflowing lines due to hard-to-break entities
    \sloppy 
    % Setup hyperref package
    \hypersetup{
      breaklinks=true,  % so long urls are correctly broken across lines
      colorlinks=true,
      urlcolor=blue,
      linkcolor=darkorange,
      citecolor=darkgreen,
      }
    % Slightly bigger margins than the latex defaults
    
    \geometry{verbose,tmargin=1in,bmargin=1in,lmargin=1in,rmargin=1in}
    
    

    \begin{document}
    
    
    \maketitle
    
    

    
    \subsection{Vectors}\label{vectors}

    A vector $\vec{v} \in \mathbb{R}^n$ is an $n$-tuple of real numbers. For
example, consider a vector that has three components:

\[
 \vec{v} = (v_1,v_2,v_3) \  \in \  (\mathbb{R},\mathbb{R},\mathbb{R}) \equiv \mathbb{R}^3.
\]

To specify the vector $\vec{v}$, we specify the values for its three
components $v_1$, $v_2$, and $v_3$.

A matrix $A \in \mathbb{R}^{m\times n}$ is a rectangular array of real
numbers with $m$ rows and $n$ columns. A vector is a special type of
matrix; we can think of a vector $\vec{v}\in \mathbb{R}^n$ either as a
row vector ($1\times n$ matrix) or a column vector ($n \times 1$
matrix). Because of this equivalence between vectors and matrices, there
is no need for a special vector object in \texttt{SymPy}, and
\texttt{Matrix} objects are used for vectors as well.

This is how we define vectors and compute their properties:

    \begin{Verbatim}[commandchars=\\\{\}]
{\color{incolor}In [{\color{incolor}137}]:} \PY{n}{u} \PY{o}{=} \PY{n}{Matrix}\PY{p}{(}\PY{p}{[}\PY{p}{[}\PY{l+m+mi}{4}\PY{p}{,}\PY{l+m+mi}{5}\PY{p}{,}\PY{l+m+mi}{6}\PY{p}{]}\PY{p}{]}\PY{p}{)}  \PY{c}{# a row vector = 1x3 matrix}
          \PY{n}{v} \PY{o}{=} \PY{n}{Matrix}\PY{p}{(}\PY{p}{[}\PY{p}{[}\PY{l+m+mi}{7}\PY{p}{]}\PY{p}{,}
                      \PY{p}{[}\PY{l+m+mi}{8}\PY{p}{]}\PY{p}{,}       \PY{c}{# a col vector = 3x1 matrix }
                      \PY{p}{[}\PY{l+m+mi}{9}\PY{p}{]}\PY{p}{]}\PY{p}{)}
\end{Verbatim}

    \begin{Verbatim}[commandchars=\\\{\}]
{\color{incolor}In [{\color{incolor}138}]:} \PY{n}{v}\PY{o}{.}\PY{n}{T}                    \PY{c}{# use the transpose operation to convert a col vec to a row vec}
\end{Verbatim}
\texttt{\color{outcolor}Out[{\color{outcolor}138}]:}
    
    
        \begin{equation*}\adjustbox{max width=\hsize}{$
        \left[\begin{matrix}7 & 8 & 9\end{matrix}\right]
        $}\end{equation*}

    

    \begin{Verbatim}[commandchars=\\\{\}]
{\color{incolor}In [{\color{incolor}139}]:} \PY{n}{u}\PY{p}{[}\PY{l+m+mi}{0}\PY{p}{]}                   \PY{c}{# 0-based indexing for entries}
\end{Verbatim}
\texttt{\color{outcolor}Out[{\color{outcolor}139}]:}
    
    
        \begin{equation*}\adjustbox{max width=\hsize}{$
        4
        $}\end{equation*}

    

    \begin{Verbatim}[commandchars=\\\{\}]
{\color{incolor}In [{\color{incolor}140}]:} \PY{n}{u}\PY{o}{.}\PY{n}{norm}\PY{p}{(}\PY{p}{)}               \PY{c}{# length of u}
\end{Verbatim}
\texttt{\color{outcolor}Out[{\color{outcolor}140}]:}
    
    
        \begin{equation*}\adjustbox{max width=\hsize}{$
        \sqrt{77}
        $}\end{equation*}

    

    \begin{Verbatim}[commandchars=\\\{\}]
{\color{incolor}In [{\color{incolor}141}]:} \PY{n}{uhat} \PY{o}{=} \PY{n}{u}\PY{o}{/}\PY{n}{u}\PY{o}{.}\PY{n}{norm}\PY{p}{(}\PY{p}{)}      \PY{c}{# unit-length vec in same dir as u}
          \PY{n}{uhat}
\end{Verbatim}
\texttt{\color{outcolor}Out[{\color{outcolor}141}]:}
    
    
        \begin{equation*}\adjustbox{max width=\hsize}{$
        \left[\begin{matrix}\frac{4 \sqrt{77}}{77} & \frac{5 \sqrt{77}}{77} & \frac{6 \sqrt{77}}{77}\end{matrix}\right]
        $}\end{equation*}

    

    \begin{Verbatim}[commandchars=\\\{\}]
{\color{incolor}In [{\color{incolor}142}]:} \PY{n}{uhat}\PY{o}{.}\PY{n}{norm}\PY{p}{(}\PY{p}{)}
\end{Verbatim}
\texttt{\color{outcolor}Out[{\color{outcolor}142}]:}
    
    
        \begin{equation*}\adjustbox{max width=\hsize}{$
        1
        $}\end{equation*}

    

    \subsubsection{Dot product}\label{dot-product}

    The dot product of the 3-vectors $\vec{u}$ and $\vec{v}$ can be defined
two ways:

\[
  \vec{u}\cdot\vec{v}
    \equiv 
    \underbrace{u_xv_x+u_yv_y+u_zv_z}_{\textrm{algebraic def.}} 
    \equiv 
    \underbrace{\|\vec{u}\|\|\vec{v}\|\cos(\varphi)}_{\textrm{geometric def.}} 
    \quad \in \mathbb{R},
\]

where $\varphi$ is the angle between the vectors $\vec{u}$ and
$\vec{v}$. In \texttt{SymPy},

    \begin{Verbatim}[commandchars=\\\{\}]
{\color{incolor}In [{\color{incolor}143}]:} \PY{n}{u} \PY{o}{=} \PY{n}{Matrix}\PY{p}{(}\PY{p}{[} \PY{l+m+mi}{4}\PY{p}{,}\PY{l+m+mi}{5}\PY{p}{,}\PY{l+m+mi}{6}\PY{p}{]}\PY{p}{)}
          \PY{n}{v} \PY{o}{=} \PY{n}{Matrix}\PY{p}{(}\PY{p}{[}\PY{o}{-}\PY{l+m+mi}{1}\PY{p}{,}\PY{l+m+mi}{1}\PY{p}{,}\PY{l+m+mi}{2}\PY{p}{]}\PY{p}{)}
          \PY{n}{u}\PY{o}{.}\PY{n}{dot}\PY{p}{(}\PY{n}{v}\PY{p}{)}
\end{Verbatim}
\texttt{\color{outcolor}Out[{\color{outcolor}143}]:}
    
    
        \begin{equation*}\adjustbox{max width=\hsize}{$
        13
        $}\end{equation*}

    

    We can combine the algebraic and geometric formulas for the dot product
to obtain the cosine of the angle between the vectors

\[
    \cos(\varphi)
        = \frac{ \vec{u}\cdot\vec{v} }{  \|\vec{u}\|\|\vec{v}\| }
        = \frac{ u_xv_x+u_yv_y+u_zv_z  }{  \|\vec{u}\|\|\vec{v}\| },
\]

and use the \texttt{acos} function to find the angle measure:

    \begin{Verbatim}[commandchars=\\\{\}]
{\color{incolor}In [{\color{incolor}144}]:} \PY{n}{acos}\PY{p}{(}\PY{n}{u}\PY{o}{.}\PY{n}{dot}\PY{p}{(}\PY{n}{v}\PY{p}{)}\PY{o}{/}\PY{p}{(}\PY{n}{u}\PY{o}{.}\PY{n}{norm}\PY{p}{(}\PY{p}{)}\PY{o}{*}\PY{n}{v}\PY{o}{.}\PY{n}{norm}\PY{p}{(}\PY{p}{)}\PY{p}{)}\PY{p}{)}\PY{o}{.}\PY{n}{evalf}\PY{p}{(}\PY{p}{)}  \PY{c}{# in radians = 52.76 degrees}
\end{Verbatim}
\texttt{\color{outcolor}Out[{\color{outcolor}144}]:}
    
    
        \begin{equation*}\adjustbox{max width=\hsize}{$
        0.921263115666387
        $}\end{equation*}

    

    Just by looking at the coordinates of the vectors $\vec{u}$ and
$\vec{v}$, it's difficult to determine their relative direction. Thanks
to the dot product, however, we know the angle between the vectors is
$52.76^\circ$, which means they \emph{kind of} point in the same
direction. Vectors that are at an angle $\varphi=90^\circ$ are called
\emph{orthogonal}, meaning at right angles with each other. The dot
product of vectors for which $\varphi > 90^\circ$ is negative because
they point \emph{mostly} in opposite directions.

The notion of the ``angle between vectors'' applies more generally to
vectors with any number of dimensions. The dot product for
$n$-dimensional vectors is $\vec{u}\cdot\vec{v}=\sum_{i=1}^n u_iv_i$.
This means we can talk about ``the angle between'' 1000-dimensional
vectors. That's pretty crazy if you think about it---there is no way we
could possibly ``visualize'' 1000-dimensional vectors, yet given two
such vectors we can tell if they point mostly in the same direction, in
perpendicular directions, or mostly in opposite directions.

The dot product is a commutative operation
$\vec{u}\cdot\vec{v} = \vec{v}\cdot\vec{u}$:

    \begin{Verbatim}[commandchars=\\\{\}]
{\color{incolor}In [{\color{incolor}145}]:} \PY{n}{u}\PY{o}{.}\PY{n}{dot}\PY{p}{(}\PY{n}{v}\PY{p}{)} \PY{o}{==} \PY{n}{v}\PY{o}{.}\PY{n}{dot}\PY{p}{(}\PY{n}{u}\PY{p}{)}
\end{Verbatim}

            \begin{Verbatim}[commandchars=\\\{\}]
{\color{outcolor}Out[{\color{outcolor}145}]:} True
\end{Verbatim}
        
    \subsubsection{Projections}\label{projections}

    Dot products are used for computing projections. Assume you're given two
vectors $\vec{u}$ and $\vec{n}$ and you want to find the component of
$\vec{u}$ that points in the $\vec{n}$ direction. The following formula
based on the dot product will give you the answer:

\[
 \Pi_{\vec{n}}( \vec{u} ) \equiv \frac{  \vec{u} \cdot \vec{n}  }{ \| \vec{n} \|^2 } \vec{n}.
\]

This is how to implement this formula in \texttt{SymPy}:

    \begin{Verbatim}[commandchars=\\\{\}]
{\color{incolor}In [{\color{incolor}146}]:} \PY{n}{u} \PY{o}{=} \PY{n}{Matrix}\PY{p}{(}\PY{p}{[}\PY{l+m+mi}{4}\PY{p}{,}\PY{l+m+mi}{5}\PY{p}{,}\PY{l+m+mi}{6}\PY{p}{]}\PY{p}{)}
          \PY{n}{n} \PY{o}{=} \PY{n}{Matrix}\PY{p}{(}\PY{p}{[}\PY{l+m+mi}{1}\PY{p}{,}\PY{l+m+mi}{1}\PY{p}{,}\PY{l+m+mi}{1}\PY{p}{]}\PY{p}{)}
          \PY{p}{(}\PY{n}{u}\PY{o}{.}\PY{n}{dot}\PY{p}{(}\PY{n}{n}\PY{p}{)} \PY{o}{/} \PY{n}{n}\PY{o}{.}\PY{n}{norm}\PY{p}{(}\PY{p}{)}\PY{o}{*}\PY{o}{*}\PY{l+m+mi}{2}\PY{p}{)}\PY{o}{*}\PY{n}{n}  \PY{c}{# projection of v in the n dir}
\end{Verbatim}
\texttt{\color{outcolor}Out[{\color{outcolor}146}]:}
    
    
        \begin{equation*}\adjustbox{max width=\hsize}{$
        \left[\begin{matrix}5\\5\\5\end{matrix}\right]
        $}\end{equation*}

    

    In the case where the direction vector $\hat{n}$ is of unit length
$\|\hat{n}\| = 1$, the projection formula simplifies to
$\Pi_{\hat{n}}( \vec{u} ) \equiv (\vec{u}\cdot\hat{n})\hat{n}$.

Consider now the plane $P$ defined by $(1,1,1)\cdot[(x,y,z)-(0,0,0)]=0$.
A plane is a two dimensional subspace of $\mathbb{R}^3$. We can
decompose any vector $\vec{u} \in \mathbb{R}^3$ into two parts
$\vec{u}=\vec{v} + \vec{w}$ such that $\vec{v}$ lies inside the plane
and $\vec{w}$ is perpendicular to the plane (parallel to
$\vec{n}=(1,1,1)$).

To obtain the perpendicular-to-$P$ component of $\vec{u}$, compute the
projection of $\vec{u}$ in the direction $\vec{n}$:

    \begin{Verbatim}[commandchars=\\\{\}]
{\color{incolor}In [{\color{incolor}147}]:} \PY{n}{w} \PY{o}{=} \PY{p}{(}\PY{n}{u}\PY{o}{.}\PY{n}{dot}\PY{p}{(}\PY{n}{n}\PY{p}{)} \PY{o}{/} \PY{n}{n}\PY{o}{.}\PY{n}{norm}\PY{p}{(}\PY{p}{)}\PY{o}{*}\PY{o}{*}\PY{l+m+mi}{2}\PY{p}{)}\PY{o}{*}\PY{n}{n}
          \PY{n}{w}
\end{Verbatim}
\texttt{\color{outcolor}Out[{\color{outcolor}147}]:}
    
    
        \begin{equation*}\adjustbox{max width=\hsize}{$
        \left[\begin{matrix}5\\5\\5\end{matrix}\right]
        $}\end{equation*}

    

    To obtain the in-the-plane-$P$ component of $\vec{u}$, start with
$\vec{u}$ and subtract the perpendicular-to-$P$ part:

    \begin{Verbatim}[commandchars=\\\{\}]
{\color{incolor}In [{\color{incolor}148}]:} \PY{n}{v} \PY{o}{=} \PY{n}{u} \PY{o}{-} \PY{p}{(}\PY{n}{u}\PY{o}{.}\PY{n}{dot}\PY{p}{(}\PY{n}{n}\PY{p}{)}\PY{o}{/}\PY{n}{n}\PY{o}{.}\PY{n}{norm}\PY{p}{(}\PY{p}{)}\PY{o}{*}\PY{o}{*}\PY{l+m+mi}{2}\PY{p}{)}\PY{o}{*}\PY{n}{n}  \PY{c}{# same as u - w}
          \PY{n}{v}
\end{Verbatim}
\texttt{\color{outcolor}Out[{\color{outcolor}148}]:}
    
    
        \begin{equation*}\adjustbox{max width=\hsize}{$
        \left[\begin{matrix}-1\\0\\1\end{matrix}\right]
        $}\end{equation*}

    

    You should check on your own that $\vec{v}+\vec{w}=\vec{u}$ as claimed.

    \subsubsection{Cross product}\label{cross-product}

    The \emph{cross product}, denoted $\times$, takes two vectors as inputs
and produces a vector as output. The cross products of individual basis
elements are defined as follows:

\[
 \hat{\imath}\times\hat{\jmath} =\hat{k}, \qquad
 \hat{\jmath}\times\hat{k} =\hat{\imath}, \qquad
 \hat{k}\times \hat{\imath}= \hat{\jmath}.
\]

Here is how to compute the cross product of two vectors in
\texttt{SymPy}:

    \begin{Verbatim}[commandchars=\\\{\}]
{\color{incolor}In [{\color{incolor}149}]:} \PY{n}{u} \PY{o}{=} \PY{n}{Matrix}\PY{p}{(}\PY{p}{[} \PY{l+m+mi}{4}\PY{p}{,}\PY{l+m+mi}{5}\PY{p}{,}\PY{l+m+mi}{6}\PY{p}{]}\PY{p}{)}
          \PY{n}{v} \PY{o}{=} \PY{n}{Matrix}\PY{p}{(}\PY{p}{[}\PY{o}{-}\PY{l+m+mi}{1}\PY{p}{,}\PY{l+m+mi}{1}\PY{p}{,}\PY{l+m+mi}{2}\PY{p}{]}\PY{p}{)}
          \PY{n}{u}\PY{o}{.}\PY{n}{cross}\PY{p}{(}\PY{n}{v}\PY{p}{)}
\end{Verbatim}
\texttt{\color{outcolor}Out[{\color{outcolor}149}]:}
    
    
        \begin{equation*}\adjustbox{max width=\hsize}{$
        \left[\begin{matrix}4\\-14\\9\end{matrix}\right]
        $}\end{equation*}

    

    The vector $\vec{u}\times \vec{v}$ is orthogonal to both $\vec{u}$ and
$\vec{v}$. The norm of the cross product $\|\vec{u}\times \vec{v}\|$ is
proportional to the lengths of the vectors and the sine of the angle
between them:

    \begin{Verbatim}[commandchars=\\\{\}]
{\color{incolor}In [{\color{incolor}150}]:} \PY{p}{(}\PY{n}{u}\PY{o}{.}\PY{n}{cross}\PY{p}{(}\PY{n}{v}\PY{p}{)}\PY{o}{.}\PY{n}{norm}\PY{p}{(}\PY{p}{)}\PY{o}{/}\PY{p}{(}\PY{n}{u}\PY{o}{.}\PY{n}{norm}\PY{p}{(}\PY{p}{)}\PY{o}{*}\PY{n}{v}\PY{o}{.}\PY{n}{norm}\PY{p}{(}\PY{p}{)}\PY{p}{)}\PY{p}{)}\PY{o}{.}\PY{n}{n}\PY{p}{(}\PY{p}{)}
\end{Verbatim}
\texttt{\color{outcolor}Out[{\color{outcolor}150}]:}
    
    
        \begin{equation*}\adjustbox{max width=\hsize}{$
        0.796366206088088
        $}\end{equation*}

    

    The name ``cross product'' is well-suited for this operation since it is
calculated by ``cross-multiplying'' the coefficients of the vectors:

\[
   \vec{u}\times\vec{v}=
   \left( 
     u_yv_z-u_zv_y, \ u_zv_x-u_xv_z, \ u_xv_y-u_yv_x 
    \right).
\]

By defining individual symbols for the entries of two vectors, we can
make \texttt{SymPy} show us the cross-product formula:

    \begin{Verbatim}[commandchars=\\\{\}]
{\color{incolor}In [{\color{incolor}151}]:} \PY{n}{u1}\PY{p}{,}\PY{n}{u2}\PY{p}{,}\PY{n}{u3} \PY{o}{=} \PY{n}{symbols}\PY{p}{(}\PY{l+s}{'}\PY{l+s}{u1:4}\PY{l+s}{'}\PY{p}{)}
          \PY{n}{v1}\PY{p}{,}\PY{n}{v2}\PY{p}{,}\PY{n}{v3} \PY{o}{=} \PY{n}{symbols}\PY{p}{(}\PY{l+s}{'}\PY{l+s}{v1:4}\PY{l+s}{'}\PY{p}{)}
          \PY{n}{Matrix}\PY{p}{(}\PY{p}{[}\PY{n}{u1}\PY{p}{,}\PY{n}{u2}\PY{p}{,}\PY{n}{u3}\PY{p}{]}\PY{p}{)}\PY{o}{.}\PY{n}{cross}\PY{p}{(}\PY{n}{Matrix}\PY{p}{(}\PY{p}{[}\PY{n}{v1}\PY{p}{,}\PY{n}{v2}\PY{p}{,}\PY{n}{v3}\PY{p}{]}\PY{p}{)}\PY{p}{)}
\end{Verbatim}
\texttt{\color{outcolor}Out[{\color{outcolor}151}]:}
    
    
        \begin{equation*}\adjustbox{max width=\hsize}{$
        \left[\begin{matrix}u_{2} v_{3} - u_{3} v_{2}\\- u_{1} v_{3} + u_{3} v_{1}\\u_{1} v_{2} - u_{2} v_{1}\end{matrix}\right]
        $}\end{equation*}

    

    The dot product is anti-commutative
$\vec{u}\times\vec{v} = -\vec{v}\times\vec{u}$:

    \begin{Verbatim}[commandchars=\\\{\}]
{\color{incolor}In [{\color{incolor}152}]:} \PY{n}{u}\PY{o}{.}\PY{n}{cross}\PY{p}{(}\PY{n}{v}\PY{p}{)}
\end{Verbatim}
\texttt{\color{outcolor}Out[{\color{outcolor}152}]:}
    
    
        \begin{equation*}\adjustbox{max width=\hsize}{$
        \left[\begin{matrix}4\\-14\\9\end{matrix}\right]
        $}\end{equation*}

    

    \begin{Verbatim}[commandchars=\\\{\}]
{\color{incolor}In [{\color{incolor}153}]:} \PY{n}{v}\PY{o}{.}\PY{n}{cross}\PY{p}{(}\PY{n}{u}\PY{p}{)}
\end{Verbatim}
\texttt{\color{outcolor}Out[{\color{outcolor}153}]:}
    
    
        \begin{equation*}\adjustbox{max width=\hsize}{$
        \left[\begin{matrix}-4\\14\\-9\end{matrix}\right]
        $}\end{equation*}

    

    The product of two numbers and the dot product of two vectors are
commutative operations. The cross product, however, is not commutative:
$\vec{u}\times\vec{v} \neq \vec{v}\times\vec{u}$.


    % Add a bibliography block to the postdoc
    
    
    
    \end{document}
