
% Default to the notebook output style

    


% Inherit from the specified cell style.




    
\documentclass{article}

    
    
    \usepackage{graphicx} % Used to insert images
    \usepackage{adjustbox} % Used to constrain images to a maximum size 
    \usepackage{color} % Allow colors to be defined
    \usepackage{enumerate} % Needed for markdown enumerations to work
    \usepackage{geometry} % Used to adjust the document margins
    \usepackage{amsmath} % Equations
    \usepackage{amssymb} % Equations
    \usepackage{eurosym} % defines \euro
    \usepackage[mathletters]{ucs} % Extended unicode (utf-8) support
    \usepackage[utf8x]{inputenc} % Allow utf-8 characters in the tex document
    \usepackage{fancyvrb} % verbatim replacement that allows latex
    \usepackage{grffile} % extends the file name processing of package graphics 
                         % to support a larger range 
    % The hyperref package gives us a pdf with properly built
    % internal navigation ('pdf bookmarks' for the table of contents,
    % internal cross-reference links, web links for URLs, etc.)
    \usepackage{hyperref}
    \usepackage{longtable} % longtable support required by pandoc >1.10
    \usepackage{booktabs}  % table support for pandoc > 1.12.2
    

    
    
    \definecolor{orange}{cmyk}{0,0.4,0.8,0.2}
    \definecolor{darkorange}{rgb}{.71,0.21,0.01}
    \definecolor{darkgreen}{rgb}{.12,.54,.11}
    \definecolor{myteal}{rgb}{.26, .44, .56}
    \definecolor{gray}{gray}{0.45}
    \definecolor{lightgray}{gray}{.95}
    \definecolor{mediumgray}{gray}{.8}
    \definecolor{inputbackground}{rgb}{.95, .95, .85}
    \definecolor{outputbackground}{rgb}{.95, .95, .95}
    \definecolor{traceback}{rgb}{1, .95, .95}
    % ansi colors
    \definecolor{red}{rgb}{.6,0,0}
    \definecolor{green}{rgb}{0,.65,0}
    \definecolor{brown}{rgb}{0.6,0.6,0}
    \definecolor{blue}{rgb}{0,.145,.698}
    \definecolor{purple}{rgb}{.698,.145,.698}
    \definecolor{cyan}{rgb}{0,.698,.698}
    \definecolor{lightgray}{gray}{0.5}
    
    % bright ansi colors
    \definecolor{darkgray}{gray}{0.25}
    \definecolor{lightred}{rgb}{1.0,0.39,0.28}
    \definecolor{lightgreen}{rgb}{0.48,0.99,0.0}
    \definecolor{lightblue}{rgb}{0.53,0.81,0.92}
    \definecolor{lightpurple}{rgb}{0.87,0.63,0.87}
    \definecolor{lightcyan}{rgb}{0.5,1.0,0.83}
    
    % commands and environments needed by pandoc snippets
    % extracted from the output of `pandoc -s`
    \DefineVerbatimEnvironment{Highlighting}{Verbatim}{commandchars=\\\{\}}
    % Add ',fontsize=\small' for more characters per line
    \newenvironment{Shaded}{}{}
    \newcommand{\KeywordTok}[1]{\textcolor[rgb]{0.00,0.44,0.13}{\textbf{{#1}}}}
    \newcommand{\DataTypeTok}[1]{\textcolor[rgb]{0.56,0.13,0.00}{{#1}}}
    \newcommand{\DecValTok}[1]{\textcolor[rgb]{0.25,0.63,0.44}{{#1}}}
    \newcommand{\BaseNTok}[1]{\textcolor[rgb]{0.25,0.63,0.44}{{#1}}}
    \newcommand{\FloatTok}[1]{\textcolor[rgb]{0.25,0.63,0.44}{{#1}}}
    \newcommand{\CharTok}[1]{\textcolor[rgb]{0.25,0.44,0.63}{{#1}}}
    \newcommand{\StringTok}[1]{\textcolor[rgb]{0.25,0.44,0.63}{{#1}}}
    \newcommand{\CommentTok}[1]{\textcolor[rgb]{0.38,0.63,0.69}{\textit{{#1}}}}
    \newcommand{\OtherTok}[1]{\textcolor[rgb]{0.00,0.44,0.13}{{#1}}}
    \newcommand{\AlertTok}[1]{\textcolor[rgb]{1.00,0.00,0.00}{\textbf{{#1}}}}
    \newcommand{\FunctionTok}[1]{\textcolor[rgb]{0.02,0.16,0.49}{{#1}}}
    \newcommand{\RegionMarkerTok}[1]{{#1}}
    \newcommand{\ErrorTok}[1]{\textcolor[rgb]{1.00,0.00,0.00}{\textbf{{#1}}}}
    \newcommand{\NormalTok}[1]{{#1}}
    
    % Define a nice break command that doesn't care if a line doesn't already
    % exist.
    \def\br{\hspace*{\fill} \\* }
    % Math Jax compatability definitions
    \def\gt{>}
    \def\lt{<}
    % Document parameters
    \title{Linear-algebra}
    
    
    

    % Pygments definitions
    
\makeatletter
\def\PY@reset{\let\PY@it=\relax \let\PY@bf=\relax%
    \let\PY@ul=\relax \let\PY@tc=\relax%
    \let\PY@bc=\relax \let\PY@ff=\relax}
\def\PY@tok#1{\csname PY@tok@#1\endcsname}
\def\PY@toks#1+{\ifx\relax#1\empty\else%
    \PY@tok{#1}\expandafter\PY@toks\fi}
\def\PY@do#1{\PY@bc{\PY@tc{\PY@ul{%
    \PY@it{\PY@bf{\PY@ff{#1}}}}}}}
\def\PY#1#2{\PY@reset\PY@toks#1+\relax+\PY@do{#2}}

\def\PY@tok@gd{\def\PY@tc##1{\textcolor[rgb]{0.63,0.00,0.00}{##1}}}
\def\PY@tok@gu{\let\PY@bf=\textbf\def\PY@tc##1{\textcolor[rgb]{0.50,0.00,0.50}{##1}}}
\def\PY@tok@gt{\def\PY@tc##1{\textcolor[rgb]{0.00,0.25,0.82}{##1}}}
\def\PY@tok@gs{\let\PY@bf=\textbf}
\def\PY@tok@gr{\def\PY@tc##1{\textcolor[rgb]{1.00,0.00,0.00}{##1}}}
\def\PY@tok@cm{\let\PY@it=\textit\def\PY@tc##1{\textcolor[rgb]{0.25,0.50,0.50}{##1}}}
\def\PY@tok@vg{\def\PY@tc##1{\textcolor[rgb]{0.10,0.09,0.49}{##1}}}
\def\PY@tok@m{\def\PY@tc##1{\textcolor[rgb]{0.40,0.40,0.40}{##1}}}
\def\PY@tok@mh{\def\PY@tc##1{\textcolor[rgb]{0.40,0.40,0.40}{##1}}}
\def\PY@tok@go{\def\PY@tc##1{\textcolor[rgb]{0.50,0.50,0.50}{##1}}}
\def\PY@tok@ge{\let\PY@it=\textit}
\def\PY@tok@vc{\def\PY@tc##1{\textcolor[rgb]{0.10,0.09,0.49}{##1}}}
\def\PY@tok@il{\def\PY@tc##1{\textcolor[rgb]{0.40,0.40,0.40}{##1}}}
\def\PY@tok@cs{\let\PY@it=\textit\def\PY@tc##1{\textcolor[rgb]{0.25,0.50,0.50}{##1}}}
\def\PY@tok@cp{\def\PY@tc##1{\textcolor[rgb]{0.74,0.48,0.00}{##1}}}
\def\PY@tok@gi{\def\PY@tc##1{\textcolor[rgb]{0.00,0.63,0.00}{##1}}}
\def\PY@tok@gh{\let\PY@bf=\textbf\def\PY@tc##1{\textcolor[rgb]{0.00,0.00,0.50}{##1}}}
\def\PY@tok@ni{\let\PY@bf=\textbf\def\PY@tc##1{\textcolor[rgb]{0.60,0.60,0.60}{##1}}}
\def\PY@tok@nl{\def\PY@tc##1{\textcolor[rgb]{0.63,0.63,0.00}{##1}}}
\def\PY@tok@nn{\let\PY@bf=\textbf\def\PY@tc##1{\textcolor[rgb]{0.00,0.00,1.00}{##1}}}
\def\PY@tok@no{\def\PY@tc##1{\textcolor[rgb]{0.53,0.00,0.00}{##1}}}
\def\PY@tok@na{\def\PY@tc##1{\textcolor[rgb]{0.49,0.56,0.16}{##1}}}
\def\PY@tok@nb{\def\PY@tc##1{\textcolor[rgb]{0.00,0.50,0.00}{##1}}}
\def\PY@tok@nc{\let\PY@bf=\textbf\def\PY@tc##1{\textcolor[rgb]{0.00,0.00,1.00}{##1}}}
\def\PY@tok@nd{\def\PY@tc##1{\textcolor[rgb]{0.67,0.13,1.00}{##1}}}
\def\PY@tok@ne{\let\PY@bf=\textbf\def\PY@tc##1{\textcolor[rgb]{0.82,0.25,0.23}{##1}}}
\def\PY@tok@nf{\def\PY@tc##1{\textcolor[rgb]{0.00,0.00,1.00}{##1}}}
\def\PY@tok@si{\let\PY@bf=\textbf\def\PY@tc##1{\textcolor[rgb]{0.73,0.40,0.53}{##1}}}
\def\PY@tok@s2{\def\PY@tc##1{\textcolor[rgb]{0.73,0.13,0.13}{##1}}}
\def\PY@tok@vi{\def\PY@tc##1{\textcolor[rgb]{0.10,0.09,0.49}{##1}}}
\def\PY@tok@nt{\let\PY@bf=\textbf\def\PY@tc##1{\textcolor[rgb]{0.00,0.50,0.00}{##1}}}
\def\PY@tok@nv{\def\PY@tc##1{\textcolor[rgb]{0.10,0.09,0.49}{##1}}}
\def\PY@tok@s1{\def\PY@tc##1{\textcolor[rgb]{0.73,0.13,0.13}{##1}}}
\def\PY@tok@sh{\def\PY@tc##1{\textcolor[rgb]{0.73,0.13,0.13}{##1}}}
\def\PY@tok@sc{\def\PY@tc##1{\textcolor[rgb]{0.73,0.13,0.13}{##1}}}
\def\PY@tok@sx{\def\PY@tc##1{\textcolor[rgb]{0.00,0.50,0.00}{##1}}}
\def\PY@tok@bp{\def\PY@tc##1{\textcolor[rgb]{0.00,0.50,0.00}{##1}}}
\def\PY@tok@c1{\let\PY@it=\textit\def\PY@tc##1{\textcolor[rgb]{0.25,0.50,0.50}{##1}}}
\def\PY@tok@kc{\let\PY@bf=\textbf\def\PY@tc##1{\textcolor[rgb]{0.00,0.50,0.00}{##1}}}
\def\PY@tok@c{\let\PY@it=\textit\def\PY@tc##1{\textcolor[rgb]{0.25,0.50,0.50}{##1}}}
\def\PY@tok@mf{\def\PY@tc##1{\textcolor[rgb]{0.40,0.40,0.40}{##1}}}
\def\PY@tok@err{\def\PY@bc##1{\fcolorbox[rgb]{1.00,0.00,0.00}{1,1,1}{##1}}}
\def\PY@tok@kd{\let\PY@bf=\textbf\def\PY@tc##1{\textcolor[rgb]{0.00,0.50,0.00}{##1}}}
\def\PY@tok@ss{\def\PY@tc##1{\textcolor[rgb]{0.10,0.09,0.49}{##1}}}
\def\PY@tok@sr{\def\PY@tc##1{\textcolor[rgb]{0.73,0.40,0.53}{##1}}}
\def\PY@tok@mo{\def\PY@tc##1{\textcolor[rgb]{0.40,0.40,0.40}{##1}}}
\def\PY@tok@kn{\let\PY@bf=\textbf\def\PY@tc##1{\textcolor[rgb]{0.00,0.50,0.00}{##1}}}
\def\PY@tok@mi{\def\PY@tc##1{\textcolor[rgb]{0.40,0.40,0.40}{##1}}}
\def\PY@tok@gp{\let\PY@bf=\textbf\def\PY@tc##1{\textcolor[rgb]{0.00,0.00,0.50}{##1}}}
\def\PY@tok@o{\def\PY@tc##1{\textcolor[rgb]{0.40,0.40,0.40}{##1}}}
\def\PY@tok@kr{\let\PY@bf=\textbf\def\PY@tc##1{\textcolor[rgb]{0.00,0.50,0.00}{##1}}}
\def\PY@tok@s{\def\PY@tc##1{\textcolor[rgb]{0.73,0.13,0.13}{##1}}}
\def\PY@tok@kp{\def\PY@tc##1{\textcolor[rgb]{0.00,0.50,0.00}{##1}}}
\def\PY@tok@w{\def\PY@tc##1{\textcolor[rgb]{0.73,0.73,0.73}{##1}}}
\def\PY@tok@kt{\def\PY@tc##1{\textcolor[rgb]{0.69,0.00,0.25}{##1}}}
\def\PY@tok@ow{\let\PY@bf=\textbf\def\PY@tc##1{\textcolor[rgb]{0.67,0.13,1.00}{##1}}}
\def\PY@tok@sb{\def\PY@tc##1{\textcolor[rgb]{0.73,0.13,0.13}{##1}}}
\def\PY@tok@k{\let\PY@bf=\textbf\def\PY@tc##1{\textcolor[rgb]{0.00,0.50,0.00}{##1}}}
\def\PY@tok@se{\let\PY@bf=\textbf\def\PY@tc##1{\textcolor[rgb]{0.73,0.40,0.13}{##1}}}
\def\PY@tok@sd{\let\PY@it=\textit\def\PY@tc##1{\textcolor[rgb]{0.73,0.13,0.13}{##1}}}

\def\PYZbs{\char`\\}
\def\PYZus{\char`\_}
\def\PYZob{\char`\{}
\def\PYZcb{\char`\}}
\def\PYZca{\char`\^}
% for compatibility with earlier versions
\def\PYZat{@}
\def\PYZlb{[}
\def\PYZrb{]}
\makeatother


    % Exact colors from NB
    \definecolor{incolor}{rgb}{0.0, 0.0, 0.5}
    \definecolor{outcolor}{rgb}{0.545, 0.0, 0.0}



    
    % Prevent overflowing lines due to hard-to-break entities
    \sloppy 
    % Setup hyperref package
    \hypersetup{
      breaklinks=true,  % so long urls are correctly broken across lines
      colorlinks=true,
      urlcolor=blue,
      linkcolor=darkorange,
      citecolor=darkgreen,
      }
    % Slightly bigger margins than the latex defaults
    
    \geometry{verbose,tmargin=1in,bmargin=1in,lmargin=1in,rmargin=1in}
    
    

    \begin{document}
    
    
    \maketitle
    
    

    
    \subsection{Linear algebra}\label{linear-algebra}

    A matrix $A \in \mathbb{R}^{m\times n}$ is a rectangular array of real
numbers with $m$ rows and $n$ columns. To specify a matrix $A$, we
specify the values for its $mn$ components
$a_{11}, a_{12}, \ldots, a_{mn}$ as a list of lists:

    \begin{Verbatim}[commandchars=\\\{\}]
{\color{incolor}In [{\color{incolor}176}]:} \PY{n}{A} \PY{o}{=} \PY{n}{Matrix}\PY{p}{(} \PY{p}{[}\PY{p}{[} \PY{l+m+mi}{2}\PY{p}{,}\PY{o}{-}\PY{l+m+mi}{3}\PY{p}{,}\PY{o}{-}\PY{l+m+mi}{8}\PY{p}{,} \PY{l+m+mi}{7}\PY{p}{]}\PY{p}{,}
                       \PY{p}{[}\PY{o}{-}\PY{l+m+mi}{2}\PY{p}{,}\PY{o}{-}\PY{l+m+mi}{1}\PY{p}{,} \PY{l+m+mi}{2}\PY{p}{,}\PY{o}{-}\PY{l+m+mi}{7}\PY{p}{]}\PY{p}{,}
                       \PY{p}{[} \PY{l+m+mi}{1}\PY{p}{,} \PY{l+m+mi}{0}\PY{p}{,}\PY{o}{-}\PY{l+m+mi}{3}\PY{p}{,} \PY{l+m+mi}{6}\PY{p}{]}\PY{p}{]} \PY{p}{)}
\end{Verbatim}

    Use the square brackets to access the matrix elements or to obtain a
submatrix:

    \begin{Verbatim}[commandchars=\\\{\}]
{\color{incolor}In [{\color{incolor}177}]:} \PY{n}{A}\PY{p}{[}\PY{l+m+mi}{0}\PY{p}{,}\PY{l+m+mi}{1}\PY{p}{]}         \PY{c}{# row 0, col 1 of A}
\end{Verbatim}
\texttt{\color{outcolor}Out[{\color{outcolor}177}]:}
    
    
        \begin{equation*}\adjustbox{max width=\hsize}{$
        -3
        $}\end{equation*}

    

    \begin{Verbatim}[commandchars=\\\{\}]
{\color{incolor}In [{\color{incolor}178}]:} \PY{n}{A}\PY{p}{[}\PY{l+m+mi}{0}\PY{p}{:}\PY{l+m+mi}{2}\PY{p}{,}\PY{l+m+mi}{0}\PY{p}{:}\PY{l+m+mi}{3}\PY{p}{]}     \PY{c}{# top-left 2x3 submatrix of A}
\end{Verbatim}
\texttt{\color{outcolor}Out[{\color{outcolor}178}]:}
    
    
        \begin{equation*}\adjustbox{max width=\hsize}{$
        \left[\begin{matrix}2 & -3 & -8\\-2 & -1 & 2\end{matrix}\right]
        $}\end{equation*}

    

    Some commonly used matrices can be created with shortcut methods:

    \begin{Verbatim}[commandchars=\\\{\}]
{\color{incolor}In [{\color{incolor}179}]:} \PY{n}{eye}\PY{p}{(}\PY{l+m+mi}{2}\PY{p}{)}         \PY{c}{# 2x2 identity matrix}
\end{Verbatim}
\texttt{\color{outcolor}Out[{\color{outcolor}179}]:}
    
    
        \begin{equation*}\adjustbox{max width=\hsize}{$
        \left[\begin{matrix}1 & 0\\0 & 1\end{matrix}\right]
        $}\end{equation*}

    

    \begin{Verbatim}[commandchars=\\\{\}]
{\color{incolor}In [{\color{incolor}180}]:} \PY{n}{zeros}\PY{p}{(}\PY{l+m+mi}{2}\PY{p}{,} \PY{l+m+mi}{3}\PY{p}{)}
\end{Verbatim}
\texttt{\color{outcolor}Out[{\color{outcolor}180}]:}
    
    
        \begin{equation*}\adjustbox{max width=\hsize}{$
        \left[\begin{matrix}0 & 0 & 0\\0 & 0 & 0\end{matrix}\right]
        $}\end{equation*}

    

    Standard algebraic operations like addition \texttt{+}, subtraction
\texttt{-}, multiplication \texttt{*}, and exponentiation \texttt{**}
work as expected for \texttt{Matrix} objects. The \texttt{transpose}
operation flips the matrix through its diagonal:

    \begin{Verbatim}[commandchars=\\\{\}]
{\color{incolor}In [{\color{incolor}181}]:} \PY{n}{A}\PY{o}{.}\PY{n}{transpose}\PY{p}{(}\PY{p}{)}  \PY{c}{# the same as A.T}
\end{Verbatim}
\texttt{\color{outcolor}Out[{\color{outcolor}181}]:}
    
    
        \begin{equation*}\adjustbox{max width=\hsize}{$
        \left[\begin{matrix}2 & -2 & 1\\-3 & -1 & 0\\-8 & 2 & -3\\7 & -7 & 6\end{matrix}\right]
        $}\end{equation*}

    

    Recall that the transpose is also used to convert row vectors into
column vectors and vice versa.

    \subsubsection{Row operations}\label{row-operations}

    \begin{Verbatim}[commandchars=\\\{\}]
{\color{incolor}In [{\color{incolor}182}]:} \PY{n}{M} \PY{o}{=} \PY{n}{eye}\PY{p}{(}\PY{l+m+mi}{3}\PY{p}{)}
          \PY{n}{M}\PY{o}{.}\PY{n}{row\PYZus{}op}\PY{p}{(}\PY{l+m+mi}{1}\PY{p}{,} \PY{k}{lambda} \PY{n}{v}\PY{p}{,}\PY{n}{j}\PY{p}{:} \PY{n}{v}\PY{o}{+}\PY{l+m+mi}{3}\PY{o}{*}\PY{n}{M}\PY{p}{[}\PY{l+m+mi}{0}\PY{p}{,}\PY{n}{j}\PY{p}{]} \PY{p}{)}
          \PY{n}{M}
\end{Verbatim}
\texttt{\color{outcolor}Out[{\color{outcolor}182}]:}
    
    
        \begin{equation*}\adjustbox{max width=\hsize}{$
        \left[\begin{matrix}1 & 0 & 0\\3 & 1 & 0\\0 & 0 & 1\end{matrix}\right]
        $}\end{equation*}

    

    The method \texttt{row\_op} takes two arguments as inputs: the first
argument specifies the 0-based index of the row you want to act on,
while the second argument is a function of the form \texttt{f(val,j)}
that describes how you want the value \texttt{val=M{[}i,j{]}} to be
transformed. The above call to \texttt{row\_op} implements the row
operation $R_2 \gets R_2 + 3R_1$.

    \subsubsection{Reduced row echelon form}\label{reduced-row-echelon-form}

    The Gauss---Jordan elimination procedure is a sequence of row operations
you can perform on any matrix to bring it to its \emph{reduced row
echelon form} (RREF). In \texttt{SymPy}, matrices have a \texttt{rref}
method that computes their RREF:

    \begin{Verbatim}[commandchars=\\\{\}]
{\color{incolor}In [{\color{incolor}183}]:} \PY{n}{A} \PY{o}{=} \PY{n}{Matrix}\PY{p}{(} \PY{p}{[}\PY{p}{[}\PY{l+m+mi}{2}\PY{p}{,}\PY{o}{-}\PY{l+m+mi}{3}\PY{p}{,}\PY{o}{-}\PY{l+m+mi}{8}\PY{p}{,} \PY{l+m+mi}{7}\PY{p}{]}\PY{p}{,}
                       \PY{p}{[}\PY{o}{-}\PY{l+m+mi}{2}\PY{p}{,}\PY{o}{-}\PY{l+m+mi}{1}\PY{p}{,}\PY{l+m+mi}{2}\PY{p}{,}\PY{o}{-}\PY{l+m+mi}{7}\PY{p}{]}\PY{p}{,}
                       \PY{p}{[}\PY{l+m+mi}{1}\PY{p}{,} \PY{l+m+mi}{0}\PY{p}{,}\PY{o}{-}\PY{l+m+mi}{3}\PY{p}{,} \PY{l+m+mi}{6}\PY{p}{]}\PY{p}{]}\PY{p}{)}
          \PY{n}{A}\PY{o}{.}\PY{n}{rref}\PY{p}{(}\PY{p}{)}  \PY{c}{# RREF of A, location of pivots}
\end{Verbatim}
\texttt{\color{outcolor}Out[{\color{outcolor}183}]:}
    
    
        \begin{equation*}\adjustbox{max width=\hsize}{$
        \left ( \left[\begin{matrix}1 & 0 & 0 & 0\\0 & 1 & 0 & 3\\0 & 0 & 1 & -2\end{matrix}\right], \quad \left [ 0, \quad 1, \quad 2\right ]\right )
        $}\end{equation*}

    

    Note the \texttt{rref} method returns a tuple of values: the first value
is the RREF of $A$, while the second tells you the indices of the
leading ones (also known as pivots) in the RREF of $A$. To get just the
RREF of $A$, select the $0^\mathrm{th}$ entry form the tuple:
\texttt{A.rref(){[}0{]}}.

    \subsubsection{Matrix fundamental
spaces}\label{matrix-fundamental-spaces}

    Consider the matrix $A \in \mathbb{R}^{m\times n}$. The fundamental
spaces of a matrix are its column space $\mathcal{C}(A)$, its null space
$\mathcal{N}(A)$, and its row space $\mathcal{R}(A)$. These vector
spaces are important when you consider the matrix product
$A\vec{x}=\vec{y}$ as ``applying'' the linear transformation
$T_A:\mathbb{R}^n \to \mathbb{R}^m$ to an input vector
$\vec{x} \in \mathbb{R}^n$ to produce the output vector
$\vec{y} \in \mathbb{R}^m$.

\textbf{Linear transformations} $T_A:\mathbb{R}^n \to \mathbb{R}^m$
(vector functions) \textbf{are equivalent to $m\times n$ matrices}. This
is one of the fundamental ideas in linear algebra. You can think of
$T_A$ as the abstract description of the transformation and
$A \in \mathbb{R}^{m\times n}$ as a concrete implementation of $T_A$. By
this equivalence, the fundamental spaces of a matrix $A$ tell us facts
about the domain and image of the linear transformation $T_A$. The
columns space $\mathcal{C}(A)$ is the same as the image space space
$\textrm{Im}(T_A)$ (the set of all possible outputs). The null space
$\mathcal{N}(A)$ is the same as the kernel $\textrm{Ker}(T_A)$ (the set
of inputs that $T_A$ maps to the zero vector). The row space
$\mathcal{R}(A)$ is the orthogonal complement of the null space. Input
vectors in the row space of $A$ are in one-to-one correspondence with
the output vectors in the column space of $A$.

Okay, enough theory! Let's see how to compute the fundamental spaces of
the matrix $A$ defined above. The non-zero rows in the reduced row
echelon form of $A$ are a basis for its row space:

    \begin{Verbatim}[commandchars=\\\{\}]
{\color{incolor}In [{\color{incolor}184}]:} \PY{p}{[} \PY{n}{A}\PY{o}{.}\PY{n}{rref}\PY{p}{(}\PY{p}{)}\PY{p}{[}\PY{l+m+mi}{0}\PY{p}{]}\PY{p}{[}\PY{n}{r}\PY{p}{,}\PY{p}{:}\PY{p}{]} \PY{k}{for} \PY{n}{r} \PY{o+ow}{in} \PY{n}{A}\PY{o}{.}\PY{n}{rref}\PY{p}{(}\PY{p}{)}\PY{p}{[}\PY{l+m+mi}{1}\PY{p}{]} \PY{p}{]}  \PY{c}{# R(A)}
\end{Verbatim}
\texttt{\color{outcolor}Out[{\color{outcolor}184}]:}
    
    
        \begin{equation*}\adjustbox{max width=\hsize}{$
        \left [ \left[\begin{matrix}1 & 0 & 0 & 0\end{matrix}\right], \quad \left[\begin{matrix}0 & 1 & 0 & 3\end{matrix}\right], \quad \left[\begin{matrix}0 & 0 & 1 & -2\end{matrix}\right]\right ]
        $}\end{equation*}

    

    The column space of $A$ is the span of the columns of $A$ that contain
the pivots in the reduced row echelon form of $A$:

    \begin{Verbatim}[commandchars=\\\{\}]
{\color{incolor}In [{\color{incolor}185}]:} \PY{p}{[} \PY{n}{A}\PY{p}{[}\PY{p}{:}\PY{p}{,}\PY{n}{c}\PY{p}{]} \PY{k}{for} \PY{n}{c} \PY{o+ow}{in}  \PY{n}{A}\PY{o}{.}\PY{n}{rref}\PY{p}{(}\PY{p}{)}\PY{p}{[}\PY{l+m+mi}{1}\PY{p}{]} \PY{p}{]}           \PY{c}{# C(A)}
\end{Verbatim}
\texttt{\color{outcolor}Out[{\color{outcolor}185}]:}
    
    
        \begin{equation*}\adjustbox{max width=\hsize}{$
        \left [ \left[\begin{matrix}2\\-2\\1\end{matrix}\right], \quad \left[\begin{matrix}-3\\-1\\0\end{matrix}\right], \quad \left[\begin{matrix}-8\\2\\-3\end{matrix}\right]\right ]
        $}\end{equation*}

    

    Note we took columns from the original matrix $A$ and not its RREF.

To find the null space of $A$, call its \texttt{nullspace} method:

    \begin{Verbatim}[commandchars=\\\{\}]
{\color{incolor}In [{\color{incolor}186}]:} \PY{n}{A}\PY{o}{.}\PY{n}{nullspace}\PY{p}{(}\PY{p}{)}                              \PY{c}{# N(A)}
\end{Verbatim}
\texttt{\color{outcolor}Out[{\color{outcolor}186}]:}
    
    
        \begin{equation*}\adjustbox{max width=\hsize}{$
        \left [ \left[\begin{matrix}0\\-3\\2\\1\end{matrix}\right]\right ]
        $}\end{equation*}

    

    \subsubsection{Determinants}\label{determinants}

    The determinant of a matrix, denoted $\det(A)$ or $|A|$, is a particular
way to multiply the entries of the matrix to produce a single number.

    \begin{Verbatim}[commandchars=\\\{\}]
{\color{incolor}In [{\color{incolor}187}]:} \PY{n}{M} \PY{o}{=} \PY{n}{Matrix}\PY{p}{(} \PY{p}{[}\PY{p}{[}\PY{l+m+mi}{1}\PY{p}{,} \PY{l+m+mi}{2}\PY{p}{,} \PY{l+m+mi}{3}\PY{p}{]}\PY{p}{,} 
                       \PY{p}{[}\PY{l+m+mi}{2}\PY{p}{,}\PY{o}{-}\PY{l+m+mi}{2}\PY{p}{,} \PY{l+m+mi}{4}\PY{p}{]}\PY{p}{,}
                       \PY{p}{[}\PY{l+m+mi}{2}\PY{p}{,} \PY{l+m+mi}{2}\PY{p}{,} \PY{l+m+mi}{5}\PY{p}{]}\PY{p}{]} \PY{p}{)}
          \PY{n}{M}\PY{o}{.}\PY{n}{det}\PY{p}{(}\PY{p}{)}
\end{Verbatim}
\texttt{\color{outcolor}Out[{\color{outcolor}187}]:}
    
    
        \begin{equation*}\adjustbox{max width=\hsize}{$
        2
        $}\end{equation*}

    

    Determinants are used for all kinds of tasks: to compute areas and
volumes, to solve systems of equations, and to check whether a matrix is
invertible or not.

    \subsubsection{Matrix inverse}\label{matrix-inverse}

    For every invertible matrix $A$, there exists an inverse matrix $A^{-1}$
which \emph{undoes} the effect of $A$. The cumulative effect of the
product of $A$ and $A^{-1}$ (in any order) is the identity matrix:
$AA^{-1}= A^{-1}A=\mathbb{1}$.

    \begin{Verbatim}[commandchars=\\\{\}]
{\color{incolor}In [{\color{incolor}188}]:} \PY{n}{A} \PY{o}{=} \PY{n}{Matrix}\PY{p}{(} \PY{p}{[}\PY{p}{[}\PY{l+m+mi}{1}\PY{p}{,}\PY{l+m+mi}{2}\PY{p}{]}\PY{p}{,} 
                       \PY{p}{[}\PY{l+m+mi}{3}\PY{p}{,}\PY{l+m+mi}{9}\PY{p}{]}\PY{p}{]} \PY{p}{)} 
          \PY{n}{A}\PY{o}{.}\PY{n}{inv}\PY{p}{(}\PY{p}{)}
\end{Verbatim}
\texttt{\color{outcolor}Out[{\color{outcolor}188}]:}
    
    
        \begin{equation*}\adjustbox{max width=\hsize}{$
        \left[\begin{matrix}3 & - \frac{2}{3}\\-1 & \frac{1}{3}\end{matrix}\right]
        $}\end{equation*}

    

    \begin{Verbatim}[commandchars=\\\{\}]
{\color{incolor}In [{\color{incolor}189}]:} \PY{n}{A}\PY{o}{.}\PY{n}{inv}\PY{p}{(}\PY{p}{)}\PY{o}{*}\PY{n}{A}
\end{Verbatim}
\texttt{\color{outcolor}Out[{\color{outcolor}189}]:}
    
    
        \begin{equation*}\adjustbox{max width=\hsize}{$
        \left[\begin{matrix}1 & 0\\0 & 1\end{matrix}\right]
        $}\end{equation*}

    

    \begin{Verbatim}[commandchars=\\\{\}]
{\color{incolor}In [{\color{incolor}190}]:} \PY{n}{A}\PY{o}{*}\PY{n}{A}\PY{o}{.}\PY{n}{inv}\PY{p}{(}\PY{p}{)}
\end{Verbatim}
\texttt{\color{outcolor}Out[{\color{outcolor}190}]:}
    
    
        \begin{equation*}\adjustbox{max width=\hsize}{$
        \left[\begin{matrix}1 & 0\\0 & 1\end{matrix}\right]
        $}\end{equation*}

    

    The matrix inverse $A^{-1}$ plays the role of division by $A$.

    \subsubsection{Eigenvectors and
eigenvalues}\label{eigenvectors-and-eigenvalues}

    When a matrix is multiplied by one of its eigenvectors the output is the
same eigenvector multiplied by a constant
$A\vec{e}_\lambda =\lambda\vec{e}_\lambda$. The constant $\lambda$ (the
Greek letter \emph{lambda}) is called an \emph{eigenvalue} of $A$.

To find the eigenvalues of a matrix, start from the definition
$A\vec{e}_\lambda =\lambda\vec{e}_\lambda$, insert the identity
$\mathbb{1}$, and rewrite it as a null-space problem:

\[
A\vec{e}_\lambda =\lambda\mathbb{1}\vec{e}_\lambda
\qquad
\Rightarrow
\qquad
\left(A - \lambda\mathbb{1}\right)\vec{e}_\lambda = \vec{0}.
\]

This equation will have a solution whenever
$|A - \lambda\mathbb{1}|=0$.(The invertible matrix theorem states that a
matrix has a non-empty null space if and only if its determinant is
zero.) The eigenvalues of $A \in \mathbb{R}^{n \times n}$, denoted
$\{ \lambda_1, \lambda_2, \ldots, \lambda_n \}$,\\are the roots of the
\emph{characteristic polynomial} $p(\lambda)=|A - \lambda \mathbb{1}|$.

    \begin{Verbatim}[commandchars=\\\{\}]
{\color{incolor}In [{\color{incolor}191}]:} \PY{n}{A} \PY{o}{=} \PY{n}{Matrix}\PY{p}{(} \PY{p}{[}\PY{p}{[} \PY{l+m+mi}{9}\PY{p}{,} \PY{o}{-}\PY{l+m+mi}{2}\PY{p}{]}\PY{p}{,}
                       \PY{p}{[}\PY{o}{-}\PY{l+m+mi}{2}\PY{p}{,}  \PY{l+m+mi}{6}\PY{p}{]}\PY{p}{]} \PY{p}{)}
          \PY{n}{A}\PY{o}{.}\PY{n}{eigenvals}\PY{p}{(}\PY{p}{)}  \PY{c}{# same as solve(det(A-eye(2)*x), x)}
                         \PY{c}{# return eigenvalues with their multiplicity}
\end{Verbatim}
\texttt{\color{outcolor}Out[{\color{outcolor}191}]:}
    
    
        \begin{equation*}\adjustbox{max width=\hsize}{$
        \left \{ 5 : 1, \quad 10 : 1\right \}
        $}\end{equation*}

    

    \begin{Verbatim}[commandchars=\\\{\}]
{\color{incolor}In [{\color{incolor}192}]:} \PY{n}{A}\PY{o}{.}\PY{n}{eigenvects}\PY{p}{(}\PY{p}{)}
\end{Verbatim}
\texttt{\color{outcolor}Out[{\color{outcolor}192}]:}
    
    
        \begin{equation*}\adjustbox{max width=\hsize}{$
        \left [ \left ( 5, \quad 1, \quad \left [ \left[\begin{matrix}\frac{1}{2}\\1\end{matrix}\right]\right ]\right ), \quad \left ( 10, \quad 1, \quad \left [ \left[\begin{matrix}-2\\1\end{matrix}\right]\right ]\right )\right ]
        $}\end{equation*}

    

    Certain matrices can be written entirely in terms of their eigenvectors
and their eigenvalues. Consider the matrix $\Lambda$ (capital Greek
\emph{L}) that has the eigenvalues of the matrix $A$ on the diagonal,
and the matrix $Q$ constructed from the eigenvectors of $A$ as columns:

\[
\Lambda = 
\begin{bmatrix}
\lambda_1   &  \cdots  &  0 \\
\vdots  &  \ddots  &  0  \\
0   &   0      &  \lambda_n
\end{bmatrix}\!,
\ \ 
Q \: = 
\begin{bmatrix}
|  &  & | \\
\vec{e}_{\lambda_1}  & \!  \cdots \! &  \large\vec{e}_{\lambda_n} \\
|  &  & | 
\end{bmatrix}\!,
\ \ 
\textrm{then}
\ \ 
A = Q \Lambda Q^{-1}.
\]

Matrices that can be written this way are called \emph{diagonalizable}.
To \emph{diagonalize} a matrix $A$ is to find its $Q$ and $\Lambda$
matrices:

    \begin{Verbatim}[commandchars=\\\{\}]
{\color{incolor}In [{\color{incolor}193}]:} \PY{n}{Q}\PY{p}{,} \PY{n}{L} \PY{o}{=} \PY{n}{A}\PY{o}{.}\PY{n}{diagonalize}\PY{p}{(}\PY{p}{)}
          \PY{n}{Q}            \PY{c}{# the matrix of eigenvectors as columns }
\end{Verbatim}
\texttt{\color{outcolor}Out[{\color{outcolor}193}]:}
    
    
        \begin{equation*}\adjustbox{max width=\hsize}{$
        \left[\begin{matrix}1 & -2\\2 & 1\end{matrix}\right]
        $}\end{equation*}

    

    \begin{Verbatim}[commandchars=\\\{\}]
{\color{incolor}In [{\color{incolor}194}]:} \PY{n}{Q}\PY{o}{.}\PY{n}{inv}\PY{p}{(}\PY{p}{)}
\end{Verbatim}
\texttt{\color{outcolor}Out[{\color{outcolor}194}]:}
    
    
        \begin{equation*}\adjustbox{max width=\hsize}{$
        \left[\begin{matrix}\frac{1}{5} & \frac{2}{5}\\- \frac{2}{5} & \frac{1}{5}\end{matrix}\right]
        $}\end{equation*}

    

    \begin{Verbatim}[commandchars=\\\{\}]
{\color{incolor}In [{\color{incolor}195}]:} \PY{n}{L}            \PY{c}{# the matrix of eigenvalues}
\end{Verbatim}
\texttt{\color{outcolor}Out[{\color{outcolor}195}]:}
    
    
        \begin{equation*}\adjustbox{max width=\hsize}{$
        \left[\begin{matrix}5 & 0\\0 & 10\end{matrix}\right]
        $}\end{equation*}

    

    \begin{Verbatim}[commandchars=\\\{\}]
{\color{incolor}In [{\color{incolor}196}]:} \PY{n}{Q}\PY{o}{*}\PY{n}{L}\PY{o}{*}\PY{n}{Q}\PY{o}{.}\PY{n}{inv}\PY{p}{(}\PY{p}{)}  \PY{c}{# eigendecomposition of A}
\end{Verbatim}
\texttt{\color{outcolor}Out[{\color{outcolor}196}]:}
    
    
        \begin{equation*}\adjustbox{max width=\hsize}{$
        \left[\begin{matrix}9 & -2\\-2 & 6\end{matrix}\right]
        $}\end{equation*}

    

    \begin{Verbatim}[commandchars=\\\{\}]
{\color{incolor}In [{\color{incolor}197}]:} \PY{n}{Q}\PY{o}{.}\PY{n}{inv}\PY{p}{(}\PY{p}{)}\PY{o}{*}\PY{n}{A}\PY{o}{*}\PY{n}{Q}  \PY{c}{# obtain L from A and Q}
\end{Verbatim}
\texttt{\color{outcolor}Out[{\color{outcolor}197}]:}
    
    
        \begin{equation*}\adjustbox{max width=\hsize}{$
        \left[\begin{matrix}5 & 0\\0 & 10\end{matrix}\right]
        $}\end{equation*}

    

    Not all matrices are diagonalizable. You can check if a matrix is
diagonalizable by calling its \texttt{is\_diagonalizable} method:

    \begin{Verbatim}[commandchars=\\\{\}]
{\color{incolor}In [{\color{incolor}198}]:} \PY{n}{A}\PY{o}{.}\PY{n}{is\PYZus{}diagonalizable}\PY{p}{(}\PY{p}{)}
\end{Verbatim}

            \begin{Verbatim}[commandchars=\\\{\}]
{\color{outcolor}Out[{\color{outcolor}198}]:} True
\end{Verbatim}
        
    \begin{Verbatim}[commandchars=\\\{\}]
{\color{incolor}In [{\color{incolor}199}]:} \PY{n}{B} \PY{o}{=} \PY{n}{Matrix}\PY{p}{(} \PY{p}{[}\PY{p}{[}\PY{l+m+mi}{1}\PY{p}{,} \PY{l+m+mi}{3}\PY{p}{]}\PY{p}{,}
                      \PY{p}{[}\PY{l+m+mi}{0}\PY{p}{,} \PY{l+m+mi}{1}\PY{p}{]}\PY{p}{]} \PY{p}{)}
          \PY{n}{B}\PY{o}{.}\PY{n}{is\PYZus{}diagonalizable}\PY{p}{(}\PY{p}{)}
\end{Verbatim}

            \begin{Verbatim}[commandchars=\\\{\}]
{\color{outcolor}Out[{\color{outcolor}199}]:} False
\end{Verbatim}
        
    \begin{Verbatim}[commandchars=\\\{\}]
{\color{incolor}In [{\color{incolor}200}]:} \PY{n}{B}\PY{o}{.}\PY{n}{eigenvals}\PY{p}{(}\PY{p}{)}  \PY{c}{# eigenvalue 1 with multiplicity 2}
\end{Verbatim}
\texttt{\color{outcolor}Out[{\color{outcolor}200}]:}
    
    
        \begin{equation*}\adjustbox{max width=\hsize}{$
        \left \{ 1 : 2\right \}
        $}\end{equation*}

    

    \begin{Verbatim}[commandchars=\\\{\}]
{\color{incolor}In [{\color{incolor}201}]:} \PY{n}{B}\PY{o}{.}\PY{n}{eigenvects}\PY{p}{(}\PY{p}{)}
\end{Verbatim}
\texttt{\color{outcolor}Out[{\color{outcolor}201}]:}
    
    
        \begin{equation*}\adjustbox{max width=\hsize}{$
        \left [ \left ( 1, \quad 2, \quad \left [ \left[\begin{matrix}1\\0\end{matrix}\right]\right ]\right )\right ]
        $}\end{equation*}

    

    The matrix $B$ is not diagonalizable because it doesn't have a full set
of eigenvectors. To diagonalize a $2\times 2$ matrix, we need two
orthogonal eigenvectors but $B$ has only a single eigenvector.
Therefore, we can't construct the matrix of eigenvectors $Q$ (we're
missing a column!) and so $B$ is not diagonalizable.

Non-square matrices don't have eigenvectors and therefore don't have an
eigendecomposition. Instead, we can use the \emph{singular value
decomposition} to break up a non-square matrix $A$ into left singular
vectors, right singular vectors, and a diagonal matrix of singular
values. Use the \texttt{singular\_values} method on any matrix to find
its singular values.


    % Add a bibliography block to the postdoc
    
    
    
    \end{document}
